\nonstopmode

\documentclass[a4paper, 12pt, oneside]{ncc}
\usepackage[warn]{mathtext}          % русские буквы в формулах, с предупреждением
\usepackage{ucs}                     % Unicode input support
\usepackage[T2A]{fontenc}            % внутренняя кодировка  TeX
\usepackage[utf8x]{inputenc}         % кодовая страница документа
\usepackage[english, russian]{babel} % локализация и переносы
\usepackage{indentfirst}             % русский стиль: отступ первого абзаца раздела
\usepackage{misccorr}                % точка в номерах заголовков
\usepackage{cmap}                    % русский поиск в pdf
\usepackage{graphicx}                % Работа с графикой \includegraphics{}
\usepackage{caption}                 % Работа с подписями для фигур, таблиц и пр.
\usepackage{soul}                    % Разряженный текст \so{} и подчеркивание \ul{}
\usepackage{soulutf8}                % Поддержка UTF8 в soul
\usepackage{fancyhdr}                % Для работы с колонтитулами
\usepackage{multirow}                % Аналог multicolumn для строк
\usepackage{ltxtable}                % Микс tabularx и longtable
\usepackage{paralist}                % Списки с отступом только в первой строчке
\usepackage[shortcuts]{extdash}      % Переносы в словах с дефисами
\usepackage[perpage]{footmisc}       % Нумерация сносок на каждой странице с 1
\usepackage{amsmath}
\usepackage{amsfonts}
\usepackage{amssymb}
% Задаем отступы: слева 30 мм, справа 10 мм, сверху до колонтитула 10 мм снизу 25 мм
\usepackage[a4paper, top=20mm, left=30mm, right=10mm, bottom=25mm]{geometry}
\usepackage{xcolor}
\usepackage{hyperref}

% цвета для гиперссылок
\definecolor{linkcolor}{HTML}{340DAB} % цвет ссылок
\definecolor{urlcolor}{HTML}{340DAB} % цвет гиперссылок

\hypersetup{unicode=true, linkcolor=linkcolor, urlcolor=urlcolor, colorlinks=true, breaklinks=true}

\def\EUR{\,\euro}

\pagestyle{fancy}
\fancyhead{}
\fancyhead[LE]{\textit{ Конференция <<Ломоносов 2021>>}}
\fancyhead[LO]{\textit{ Конференция <<Ломоносов 2021>>}}
\fancyfoot{}
\fancyfoot[RE,RO]{\thepage}
\renewcommand{\headrulewidth}{0pt}
\renewcommand{\footrulewidth}{0pt}

\title{ Типичные особенности интегрируемых гамильтоновых систем }
\author{ Онуфриенко Мария Викторовна }

\begin{document}
\begin{flushright}
Секция <<Геометрия и топология>>
\end{flushright}

\begin{center}
\textbf{Типичные особенности интегрируемых гамильтоновых систем}
\end{center}

\begin{center}
\textbf{Научный руководитель -- Кудрявцева Елена Александровна  }\\
\vspace{0.2cm}
\textbf{\textit{Онуфриенко Мария Викторовна}}\\
\textit{Студент (специалист)}\\
Московский государственный университет имени М.В.Ломоносова, Механико-математический факультет, Кафедра дифференциальной геометрии и приложений, Москва, Россия\\
\textit{E-mail: mary.onufrienko@gmail.com}
\\
\end{center}

Фиксируем любое $s\in\mathbb N$ и рассмотрим действие группы $G=\mathbb{Z}_s$ на плоскости $\mathbb{R}^2$ вида 
$z\longrightarrow e^{2\pi i/{s}}z,$ где $z=x+iy\in\mathbb C\approx\mathbb{R}^2.$
Рассмотрим морсовские функции $g_0=g_0^{\pm,\pm}(z)=\pm|z|^2=\pm (x^2+y^2)$ при любом $s\ge1$, и $g_0=g_0^{+,-}(z)=x^2-y^2$ при $s=1,2$. 

Рассмотрим два семейства $\mathbb{Z}_s-$инвариантных ростков $g_k=g_k(z,\lambda,a)$, $k=1,2$, в нуле:
$$
g_1=g_1(z,\lambda,a)=\left\{
\begin{array}{ll}
\pm x^2+y^3+\lambda y, & s=1, \\
\pm x^2\pm y^4 +\lambda y^2, & s=2, \\
Re(z^3)+\lambda|z|^2, & s=3, \\
Re(z^s)\pm a|z|^4+\lambda |z|^2, & s\ge4,\ a^2\neq 1 \mbox{ при }s=4,\ a>0 \mbox{ при }s\ge5,
\end{array}
\right.
%\eqno(2)
$$
$$
g_2=g_2(z,\lambda,a)=\left\{
\begin{array}{ll}
\pm x^2\pm y^4-\lambda_2 y^2+\lambda_1 y, & s=1, \\
\pm x^2\pm y^6 +\lambda_2 y^4+\lambda_1 y^2, & s=2, \\
Re(z^4)\pm(1+\lambda_2)|z|^4 \pm a|z|^6+\lambda_1 |z|^2, & s=4,\ a>0, \\
Re(z^5)\pm a|z|^6+\lambda_2|z|^4+\lambda_1 |z|^2, & s=5,\ a>0, \\
Re(z^6)+a_1 |z|^6\pm a_2 |z|^8+\lambda_2|z|^4+\lambda_1 |z|^2, & s=6,\ a_1^2\ne1, a_1a_2\ne0.
\end{array}
\right.
$$
Здесь $\lambda\in\mathbb{R}^k$ --- малый параметр, $a\in\mathbb{R}^m$ --- <<модуль>>, $m\in\{0,1,2\}$ --- <<модальность>>.

\smallskip
\textbf{Теорема 1. } {\sl
%Рассмотрим действие группы $G=\mathbb{Z}_s$ на плоскости $\mathbb{R}^2$ вида {\rm (1)}, $s\in\mathbb{N}$. 
Рассмотрим классы {\em правой} $G-$эквивалентности $G-$инва\-ри\-ант\-ных ростков функций $g_k(z,0,\hat a)$ двух переменных в нуле, $k=0,1,2$.
Эти особенности имеют $G-$кораз\-мер\-ность $k$, $G-$кратность Милнора $k+m+1$, $G-$версальную деформацию $g_k(z,\lambda,a)+\lambda_0$, и образуют полный список $G-$инва\-ри\-ант\-ных особенностей $G-$ко\-раз\-мер\-ности $k\le 2$. Дополнение к их объединению в множестве $\mathfrak{n}^2_G
%:=\mathfrak{m}^2\cap \mathcal{E}_G
$
$G-$инва\-ри\-ант\-ных ростков в нуле, имеющих критическую точку 0 с критическим значением 0, имеет коразмерность $>2$ в $\mathfrak{n}_G^2$.}

\smallskip
Теорема 1 не следует из классификации [1] особенностей $G-$кратности Милнора $\le 5$. 

{\em Интегрируемая система} на $2n$-мерном симплектическом многообразии $(M,\Omega)$ задается гладким отображением $F=(f_1,\dots,f_n):\ M \longrightarrow{\mathbb R}^n,$ где $\{f_i,f_j\}=0$. Возникает {\em лагранжево слоение с особенностями} на $M$, слои которого --- это связные компоненты множеств $F^{-1}(c)$. Пусть $M$ компактно, поля $X_{f_j}$ касаются $\partial M$, и $F$ имеет <<хорошее>> поведение около $\partial M$. Отображение $F$ порождает гамильтоново $\mathbb{R}^n-$действие на $M$.

Рассмотрим свободное действие группы $\mathbb{Z}_s$ на полнотории $V:=D^2\times S^1\subset\mathbb{R}^2\times S^1$ вида 
$(z,\varphi_1)\longrightarrow (e^{2\pi \ell i/{s}}z,\varphi_1+\frac{2\pi}{s}),$ 
где $\varphi_1\in S^1=\mathbb R/(2\pi\mathbb Z),
%\eqno(2)
$
$0\le\ell<s$, $(\ell,s)=1$. Рассмотрим <<цилиндр>> $W:=D^{n-1}\times(S^1)^{n-2}$ с координатами $(\lambda,\varphi')=(\lambda_1,\dots,\lambda_{n-1},\varphi_2,\dots,\varphi_{n-1})$.

Оказывается, локальные особенности (т.е.\ $\mathbb{R}^n-$орбиты) коранга 1 типичных интегрируемых систем имеют окрестности, послойно диффеоморфные {\em стандартной модели} вида
$$
F_{st}:(V/\mathbb{Z}_s)\times W \to \mathbb R^n, \
F_{st}(z,\varphi_1,\lambda,\varphi')=(g_k(z,\lambda',a(\lambda)),\lambda), \quad
\Omega_{st} = \mathrm dx\wedge\mathrm dy + \sum\limits_{j=1}^{n-1}\mathrm d\lambda_j\wedge\mathrm  d\varphi_j,
$$
где $0\le k<n$, $\lambda'=(\lambda_1,\dots,\lambda_k)$,
$a(\lambda)$ и $a_1(\lambda)$ --- гладкие функции при $(k,s)\in\{(1,4),(2,6)\}$,
$a(\lambda)\equiv1$ для остальных пар $(k,s)$; 
$a(\lambda)=(a_1(\lambda),1)$ при $(k,s)=(2,6)$.

\textbf{Теорема 2 (Кудрявцева Е.\,А., Онуфриенко М.\,В.). } {\sl Пусть $n=\frac12\dim M\in\{2,3\}$. Рассмотрим класс $\mathcal I=\mathcal I(M)$ интегрируемых систем на $M$, для которых функции $f_2,\dots,f_n$ порождают локально-свободное гамильтоново действие $(n-1)-$мерного тора на $M$. Если некоторая окрестность $\mathbb{R}^n-$орбиты послойно диффеоморфна стандартной модели, то эта орбита структурно устойчива относительно возмущений в классе $\mathcal I$. Если орбита структурно устойчива, то некоторая ее окрестность послойно гомеоморфна стандартной модели. Класс $\mathcal I_{st}\subset\mathcal I$ систем, все локальные особенности которых послойно диффеоморфны стандартным, открыт в $\mathcal I$ (относительно $C^\infty-$топологии), и $\mathcal I\setminus\mathcal I_{st}$ имеет коразмерность $>0$.}

\smallskip
Теорема 2 при $n=2$, $k=1$ описывает параболические траектории с резонансами [2], а при $n=3$, $k=2$ --- их типичные бифуркации.

\begin{center}\textbf{Источники и литература}\end{center}
\begin{enumerate}
\item {\em Wassermann G.} Classification of singularities with compact Abelian symmetry // Singularities Banach Center Publications. 1988. V.~20. P.~475--498.{\sloppy

}
\item {\em Калашников В.В.} Типичные интегрируемые гамильтоновы системы на четырехмерном симплектическом многообразии // Изв.\ РАН, Сер.\ матем. 1998. Т.~62. No.~2. С.~49--74.{\sloppy

}
\end{enumerate}
\\


\end{document}