\documentclass[12pt,a4paper]{article}
\usepackage[utf8]{inputenc}
\usepackage[english,russian]{babel}
\usepackage{indentfirst}
\usepackage{misccorr}
\usepackage{graphicx}
\usepackage{amsmath}

%\graphicspath{{pictures/}}
\DeclareGraphicsExtensions{.pdf,.png,.jpg, .mps, }


\begin{document}


{\bf О некоторых принципах построения квадратурных формул}

\section {Обычные квадратуры}

Пусть в силу некоторых причин (например, интересующий нас интеграл не берется в квадратурах) требуется приближенно вычислить определенный интеграл 
$$I(f)\equiv \int_{a}^{b} {p(x)f(x)dx}$$
Функция $p(x)$ называется весовой функцией. Выделение весовой функции в подынтегральном выражении может быть удобно по нескольким причинам. Одна из них заключается в том, что при наличии особенностей у подынтегральной функции, их можно объявить весовой функцией, и тогда сама функция $f(x)$ станет гладкой, что даст возможность написать приемлемую оценку для погрешности.\\
Первый и самый очевидный способ построения квадратурной формулы, называемой иначе квадратурой, заключается в том, чтобы вместо функции $f(x)$ использовать ее приближение многочленом Лагранжа. А именно: фиксируем  узлы $x_1, ..., x_n \in [a,b]$, строим по ним интерполяционный многочлен Лагранжа $L_n(x)$ и полагаем 
$$S_n(f):= \int_{a}^{b} {p(x)L_n(x)dx} \eqno(1)$$
Тогда $$S_n(f)=\int_{a}^{b} {p(x)L_n(x)dx}=\int_{a}^{b} {p(x)\sum_{j=1}^{n} f(x_j)\Phi_j(x)dx} = \sum_{j=1}^{n}  \int_{a}^{b} {p(x)\Phi_j(x)dx}  f(x_j)=\sum_{j=1}^{n} c_jf(x_j),$$
где $$c_j=\int_{a}^{b} {p(x)\Phi_j(x)dx}, \eqno(2)$$
а $$\Phi_j(x)=\frac{(x-x_1)...(x-x_n)}{(x-x_j)\prod_{1\leq i \neq j \leq n} (x_j-x_i)}$$

Отметим, что найти коэффициент $c_j$ в квадратуре можно не только по формуле $(2)$, но и другим способом. Для этого заметим, что по построению квадратура точна для всех многочленов степени не выше $n$, поэтому коэффициенты  $c_j$ можно было искать из условия точности квадратуры на мономах $1, x, x^2, ..., x^{n-1}$, то есть составить и решить  систему $$\int_{a}^{b} p(x)dx = \sum_{j=1}^{n} c_j$$ $$\int_{a}^{b} p(x)xdx = \sum_{j=1}^{n} c_jx_j$$ $$...$$ $$\int_{a}^{b} p(x)x^{n-1}dx = \sum_{j=1}^{n} c_jx_j^{n-1}$$
Решение у такой системы существует и единственно, поскольку ее определителем является определитель Вандермонда, который, как известно, не равен нулю, если узлы различны.
Покажем, что коэффициенты $c_j$, полученные первым и вторым способом, действительно совпадут.\\
В самом деле, пусть коэффициенты $b_j$ получены как решение системы уравнений, то есть полученная квадратура точна на любых мономах $1, x, x^2, ..., x^{n-1},$ а значит, точна и на любом многочлене степени не выше $n-1$, благодаря аддитивности как интеграла, так и квадратуры. Тогда подставим в качестве многочлена $\Phi_j(x)$, и воспользуемся тем, что на нем квадратура точна, ведь этот многочлен степени ровно $n-1$.
Получим $$c_j=\int_{a}^{b} {p(x)\Phi_j(x)dx}=\sum_{j=1}^{n} b_j\Phi_j(x_i)=b_j$$

Второй способ хорош тем, что из условия точности квадратуры на мономах степени не выше $m$ легко выводится оценка погрешности квадратуры.

{\bf Теорема 1} Пусть  узлы и коэффициенты $\{c_i, x_i\}_{i=1]^{n} $таковы, что $I(P_m)=S_n(P_m) \forall P_m(x).$
Тогда $$\left| I(f)-S_n(f) \right| \leq \frac{\left|| f^{(m+1)} \right\\}{(m+1)!} \left( \int_{a}^{b} |p(x)|dx + \sum_{i=1}^{n} |c_i| \right) \cdot 2^{1-(m+1)} \cdot \left( \frac{b-a}{2} \right)^{m+1}$$

Приведем несколько примеров, демонстрирующих, как находить коэффициенты и погрешность квадратуры.\\
{\bf Пример 1} 
$n=1, p\equiv 1, x_1=\frac{a+b}{2}.$
Тогда квадратура имеет вид $S_1(f)=c_1f(x_1)$, и решая систему из одного уравнения $b-a=\int_{a}^{b}dx = c_1$, находим коэффициент $c_1=b-a$, то есть квадратура имеет вид $$S_1(f)=(b-a)f\left(\frac{a+b}{2}\right)$$
Она называется формулой прямоугольников по центральной точке.
По построению эта квадратура точна для многочленов степени  $m=0$, но из-за удачно выбранного узла она оказывается точна и для многочленов степени $m=1$. А именно, произвольный многочлен степени $m=1$ можно представить в виде $f(x)=\alpha\left(x-\frac{a+b}{2}\right)+\beta.$ Тогда $f\left(\frac{a+b}{2}\right)=\beta.$
Поэтому $$ S_1(f)=(b-a)f\left(\frac{a+b}{2}\right)=(b-a)\beta=0+(b-a)\beta=\int_{a}^{b} \left[ \alpha\left(x-\frac{a+b}{2}\right)+\beta \right] dx = \int_{a}^{b} f(x)dx.$$
Мы видим, что квадратура действительно точна для многочленов степени $m=1$.

\end{document}