\documentclass[a4paper,11pt,english]{article}
\usepackage[utf8]{inputenc}
\usepackage{babel,amsmath,amssymb,amsthm,enumitem}
\usepackage[sort]{natbib}
\usepackage[colorlinks,allcolors=blue]{hyperref}
\usepackage[margin=1.25in]{geometry}
\usepackage[small,bf]{titlesec}
\titlelabel{\thetitle.\hspace{0.5em}}
\PassOptionsToPackage{usenames,dvipsnames}{xcolor}
\usepackage{tikz}

\newtheorem{theorem}{Theorem}
\newtheorem{lemma}{Lemma}
\newtheorem{corollary}{Corollary}
\newtheorem{proposition}{Proposition}
\theoremstyle{definition}
\newtheorem{remark}{Remark}
\newtheorem{definition}{Definition}

\newcommand{\floor}[1]{\lfloor{#1}\rfloor}
\newcommand{\ceil}[1]{\lceil{#1}\rceil}
\DeclareMathOperator{\E}{E}
\DeclareMathOperator{\Var}{Var}
\DeclareMathOperator{\I}{I}
\DeclareMathOperator{\sgn}{sgn}
\DeclareMathOperator{\Law}{Law}
\DeclareMathOperator{\argmax}{arg\,max}
\renewcommand{\hat}{\widehat}
\renewcommand{\tilde}{\widetilde}
\renewcommand{\epsilon}{\varepsilon}
\renewcommand{\P}{\mathrm{P}}
\newcommand{\F}{\mathcal{F}}
\newcommand{\R}{\mathbb{R}}
\newcommand{\bt}{\tilde{b}}
\renewcommand{\b}{\bar b}
\newcommand{\M}{\mathcal{M}}
\let\Ss\S
\renewcommand{\S}{\mathcal{S}}
\newcommand{\cadlag}{c\`adl\`ag}

\newcommand{\red}[1]{{\color{red}#1}}

\title{An evolutionary finance model with short-lived assets\\ and
affine payoffs}
\author{}
\date{}

\begin{document}
\maketitle

\section{Introduction}

\section{The model}

Let $(\Omega,\F,\P)$ be a probability space with a complete discrete-time
 filtration $\mathbb{F} = (\F_t)_{t\ge 0}$.
The market in the model consists of $M\ge 2$ agents and $N\ge 2$ short-lived assets. Agent
$m$, where $m=1,\ldots,M$, is  characterized by her wealth $W^m$ and her
strategy $\Lambda^m$. The wealth $W^m = (W_t^m)_{t=0}^\infty$ is an
$\mathbb{F}$-adapted random sequence, where $W_t^m=W_t^m(\omega)$ is the wealth
held by the agent at time $t$. The variables $W_t^m$  depend on the payoffs yielded by the assets and the
strategies of the other agents through a certain equation which will be
stated later. The total market wealth at time $t$ will be denoted by $C_t =
W_t^1+\ldots+W_t^M$. 

A strategy of agent $m$ is a sequence $\Lambda^m =
(\Lambda_t^m)_{t=0}^\infty$ of vector-valued functions $\Lambda_t^m =
\Lambda_t^m(\omega,c)$ with values in the standard $N$-simplex $\Delta_N
=\{\lambda \in \R^N_+ : \lambda^1+\ldots+\lambda^N = 1\}$ and measurable with
respect to $\F_t\otimes \mathcal{B}(\R_+)$. For $n=1,\ldots,N$, the $n$-th
coordinate $\Lambda_t^{m,n}$ specifies the proportion of wealth which agent
$m$ allocates for buying asset $n$ at time $t$. Short sales are not allowed.
A strategy may depend on a random state of the world $\omega$ and the total
market wealth $C_t$ (through the argument $c$ of $\Lambda_t^m$), so that if
the total market wealth at state $\omega$ is $C_t(\omega)$, then agent $m$
allocates her wealth in proportions given by the vector $\Lambda_t^m(\omega,
C_t(\omega))$.

It is possible to consider strategies of a more general form, for example
depending on full market history, but this will not increase the generality
of the main results. On the other hand, the dependence on the total market
wealth is necessary and cannot be removed from the model because the
survival strategy that we construct below needs it.

Each asset $n=1,\ldots,N$ at each moment of time $t\ge 1$ yields a random
payoff which depends on the total amount of wealth invested by the agents in
this asset at time $t-1$ according to the formula
\begin{equation}
P_t^n(\omega) = X_t^n(\omega) \sum_{m=1}^M
\lambda_{t-1}^{m,n}(\omega)W_{t-1}^m(\omega) + Y_t^n(\omega),\label{P}
\end{equation}
where $X^n = (X_t^n)_{t=1}^\infty$ and $Y^n = (Y_t^n)_{t=1}^\infty$ are some exogenously
given non-negative random sequence adapted to the filtration $\mathbb{F}$, and
\[
\lambda_{t-1}^{m,n}(\omega) = \Lambda_{t-1}^{m,n}(\omega,
C_{t-1}(\omega)),\qquad C_{t-1}(\omega) = \sum_{m=1}^M W_{t-1}^m(\omega).
\]
We will assume that for each $t\ge 0$ and
$n=1,\ldots,N$ it holds that
\begin{equation}
\P(Y_{t+1}^n >0 \mid \F_t) >0\ \text{a.s.}\label{Y-positive}
\end{equation}

The payoff of each asset is divided between the agents proportionally to the amount of
wealth they allocated for investing in this asset at time $t-1$. As a result, the wealth
sequence of agent $m$ satisfies the recursive relation
\begin{equation}
W_{t+1}^m = \sum_{n=1}^N \frac{\lambda_t^{m,n} W_t^m}{\sum_{k=1}^M
\lambda_t^{k,n} W_t^k} P_{t+1}^n = \sum_{n=1}^N\lambda_t^{m,n} W_t^m
\biggl(X_{t+1}^n + \frac{Y_{t+1}^n}{\sum_{k=1}^M \lambda_t^{k,n} W_t^k}
\biggr).\label{capital}
\end{equation}
It is clear that given an initial condition $W_0=(W_0^1,\ldots,W_0^M)$ and a
strategy profile $\Lambda=(\Lambda^1,\ldots,\Lambda^M)$, the sequence
$W_t=(W_t^1,\ldots,W_t^M)$ is well-defined by the above relation, provided
that for all $t \ge 0$ with probability 1 we have
\begin{equation}
\sum_{k=1}^M \lambda_t^{k,n} W_t^k \neq 0.\label{feasible}
\end{equation}
In what follows we will always assume that the objects defining the
market model (i.e.\ agents' strategies and asset payoffs) are such
that inequality (\ref{feasible}) holds true. A  sufficient condition for its
validity consists in that for each $t\ge 1$ we have, with probability 1,
\begin{equation}
\label{feasible-2}
\sum_{n=1}^N (X_t^n + Y_t^n) >0,
\end{equation}
and there is at least one agent who uses a fully diversified strategy, i.e.\
for some $m=1,\ldots,M$ and all $t\ge 0$, $n=1,\ldots, N$ we have
\begin{equation}
\label{feasible-2}
\lambda_t^{m,n} > 0.
\end{equation}
It is not difficult to see that if inequalities
\eqref{feasible-2}--\eqref{feasible-2} are true, then agent $m$ has 
strictly positive wealth at all moments of time, so \eqref{feasible} also
holds. The survival strategy $\hat\Lambda$ which we construct below will
satisfy assumption \eqref{feasible-2}.

It is easy to see that if one formally puts $Y_{t+1}^n=0$ for all $t,n$
(although our models does not permit to do so in view of assumption
(\ref{Y-positive})), the model reduces to the standard mathematical finance
model of an asset market with exogenous prices, where $X_{t+1}^n$ are equal
to relative changes of asset prices, i.e. $X_{t+1}^n = S_{t+1}^n / S_t^n$.
When $X_{t+1}^n=0$ for all $t,n$, it becomes the evolutionary finance model
of a market with short-lived assets introduced by \cite{AmirEvstigneev+13}.

\section{Survival strategies}
\begin{definition}
A strategy $\hat \Lambda$ is called \emph{survival} if for any strategy profile
$\Lambda = (\Lambda^{1},\ldots,\Lambda^M)$ with $\Lambda^1 = \hat \Lambda$ and any
initial wealth vector $W_0=(W_0^1,\ldots,W_0^M)$ with $W_0^1 > 0$ it holds that
\[
\inf_{t\ge 0} W_t^1 > 0\ \text{a.s.}
\]
\end{definition}

Denote by $C_t$ the total market wealth, and $r_t^m$ the relative wealth of agent $m$:
\[
C_t = \sum_{m=1}^M W_t^m, \qquad r_t^m = \frac{W_t^m}{C_t}.
\]
\begin{definition}
A strategy $\hat \Lambda$ is called \emph{relative growth optimal} if for any strategy profile
$\Lambda = (\Lambda^{1},\ldots,\Lambda^M)$ with $\Lambda^1 = \hat \Lambda$ and any
initial wealth vector $W_0=(W_0^1,\ldots,W_0^M)$ with $W_0^1 > 0$ it holds that $\ln r_t^1$ is a submartingale.
\end{definition}

\begin{proposition}
A relative growth optimal strategy is survival.
\end{proposition}
\begin{proof}
The result follows from that a non-positive submartingale converges to a finite limit with probability one. Hence, if agent 1 uses a relative growth optimal strategy, $\ln r_t^1$ has a finite limit as $t\to\infty$.
\end{proof}

Now we will construct a relative growth optimal strategy. Let us introduce 
notation, which will be used in what follows.

For each $t\ge 0$, let $\P_t(\omega, d\tilde \omega)$ denote
some variant of the regular conditional probability conditioned on $\F_t$,
i.e.\ $\P_t(\omega, A) = \P(A \mid \F_t)(\omega)$ a.s.\ 
for any $A\in \F$. By $\E_t$ we will denote the conditional expectation
computed with respect to $\P_t$, i.e. $(\E_t Z)(\omega) = \int_\Omega Z(\tilde \omega)
\P(\omega, d\tilde \omega)$ for any random variable $Z$ for which the
integral is well-defined.

For an $\F\otimes\mathcal{B}(\R^d)$-measurable function $f(\omega,x)$, the
conditional expectation $\E_t f$ will be interpreted as the
$\F_t\otimes\mathcal{B}(\R^d)$-measurable function
\[
(\E_t f)(\omega,x) = \int_\Omega f(\tilde \omega,x) \P(\omega,d\tilde \omega).
\]
 

\begin{lemma}
\label{lemma1}
For each $t\ge 0$, consider the $\F_t\otimes \mathcal{B}(\R_+)\otimes
\mathcal{B}(\Delta_N)$-measurable function $L_t \colon \Omega
\times \R_+ \times \Delta_N \to \Delta_N$ defined by
\[
L_t^n(\omega,c,\lambda) = \E_t\biggl(\frac{c\lambda^n
X_{t+1}^n + Y_{t+1}^n}{\sum_{i=1}^N (c \lambda^i X_{t+1}^i + Y_{t+1}^i)}\biggr).
\]
Then there exists an $\F_t\otimes\mathcal{B}(\R_+)$-measurable function $\hat
\Lambda_t(\omega,c)$ with values in $\Delta_N$ such that
\begin{equation}
L(\omega,c,\hat \Lambda_t(\omega,c)) = \hat \Lambda_t(\omega,c)\
\text{for all $\omega,c$}.\label{lambda-hat}
\end{equation}
\end{lemma}
\begin{proof}
(Not finished).
For fixed $(w,c)$, the map $\lambda\mapsto
L_t^n(\omega,c,\lambda)$ is a continuous map of the compact convex set $\Delta_N$ to
itself. (Next need to apply Brouwer's  fixed point theorem and prove the measurability.)
\end{proof}

\begin{theorem}
\label{theorem1}
The strategy $\hat \Lambda$ defined by \eqref{lambda-hat} is relative growth optimal.
\end{theorem}

The following auxiliary result will be used in the proof of
Theorem~\ref{theorem1}.

\begin{lemma}
\label{lemma-logsum}
Suppose $\alpha,\beta\in\R_+^N$ are two vectors such that
$\sum_n \alpha^n \le 1$, $\sum_n \beta^n \le 1$ and for each $n=1,\ldots,N$ it
holds that if $\beta^n=0$, then also $\alpha^n=0$. Then 
\begin{equation}
\sum_{n=1}^N\alpha^n \ln\frac{\alpha^n}{\beta^n} \ge
\frac{\|\alpha-\beta\|^2}{4} + \sum_{n=1}^N (\alpha^n-\beta^n),\label{logsum-1}
\end{equation}
where we define $\alpha^n \ln\frac{\alpha^n}{\beta^n} = 0$ if $\alpha^n=0$ or $\beta^n=0$.
\end{lemma}
\begin{proof}
We follow the lines of the proof of Lemma~2 in~\cite{AmirEvstigneev+13},
which establishes the above inequality in the case $\sum_n \alpha^n =
\sum_n \beta^n=1$.
Using that $\ln x \le 2(\sqrt x -1)$ for any $x>0$, we obtain
\[
\begin{split}
\sum_{n=1}^N \alpha^n \ln\frac{\alpha^n}{\beta^n} &= -\sum_{n\,:\,\alpha^n\neq 0}\alpha^n\ln\frac{\beta^n}{\alpha^n} \ge
2\sum_{n=1}^N (\alpha^n-\sqrt{\alpha^n\beta^n}) \\&= \sum_{n=1}^N(\sqrt{\alpha^n} -
\sqrt{\beta^n})^2 + \sum_{n=1}^N (\alpha^n - \beta^n).
\end{split}
\]
Then we can use the inequality $(\sqrt x - \sqrt y)^2 \ge (x-y)^2/4$, which
is true for any $x,y\in[0,1]$, and obtain \eqref{logsum-1}. 
\end{proof}

\begin{proof}[Proof of Theorem~\ref{theorem1}]
By the standard argument based on introduction of the representative agent, we can
assume that the number of agents in the model $N=2$.

Let $r_t$ denote the relative wealth of agent 1, i.e.\
\[
r_t = \frac{W_t^1}{C_t},
\]
where $C_t = W_t^1 + W_t^2$. 

Let $\hat\lambda_t(\omega) = \hat\Lambda_t(\omega,C_{t-1}(\omega))$ and denote
by $\bar \lambda_t = (\bar\lambda_t^1,\ldots,\bar\lambda_t^N)$ the representative investment proportions of the two
agents, i.e.
\[
\bar \lambda_t^n(\omega) = r_t(\omega) \lambda_t^{1,n}(\omega) + (1-r_t(\omega)) \lambda_t^{2,n}(\omega).
\]
Also introduce the notation
\[
\mu_t^n = \frac{\hat\lambda_t^n}{\bar \lambda_t^n}.
\]
From equation \eqref{capital}, one can see that
\[
W_{t+1}^1 = r_t \sum_{n=1}^N ( C_t\hat\lambda_t^{n} X_{t+1}^n  + \mu_t^n Y_{t+1}^n), \qquad C_{t+1} = \sum_{n=1}^N
(C_t\bar \lambda_t^n X_{t+1}^n + Y_{t+1}^n).
\]
From these two relations, we find
\[
\ln r_{t+1} - \ln r_t = \ln \biggl(\sum_{n=1}^N (C_t\hat\lambda_t^{n} X_{t+1}^n
 + \mu_t^n Y_{t+1}^n)\biggr) - \ln\biggl(\sum_{n=1}^N (C_t \bar\lambda_t^n
 X_{t+1}^n + Y_{t+1}^n)\biggr).
\]
Then we can write
\begin{equation}
\E_t \ln r_{r+1} - \ln r_t = F_t + G_t,\label{loc-submart}
\end{equation}
where
\[
F_t = \E_t \ln \biggl(\frac{\sum_n (C_t \hat\lambda_t^n X_{t+1}^n +
\mu_t^nY_{t+1}^n)}
{\sum_i (C_t\hat\lambda_t^i X_{t+1}^i + Y_{t+1}^i)}\biggr), \qquad
G_t = \E_t \ln\biggl( \frac{\sum_i (C_t\hat\lambda_t^i X_{t+1}^i + Y_{t+1}^i)}{\sum_n (C_t\bar\lambda_t^n X_{t+1}^n + Y_{t+1}^n)}\biggr).
\]
Let us show that $F_t+ G_t \ge 0$. Let
\begin{equation}
d_t^n = 1- \E_t \frac{C_tX_{t+1}^n}{\sum_i (C_t \hat \lambda_t^i X_{t+1}^i +
Y_{t+1}^i)},\label{dt}
\end{equation}
so that from the definition of the strategy $\hat\Lambda$ we have
\[
d_t^n \hat\lambda_t^n =  \E_t \biggl(\frac{Y_{t+1}^n}{\sum_i(C_t\hat\lambda_t^iX_{t+1}^i + Y_{t+1}^i)} \biggr).
\]
Note that $d_t^n>0$ due to assumption (\ref{Y-positive}). Using the
concavity of the logarithm and applying Lemma~\ref{lemma-logsum}, we obtain
\begin{multline}
\label{F}
F_t \ge \E_t\biggl( \sum_{n=1}^N
\frac{Y_{t+1}^n}{\sum_i(C_t\hat\lambda_t^iX_{t+1}^i + Y_{t+1}^i)}\ln
\mu_t^n\biggr) = \sum_{n=1}^N d_t^n \hat\lambda_t^n \ln \mu_t^n \\
= \sum_{n=1}^n d_t^n \hat\lambda_t^n
\ln\frac{d_t^n\hat\lambda_t^n}{d_t^n\bar\lambda_t^n} \ge
\frac 14\sum_{n=1}^N (d_t^n(\hat\lambda_t^n - \bar \lambda_t^n))^2 + \sum_{n=1}^N d_t^n(\hat\lambda_t^n - \bar \lambda_t^n),
\end{multline}
where in the first inequality we considered the argument of the logarithm in
the definition of $F_t$ as the convex combination of the values
\[
1,\ \mu_t^1,\ \ldots ,\ \mu_t^N
\]
with the coefficients
\[
\frac{\sum_n C_t \hat\lambda_t^n X_{t+1}^n}{\sum_i(C_t\hat\lambda_t^iX_{t+1}^i +
Y_{t+1}^i)},\quad
\frac{Y_{t+1}^1}{\sum_i(C_t\hat\lambda_t^iX_{t+1}^i +
Y_{t+1}^i)},\quad \ldots\ ,\quad 
\frac{Y_{t+1}^N}{\sum_i(C_t\hat\lambda_t^iX_{t+1}^i +
Y_{t+1}^i)}.
\]
Using the inequality $\ln a \ge 1 - a^{-1}$, which is valid for
any $a>0$, we find
\begin{equation}
\label{G}
G_t \ge \E_t \frac{\sum_n C_t(\hat\lambda_t^n - \bar \lambda_t^n)
X_{t+1}^n}{\sum_i(C_t\hat\lambda_t^i X_{t+1}^i + Y_{t+1}^i)} = \sum_{n=1}^N
d_t^n (\bar\lambda_t^n - \hat \lambda_t^n).
\end{equation}
Thus, $F_t+G_t \ge 0$. Therefore, from (\ref{loc-submart}), one can see that
$\ln r_t$ is a generalized submartingale\footnote{Recall
that a sequence $S_t$ is called a generalized submartingale if $\E
|S_0|<\infty$ and $\E(S_t\mid\F_{t-1})\ge S_{t-1}$ for all $t\ge 1$ (but not
necessarily $\E|S_t|<\infty$). One can show that if $S_t\le C_t$ for
all $t$ with some integrable random variables $C_t$, then $S_t$ is
integrable, and hence a true submartingale.}, and hence a true submartingale
since it is bounded from above.
\end{proof}

To state the second theorem, let us introduce the notation
$\nu_t=(\nu_t^1,\ldots,\nu_t^N)$ for the representative strategy of agents
$m=2,\ldots,M$, i.e.
\[
\nu_t^n(\omega) = \frac{1}{1-r_t^1(\omega)}\sum_{m=2}^m r_t^m(\omega) \lambda_t^{m,n}(\omega).
\]
\begin{theorem}
\label{theorem2} Suppose that agent 1 uses the strategy $\hat\Lambda$. Then
she is a single survivor, i.e.\ $\lim_{t\to\infty} r_t^1 = 1$, a.s.\ on the
set
\begin{equation}
\biggl\{\omega : \sum_{t=0}^\infty\sum_{n=1}^N (d_t^n(\omega)(\lambda_t^{1,n}(\omega) -
\nu_t^n(\omega)))^2 = \infty\biggr\},\label{far}
\end{equation}
where $d_t^n$ are defined as in \eqref{dt}.
\end{theorem}
\begin{proof}
In the proof of Theorem~\ref{theorem1} we showed that $\ln r_t^1$ is a convergent
submartingale, and as follows from \eqref{loc-submart}, \eqref{F},
\eqref{G}, its compensator (i.e.\ a predictable non-decreasing sequence $A_t$
such that $\ln r_t - A_t$ is a martingale)
\[
A_t = \sum_{s=1}^t (\E_{s-1}\ln r_{s}^1 - \ln r_{s-1}^1)
\]
can be bounded
from below by
\[
A_{t+1} \ge \frac14\sum_{s=1}^t \sum_{n=1}^N (d_s^n(\lambda_s^{1,n} - \bar \lambda_s^n))^2 =
\frac14\sum_{s=1}^t (1-r_s)^2 \sum_{n=1}^N (d_s^n(\lambda_s^{1,n} - \nu_s^n))^2.
\]
Since the compensator of a convergent submartingale converges with
probability 1, on the set
\eqref{far} we necessarily have $\lim_{t\to\infty} r_t = 1$ a.s., which proves the theorem.
\end{proof}


\section{Relation to other results}
\paragraph{1. The modeld with exogenous returns.} Assume that in our model $Y_t^n \equiv 0$. Then equation \eqref{capital} becomes
\[
W_{t+1}^m = \sum_{n=1}^N \lambda_t^{m,n} W_t^m X_{t+1}^n.
\]
This is the familiar equation which defines the wealth of a self-financing strategy in a market with exogenous returns $X_t^n$. For example, if the asset prices are $S_t^n$, then it is natural to put $X_{t}^n = S_{t}^n/S_{t-1}^n$. Clearly, in this case the wealth of an agent depends only on his/her strategy and does not depend on the strategy of the other agents. Let us denote the corresponding wealth sequence by $W_t(\lambda)$.

A strategy $\hat\lambda$ is called \textit{growth optimal} (or a \textit{numeraire portfolio}), if for any other strategy $\lambda$
\[
\frac{W_t(\lambda)}{W_t(\hat\lambda)}\ \text{is a supermartingale},
\]
which is equivalent to our definition of a relative growth optimal strategy when $Y_t^n\equiv0$. 

The classic result (see, e.g, \cite{AlgoetCover88}) is that if the log-returns are integrable (i.e. $\E \ln X_{t+1}^n < \infty$), then the growth optimal strategy can be found by maximizing the expected log-return of the portfolio:
\[
\hat\lambda_t \in \argmax_{\lambda} \E\left(\ln \frac{W_{t+1}(\lambda)}{W_t(\lambda)} \mid \F_t\right)= 
\argmax_{\lambda} \E\left(\ln \sum_{n=1}^N \lambda_t^{n}  X_{t+1}^n\mid \F_t\right).
\]
If the log-returns are not integrable, this problem may be have no solution, however it is easy to see that if one introduces the relative returns $R_t^n = X_t^n /\sum_{i=1}^N X_t^i$, then the growth-optimal strategy can be found as 
\begin{equation}
\label{log-max}
\hat\lambda_t \in
\argmax_{\lambda} \E\left(\ln \sum_{n=1}^N \lambda_t^{n}  R_{t+1}^n\mid \F_t\right).
\end{equation}
This maximization problem always has a solution, provided that $\sum_{n=1}^N R_t^n > 0$ a.s. However, note that the solution may be not unique if, e.g., the $R_t^n$ are linearly dependent. 

Let us show that our strategy $\hat\lambda_t$ solves \eqref{log-max}. 
\red{Show that our strategy is a solution of \eqref{log-max} (or, it will be better to show that any fixed point defined in Lemma 1 is a solution of \eqref{log-max})}.


\paragraph{2. The model with short-lived assets and endogenous prices.} Now suppose that $X_t^n\equiv 0$. In this case we get the evolutionary finance model with short-lived assets of \cite{AmirEvstigneev+13}. They found the optimal strategy
\[
\hat\lambda_t^n = \E\left( \frac{Y_{t+1}^n}{\sum_{k=1}^N Y_t^{n}} \mid \F_t\right),
\]
which clearly agrees with formula \eqref{lambda-hat}.

\paragraph{3. The model with endogenous short-lived assets with exogenous assets.} 
Consider the model of a market with $N=N_1+N_2$ assets of two types: assets of the first type ($n=1,\dots,N_1$) have exogenous prices and returns like in the first example, while assets of the second type ($n=N_1+1,\ldots,N$) have endogenous prices and are short-lived like in the second example. For simplicity, assume that the log-returns of the assets of the first type are integrable, i.e.\ $\E |\ln X_t^n| < \infty$ for $n=1,\ldots,N_1$.

This model was considered by \cite{Zhitlukhin22} in a somewhat more general form
and it was found that the relative growth optimal strategy $\hat\lambda_t = (\hat\alpha_t^1,\dots,\hat\alpha_t^{N_1}, \hat\beta_t^1, \dots, \hat\beta_t^{N^2})$, can be found as follows. First, one finds $\hat\alpha_t^n$ by solving the optimization problem
\[
\hat\alpha_t = \argmax_{\alpha}\Biggl\{ \E\biggl( \ln\biggl( C_t \sum_{n=1}^{N_1} \alpha^n X_{t+1}^n + \sum_{n=N_1+1}^N Y_t^n\biggr) \mid \F_t \biggr) - \sum_{n=1}^{N_1} \alpha\Biggr\},
\]
where $C_t = \sum_{m=1}^M W_t^m$ denotes the total market wealth, and the maximum in the above formula is taken over the set $\{\alpha \in \R_+^{N_1} : a^1+\dots+a^{N_1} \le 1\}$. 
Then, $\hat\beta_t^n$, $n=1,\dots,N_2$, are defined by
\[
\hat \beta_t^n = \E \left( \frac{Y_{t+1}^{N_1+n}}{C_t\sum_{n=1}^{N_1} \alpha_t^n X_{t+1}^n  + \sum_{n=1}^{N_2} Y_{t+1}^{N_1+n}} \mid \F_t \right). 
\]
\red{Show how to obtain this formula from our formula.}


\small 
\setlength{\bibsep}{0.2em plus 0.3em}
\bibliographystyle{apalike}
\bibliography{affine}

\end{document}
