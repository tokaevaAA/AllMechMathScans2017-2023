
\documentclass{article}

\usepackage[T2A]{fontenc}  
\usepackage[utf8]{inputenc}
\usepackage[english,russian]{babel}
\usepackage{indentfirst}
\usepackage{misccorr}
\usepackage{graphicx}
\usepackage{amsmath}
\usepackage{amssymb}

\begin{document}



{\bf Развернутые ответы на вопросы 2(i), 7(ii), 7(iii), 8(i),8(iii), 8(iv)(a), 8(iv)(b)}\\

{\bf \large  2(i)} Если ценная бумага, по которой выплачиваются дивиденды,  торгуется $"$без дивиденда$"$, 
то это значит, что покупатель, приобретая ее, не получает права на получение ближайшего дивиденда (но получает право на получение всех последующих дивидендов).\\



{\bf \large  7(ii)} Почему YTM оказалось меньше, чем $f_{3,1}=7\%$?\\
Вспомним, что YTM является своеобразным средним значением спотовых ставок $y_1, y_2, y_3, y_4$ - см. формулу (6.7) на стр. 278:\\

$ \boxed {\sum\limits_{k=1}^{n} C_k(1+YTM)^{-t_k} + F(1+YTM)^{-t_n} =  \sum\limits_{k=1}^{n} C_k(1+y_{t_k})^{-t_k} + F(1+y_{t_n})^{-t_n} }$

А в нашем случае все эти эти спотовые ставки оказались меньше $f_{3,1}=7\%$ : \\

$ 1 + y_1=1.04 \Rightarrow  y_1=4\%$\\

$ (1+y_2)^2 = 1.04 \cdot 1.05  \Rightarrow  y_2=(1.092)^{1/2} - 1= 0.04498803820905062 = 4.4988\%$\\

$ (1+y_3)^3 = 1.04 \cdot 1.05 \cdot 1.06   \Rightarrow  y_3=(1.15752)^{1/3} - 1=  0.04996825300838603 = 4.4968\%$\\

$ (1+y_4)^4 = 1.04 \cdot 1.05 \cdot 1.06 \cdot 1.07   \Rightarrow  y_4=(1.2385464)^{1/4} - 1=  0.05494075450104474= 5.494\%$\\

А раз все эти спотовые ставки оказались меньше $f_{3,1}=7\%$, то и их среднее будет меньше $f_{3,1}=7\%$.\\

Вообще можно было и не считать спотовые ставки, а сказать, что YTM является своеобразным средним значением спотовых ставок, 
а спотовые ставки являются средним геометрическим всех предыдущих однолетних форвардных ставок (см. формулу (6.24) на стр. 290):

$ \boxed { 1+y_n = ((1+f_0 )(1+f_1) \dots (1+f_{n-1}))^{1/n} }$, где $f_u  \equiv f_{u,1}$\\

А в нашем случае по условию все эти однолетние форвардные ставки меньше $f_{3,1}=7\%$, 
поэтому и спотовые ставки меньше $f_{3,1}=7\%$, поэтому YTM меньше $f_{3,1}=7\%$.\\


{\bf \large  7(iii)}  Почему  может возрастать последовательность однолетних форвардных ставок?\\
Мы уже выяснили, что спотовые ставки являются средним геометрическим всех предыдущих однолетних форвардных ставок (см. формулу (6.24) на стр. 290):

$ \boxed { 1+y_n = ((1+f_0 )(1+f_1) \dots (1+f_{n-1}))^{1/n} }$, где $f_u  \equiv f_{u,1}$\\

Поэтому раз возрастает последовательность однолетних форвардных ставок, то возрастает и последовательность спотовых ставок, а это можно объяснить тремя способами:\\

1) (чистая теория ожиданий) инвесторы ожидают роста спотовых краткосрочных процентных ставок\\
2)(теория предпочтения ликвидности) инвесторы учитывают риски владения долгосрочными облигациями и требуют за них увеличенную доходность\\
3)(теория сегментации рынка) краткосрочные и долгосрочные облигации интересуют разные группы людей, и в данный момент на рынке сложился повышенный спрос( или недостаточное предложение)  на краткосрочные бумаги и имеется избыточное предложение ( или недостаточный спрос)  долгосрочных.\\


{\bf \large  8(i)} Величина TWRR, применяемая для оценки эффективности работы например фонда,  находится из условия\\

$  \boxed { (1+TWRR)^{T} = (1+r_1^{eff}) \dots (1+r_n^{eff}) } $, \\

 где $r_n^{eff} = \frac{ F(t_n)-0) - F(t_{n-1} +0) }{ F(t_{n-1} +0)}$, \\

где $F(t_n+0)$- стоимость фонда в начале промежутка $\left[ t_n+0; t_{n+1}-0\right]$, \\

а $F(t_{n+1}-0)$- стоимость фонда в конце промежутка $\left[ t_n+0; t_{n+1}-0\right]$\\

TWRR разумно использовать для оценки эффективности деятельности например управляющего фонда, 
потому что он никогда не знает, когда деньги поступят в фонд  или уйдут из него, а только может выбирать виды активов, в которые нужно инвестировать. 
И 1+ TWRR как раз  является взвешенным средним геометрическим эквивалентных коэффициентов годового роста для промежутков, где средства фонда росли/ убывали только за счет деятельности управляющих фонда.\\



{\bf \large  8(iii)} Величина MWRR  -  это внутренняя ставка доходности IRR для фондов, 
то есть корень уравнения доходности, причем только в том случае, если это уравнение имеет единственный корень на промежутке  $ -1 < i < +\infty$\\

Уравнение доходности:\\
$F(0) + \sum\limits_{n=1}^{N} M(t_n) (1+i)^{-t_n} = F(T)(1+i)^{-T}$, где F(t) -стоимость фонда в момент t, а M(t) - сумма новых денег.\\

MWRR разумно использовать для оценки эффективности  инвестиционной деятельности индивидуальных инвесторов, поскольку они сами принимают решение о том, когда и сколько денег вкладывать.\\

{\bf \large  8(iv) (a) } Запишем уравнение доходности для фонда (в момент $t=2$):\\

$100z^2 + 16 = 270z^{-1.5}$\\

$f(z)= 100z^{3.5} + 16z^{1.5} - 270 = 0$\\

Поскольку $100>0$ и $270>0$, то функция f(z) на промежутке $z>0$ монотонно возрастает от $-\infty$ до $+\infty$.\\

Поэтому уравнение $f(z)=0$ имеет единственный положительный корень, который и будет $k_{TW}$\\

{\bf   8(iv)(b)} Почему 29\% < MWRR < 30\%?\\

$f(1.29) = -2.740464789783573 < 0$\\
$f(1.30) = 4.21218973634273650 > 0$\\

То есть f имеет разные знаки на концах отрезка $\left[ 1.29, 1.3\right]$. \\
Значит,  f  имеет корень внутри этого отрезка.



{\large \bf Решения:}\\




{\bf \large1.} Вычислите в виде процентов с четырьмя знаками после запятой номинальную
 годовую процентную ставку, начисляемую раз в полугодие, которая эквивалентна:\\
 (i) эффективной месячной учетной ставке $0.5\%.$  [2 балла]\\
 (ii) номинальной годовой учетной ставке $6\%,$ применяемой каждые два года.  [2 балла]\\
 (ii) номинальной годовой процентной ставке $6\%,$ применяемой ежеквартально.  [2 балла]\\
$\left[\right. $Всего 6 баллов $\left.\right]$\\

{\bf \largeРешение:}\\
Нас просят найти $i^{(2)}.$ \\

{\bf \large  (i)} Дано $d_{*}^{(12)} = 0.5\% = 0.005$\\
$1-d_{*}^{(2)} =(1-d_{*}^{(12)} )^{6}$\\
$\Rightarrow d_{*}^{(2)} =1- (1-d_{*}^{(12)} )^{6} = 1- 0.995^{6} = 1- 0.9703725093562656 =0.029627490643734378 $\\
$\Rightarrow i_{*}^{(2)} =\frac {d_{*}^{(12)}} {1-d_{*}^{(12)} } = \frac {0.029627490643734378} {0.9703725093562656} = 0.030532079544781134 $\\
$\Rightarrow i^{(2)} = 2i_{*}^{(2)}  = 0.06106415908956227$\\
Округляем до процентов с четырьмя знаками после запятой: $ i^{(2)}  = \bf  0.061064 $\\

{\bf \large  (ii)} Дано $d^{(0.5)} = 6\% = 0.06$\\
$\Rightarrow d_{*}^{(0.5)} = \frac{d^{(0.5)} }{0.5} = 0.12$\\
$(1-d_{*}^{(2)})^4=1-d_{*}^{(0.5)} $\\
$\Rightarrow d_{*}^{(2)} =1-(1-d_{*}^{(12)} )^{\frac {1} { 4}} = 1-0.88^{\frac {1} {4}} = 1- 0.9685469281169012 = 0.03145307188309876 $\\
$\Rightarrow i_{*}^{(2)} =\frac {d_{*}^{(12)}} {1-d_{*}^{(12)} } = \frac {0.03145307188309876} {0.9685469281169012} = 0.032474494492746406$\\
$\Rightarrow i^{(2)} = 2i_{*}^{(2)}  = 0.06494898898549281$\\
Округляем до процентов с четырьмя знаками после запятой: $ i^{(2)}  = \bf   0.064949$\\

{\bf \large  (iii)} Дано $i^{(4)} = 6\% = 0.06$\\
$\Rightarrow i_{*}^{(4)} = \frac{ i^{(4)} }{4} = 0.015$\\
$1+i_{*}^{(2)}=(1+i_{*}^{(4)})^2 $\\
$\Rightarrow i_{*}^{(2)} = (1+i_{*}^{(4)})^2 -1 = 1.030225 - 1 = 0.030225$\\
$\Rightarrow i^{(2)} = 2i_{*}^{(2)}  =  0.06045$\\
Округляем до процентов с четырьмя знаками после запятой: $ i^{(2)}  = \bf   0.060450$\\

{\bf \large 2.} (i) Объясните смысл термина «без дивиденда» («ex-dividend») применительно к продаже ценной бумаги, по которой выплачиваются дивиденды. [1 балл]\\
Человек купил 10,000 акций 1 декабря 2017. Дивиденды по ним выплачиваются 1 января и 1 июля каждого года; предполагается, что они будут выплачиваться до бесконечности. Очередные дивиденды, которые будут выплачены 1 января 2018 года, составляют \$0.07 на акцию. Ожидается, что для любого календарного года обе дивидендные выплаты будут одинаковы, но от года к году будут расти на $2\%$ в год.\\
Допустим, что 1 декабря 2017 года эти акции продаются ex-dividend, а для оценки текущей стоимости денежных потоков используется эффективная годовая процентная ставка $7\%.$\\
(ii) Вычислите стоимость этого пакета в день покупки, предполагая, что покупатель будет держать его вечно. [5 баллов]\\
$\left[\right. $Всего 6 баллов $\left.\right]$\\

{\bf \largeРешение:}\\
Нас просят найти PV этого денежного потока на 1 декабря 2017 года. \\

{\bf \large  (i)} Если ценная бумага, по которой выплачиваются дивиденды,  торгуется $"$без дивиденда$"$, то это значит, что покупатель, приобретая ее, не получает права на получение ближайшего дивиденда (но получает право на получение всех последующих дивидендов).\\

{\bf \large  (ii)} Сначала обозначим 1 января  2017 года за $t=0,$ будем измерять время годами и найдем  стоимость $PV_1$ в момент 1 января  2017 года, то есть $t=0$   денежного потока, генерируемого одной акцией, продаваемой с дивидендом (а потом этот лишний первый дивиденд обратно вычтем).  Чтобы найти стоимость $PV_2$ нужного денежного потока в момент  1 декабря 2017 года, то есть $t= - \frac {1}{12},$ нужно из  $PV_1$ вычесть обратно прибавленный нами первый дивиденд, а потом дисконтировать полученную сумму на месяц назад, то есть умножить на $v^{\frac {1}{12}},$ где $v=\frac {1}{1+i},$ $i=0.07.$\\

Итак, рассмотрим одну акцию, продаваемую с дивидендом.\\
Она генерирует поток\\
\{ ( $\$0.07  $, t=0), ( $\$0.07  $, t=0.5) \},  \\
\{ ( $\$0.07 \cdot 1.02 $, t=1), ( $\$0.07 \cdot 1.02 $, t=1.5) \},  \\
\{ ( $\$0.07 \cdot (1.02)^2 $, t=2), ( $\$0.07 \cdot (1.02)^2$, t=2.5) \},  \\
\{ ( $\$0.07 \cdot (1.02)^3 $, t=3), ( $\$0.07 \cdot (1.02)^3$, t=3.5) \} \\
$\dots$\\

По условию каждая из  выплат за k-й год, где  $ k= 1,2, \dots, $ равна $0.07 (1.02)^{k-1}$\\
Для нашего удобства заменим две выплаты, происходящие во время  k-го года , $ k= 1,2, \dots$ на одну выплату в начале  k-го года, дисконтировав  июльскую выплату назад на полгода к январю, то есть умножив на $v^{\frac {1}{2}},$ где $v=\frac {1}{1+i},$ $i=0.07.$\\
Величина этой выплаты равна\\
 $ P_k = 0.07 (1.02)^{k-1} + v^{\frac 1 2} 0.07 (1.02)^{k-1} =(1+v^{\frac 1 2} ) 0.07 (1.02)^{k-1}$\\

Теперь мы получаем бесконечный поток с величиной выплаты $P_k, $  где  $ k= 1,2, \dots, $  в моменты $t=0,1, \dots$\\
Его текущая стоимость \\
$PV_1 = \sum\limits_{k=1}^{\infty} v^{k-1}P_k = \sum\limits_{k=1}^{\infty} v^{k-1} (1+v^{\frac 1 2} ) 0.07 (1.02)^{k-1}
= 0.07 (1+v^{\frac 1 2} )  \sum\limits_{k=1}^{\infty} (1.02v)^{k-1}   =  0.07\frac{ 1+v^{\frac 1 2} }{1-1.02v}  
 =   0.07 \frac{ (1+i) +\sqrt {(1+i) } }  {(1+i )-1.02} =    0.07 \frac{ (1+0.07) +\sqrt {(1+0.07) }  }  {(1+0.07 )-1.02}  =
 = 0.07 \frac{ 1.07 +\sqrt {1.07 }) }  {0.05}  =  0.07 \frac{ 1.07 + 1.03440804327886  }  {0.05} = 0.07 \frac{ 2.10440804327886 }  {0.05}= 2.946171260590404$\\
  
Теперь вычтем обратно самый первый дивиденд; он выплачивается в момент $t=0,$ и $PV_1$  рассчитана тоже  в момент $t=0,$ поэтому этот первый дивиденд никуда дисконтировать не надо, а нужно просто вычесть 0.07.\\

$ \Rightarrow \tilde {PV_1} = PV_1 -0.07 = 2.876171260590404$\\

Теперь  для получения ответа дисконтируем  $ \tilde {PV_1} $ на месяц назад:

$ \Rightarrow  PV_2 = v^{\frac {1} {12}} \tilde {PV_1} = \frac {1} {(1+i)^{\frac {1} {12}}} \tilde {PV_1} 
= \frac { 2.876171260590404} {(1+0.07)^{\frac {1} {12}}} = \frac { 2.876171260590404} {1.005654145387405} = 2,860000402506595 $\\

Это текущая стоимость на 1 декабря 2017 года денежного потока одной акции. Поэтому текущая стоимость на 1 декабря 2017 года денежного потока портфеля из 10000 таких акций равна $10000*PV_2 =  \bf \$28600$\\


{\bf \large 3.} Человек покупает в страховой компании ренту за разовую премию. Рента будет выплачивать \pounds10,000 в конце каждого года на протяжении 15 лет. Страховая компания инвестирует премию в облигацию, которая платит купон раз в год по ставке $6\%$ годовых и будет погашена по номиналу ровно через девять лет.\\
(i) (a) Вычислите средний дисконтированный срок выплат для ренты при годовой эффективной процентной ставке $5\%.$  [2 балла]\\
(b) Вычислите средний дисконтированный срок выплат для облигации при годовой эффективной процентной ставке $5\%.$ [3 балла]\\
(ii) Объясните, получит страховая компания прибыль или понесёт убытки, если процентные ставки слегка уменьшатся для всех сроков. [3 балла]\\
$\left[\right. $Всего 8 баллов $\left.\right]$\\

{\bf \largeРешение:}\\
Нас просят найти $MDT_{renta}$ и $MDT_{obligation}$\\

Согласно общим формулам(см. стр 262):\\

Для стандартной запаздывающей ренты на n лет: $ \boxed  {  MDT_{renta} = \frac {(Ia)_{\overset{-}n | }}   { a_{\overset{-}n | } }   }  $\\
Для  n-летней облигации с выплатой купона в конце каждого года по ставке $i_C$ средняя дисконтированная продолжительность промежутка выплат задается формулой $   \boxed  {   MDT_{obligation} = \frac {i_C \cdot  (Ia)_{\overset{-}n | }    + n v^n} { i_C  \cdot a_{\overset{-}n | }  + v^n }  }$\\

Где $  \boxed { a_{\overset{-}n | }  = \frac {1 - v^n} {i} } $\\

$  \boxed { {(Ia)_{\overset{-}n | }} =  \frac {1 + a_{\overset{-}n | }  -(n+1)v^n } {i}  = \frac {(1-v^n) + a_{\overset{-}n | }  -n v^n } {i}   =   \frac {(1+i) a_{\overset{-}n | }  -n v^n } {i} } $ \\


{\bf \large  (i)} 
Для ренты по условию $n=15$ лет\\

$ \bullet  a_{\overset{-}15 | }  = \frac {1 - v^{15}} {i}= \frac {1 - (1+0.05)^{-15}} {0.05} \\
=  \frac {1 -  0.48101709809097} {0.05} =  \frac {0.51898290190903} {0.05} = 10.379658038180601$\\

$ \bullet  (Ia)_{\overset{-}15 | } = \frac {(1+i) a_{\overset{-}15 | }  -15v^{15} } {i}  =  \frac {(1+0.05) a_{\overset{-}15 | }  -15 (1+0.05)^{-15} } {0.05} 
=   \frac {1.05  \cdot 10.379658038180601  -15  \cdot 0.48101709809097 } {0.05}  =
 \frac {10.898640940089631 - 7.21525647136455 } {0.05}   =   \frac {3.683384468725081 } {0.05}  = 73.66768937450162$

$\Rightarrow  MDT_{renta} = \frac {(Ia)_{\overset{-}15 | }}   { a_{\overset{-}15 | } }   
=\frac { 73.66768937450162}   { 10.379658038180601  } =  7.097313717226706  = \bf 7.0973 (years)$ \\


{\bf \large  (ii)} 
Для облигации по условию $n=9$ лет, $i_C = 6\% = 0.06$\\

$ \bullet  a_{\overset{-}9 | }  = \frac {1 - v^{9}} {i}= \frac {1 - (1+0.05)^{-9}} {0.05} \\
=  \frac {1 -  0.644608916217797} {0.05} =  \frac {0.355391083782203} {0.05} = 7.107821675644053$\\


$ \bullet  (Ia)_{\overset{-}9 | } = \frac {(1+i) a_{\overset{-}9 | }  -9v^{9} } {i}  =  \frac {(1+0.05) a_{\overset{-}9 | }  -9 (1+0.05)^{9} } {0.05} 
=   \frac {1.05  \cdot 7.107821675644053  -9  \cdot 0.644608916217797} {0.05}  =
 \frac {7.4632127594262565 - 5.801480245960173 } {0.05}   =   \frac {1.6617325134660836 } {0.05}  = 33.23465026932167$\\
 
 $\Rightarrow  MDT_{obligation} = \frac {i_C \cdot  (Ia)_{\overset{-}9 | }    + 9 v^9} { i_C  \cdot a_{\overset{-}9 | }  + v^9 }
 =  \frac {0.06 \cdot  33.23465026932167   + 9  \cdot 0.644608916217797} { 0.06  \cdot 7.107821675644053 + 0.644608916217797 }
 =  \frac {1.9940790161593  + 5.801480245960173} { 0.4264693005386432+ 0.644608916217797 } 
 =   \frac {7.795559262119473} { 1.0710782167564403 } = 7.278235277463548 = \bf   7.2782 (years) $ \\

 
 {\bf \large  (iii)} 
 Есть формула\\
 $\frac {PV(i + \Delta) - PV(i) } { PV(i) } \approx  \frac {PV^{'}(i) }  { PV(i) } \cdot \Delta = -v \cdot MDT \cdot \Delta$\\
 
 То есть при уменьшении процентной ставки i на $\Delta,$ относительное увеличение текущей стоимости денежного потока  будет равно vMDT. А поскольку текущие стоимости у облигации и у ренты равны, но  у облигации( то есть активов страховой) MDT больше, чем у ренты (то есть обязательств), то стоимость облигации (активов) увеличится больше, чем стоимость ренты (обязательcтв),   то есть   уменьшение процентной ставки приведет к {\bf прибыли} для страховой компании.\\
 
 

{\bf \large 4.} Интенсивность процентов, $\delta(t),$  является функцией времени и в любой момент времени $t$  (лет) даётся формулой:
\[
\begin{cases}
0.03 + 0.005t  &\text {$ 0 \le t < 2 $} \\
 0.045 - 0.0025t &\text {$ 2 \le t < 10 $}\\
0.02                 &\text {$  t \ge 10 $} 
\end{cases}
\]

(i) Для инвестиции в размере $\pounds 15,000,$  сделанной в момент $t=1,$ вычислите накопление в момент  $t=9.$  [4 балла]\\
(ii) Вычислите текущую стоимость (в момент  $t=0$  ) денежного потока, который выплачивается непрерывно с интенсивностью $\rho(t) = 60e^{-0.02t}$ от момента $t=10$  до момента $t=12.$ [6 баллов] \\
$\left[\right. $Всего 10 баллов $\left.\right]$\\



\textbf {Решение:}
{\bf \large  (i)}  Согласно общей формуле: $ \boxed {A=Pe^ { \int\limits_{t_1}^{t_2} \delta(t)dt }} $, где A - это накопление (в момент  $t_2$), а P - это инвестиция  (в  момент  $t_1$);\\
В нашем случае  $t_1=1$ и  $t_2=9$  $\Longrightarrow$   $ A=Pe^ { \int\limits_{1}^{9} \delta(t)dt }$;\\
Но  $\int\limits_{1}^{9} \delta(t)dt $ = $\int\limits_{1}^{2} \delta(t)dt $ +$\int\limits_{2}^{9} \delta(t)dt $;\\

$\bullet$ $\int\limits_{1}^{2} \delta(t)dt  = 
\int\limits_{1}^{2}   (0.03 +0.005t) dt = 
0.03\int\limits_{1}^{2} dt  +0.005 \int\limits_{1}^{2} tdt=
0.03t \bigg |_1^2 + 0.005  \frac{1}{2} t^2 \bigg |_1^2 =0.03 + 0.005 \cdot \frac{1}{2}(4-1) = 0.03 + 0.0075=\underline {0.0375}$\\

$\bullet$ $\int\limits_{2}^{9} \delta(t)dt  = 
\int\limits_{2}^{9}   (0.045 -0.0025t) dt = 
0.045\int\limits_{2}^{9} dt  -0.0025 \int\limits_{2}^{9} tdt=
0.045t \bigg |_2^9 -0.0025 \cdot \frac{1}{2} t^2 \bigg |_2^9 =
0.045(9-2) - 0.0025 \cdot \frac{1}{2}(81-4) =
0.045 \cdot 7 - 0.0025 \cdot \frac{77}{2}= 0.315 - 0.09625= \underline {0.21875}$\\

 $\Longrightarrow$ $ \int\limits_{1}^{9} \delta(t)dt = 0.0375 +0.21875= \underline {0.25625}$\\
 
$\Longrightarrow$ $ e^ { \int\limits_{1}^{9} \delta(t)dt } = e^{0.25625} \approx 1.292$\\

$\Longrightarrow$ $ A=Pe^ { \int\limits_{1}^{9} \delta(t)dt } \approx 15000 \cdot 1.292  = 19381.1356 \approx\bf \pounds19381.14$\\


{\bf \large  (ii)}  Согласно общей формуле для текущей стоимости денежного потока с заданной интенсивностью:\\
$ \boxed {PV = \int\limits_{0}^{12} \rho(t)  v_t dt $, where $v_t =  e^ { -\int\limits_{0}^{t} \delta(u)du } }$\\
Но  $\rho(t) \equiv 0$ for $t \notin [10;12]$\\
$\Longrightarrow$ $ PV = \int\limits_{10}^{12} \rho(t)  e^ { -\int\limits_{0}^{t} \delta(u)du }dt  $\\
Нам нужно вычислить $  \int\limits_{0}^{t} \delta(u)du$ for $ t \in [10;12] $\\

$\int\limits_{0}^{t} \delta(u)du  
= \int\limits_{0}^{2} \delta(u)du  + \int\limits_{2}^{10} \delta(u)du + \int\limits_{10}^{t} \delta(u)du=\\
= \int\limits_{0}^{2}   (0.03 +0.005u) du +  \int\limits_{2}^{10}   (0.045 -0.0025u) du + \int\limits_{10}^{t}  0.02 du=\\
= 0.03u \bigg |_0^2 +0.005 \cdot \frac{1}{2} u^2 \bigg |_0^2 
 + 0.045u \bigg |_2^{10} -0.0025 \cdot \frac{1}{2} u^2 \bigg |_2^{10}+
 +0.02u \bigg |_{10}^t =$
 
 $=0.03 \cdot 2  +0.005 \cdot 2 + 0.045 \cdot 8 - 0.0025 \cdot \frac{1}{2} \cdot 96 +0.02(t-10)=\\$
 
 $=0.06 + 0.01 +0.36 - 0.12 -0.2 +0.02t = \underline {0.11 +0.02t} $\\
 
 
 $\Longrightarrow$ $ e^ { -\int\limits_{0}^{t} \delta(u)du}  = e^{-0.11 -0.02t}$\\
 
$\Longrightarrow$ $PV=  \int\limits_{10}^{12} \rho(t)  e^ { -\int\limits_{0}^{t} \delta(u)du }dt  =
\int\limits_{10}^{12} 60 \cdot e^{0.02t } \cdot   e^ { -0.11 -0.02t }dt  = 
60 \cdot e^{-0.11}  \int\limits_{10}^{12} dt =
 60 \cdot e^{-0.11} \cdot 2 = 120 \cdot e^{-0.11}  = 107.50009623558338 \approx \bf \pounds107.50$\\
 

 

{\bf \large 5.} 1 февраля 2017 года инвестор обдумывал покупку обычных акций компании Online Education PLC. Дивиденды выплачиваются раз в год, 1 февраля, и только что были выплачены дивиденды в размере £0.40 на акцию. В момент покупки акции ожидалось, что дивиденды будут ежегодно возрастать: на 5\% за первый год, на 4\% за второй, на 3\% за третий и последующие годы. Инвестор не имел права на получение только что выплаченных дивидендов.\\
(i) Вычислите максимальную цену, которую мог бы заплатить за акцию инвестор, если он предполагает держать акцию бессрочно и рассчитывает на эффективную годовую доходность от этой операции в размере 9\%. [6 баллов]\\
Инвестор купил пакет акций 1 февраля 2017 года по цене £7.00 за акцию и продал его 1 февраля 2019 года, немедленно после получения причитающихся ему дивидендов, по цене \pounds7.50 за акцию.\\
(ii) Вычислите эффективную годовую доходность этой операции для инвестора, используя следующую информацию: [5 баллов]\\

 \begin{tabular}{ccc}
Дата     & \text{ Индекс инфляции} & Дивиденды на одну акцию \\
1 февраля 2017 & \text{211.0   } & {\pounds0.400} \\
1 февраля 2018 & \text{215.7   } & {\pounds0.428} \\
1 февраля 2019 & \text{221.2  } & {\pounds0.449} \\

\end{tabular}

$\left[\right. $Всего 11 баллов $\left.\right]$\\

{ \bf \large Решение:}
{\bf \large  (i)}  Примем момент 1 февраля 2017 в качестве начального и будем измерять время годами. \\
При $i=9\%=0.09, v=\frac{1}{1+i}=0.917431193$ текущая стоимость одной акции равна:\\

$PV = 0.4 \cdot 1.05 \cdot v +  0.4 \cdot 1.05 \cdot 1.04 \cdot  v^{2} + \sum\limits_{k=1}^{\infty} 0.4 \cdot 1.05 \cdot 1.04 \cdot 1.03^{k} \cdot v^{k+2}=\\
=0.4 \cdot 1.05 \cdot v +  0.4 \cdot 1.05 \cdot 1.04 \cdot  v^{2} + 0.4 \cdot 1.05 \cdot 1.04   \sum\limits_{k=1}^{\infty} 1.03^{k} \cdot v^{k+2}=\\
= 0.4 \cdot 1.05 \cdot v +  0.4 \cdot 1.05 \cdot 1.04 \cdot  v^{2} + 0.4 \cdot 1.05 \cdot 1.04  \cdot 1.03 \cdot v^{3}   \sum\limits_{k=1}^{\infty} 1.03^{k-1} \cdot v^{k-1} = \\
= 0.4 \cdot 1.05 \cdot v +  0.4 \cdot 1.05 \cdot 1.04 \cdot  v^{2} + 0.4 \cdot 1.05 \cdot 1.04  \cdot 1.03 \cdot v^{3}  \cdot \frac{1}{1-1.03v} = 7.064220183 \approx  \bf  \pounds 7.06$\\

{\bf \large  (ii)} Покупка акции  в момент t=0 (1 февраля 2017)  за $\pounds 7 $ дает инвестору право получить такой денежный поток (учитывая факт продажи этой акции за   $ \pounds 7.5  $  в момент t=2 (1 февраля 2019)):\\


$\pounds 0.428$ в момент $t=1$  (1 февраля 2018) - дивиденды\\
$\pounds 0.449$ в момент $t=2$  (1 февраля 2019) - дивиденды\\
$\pounds 7.5   $ в момент $t=2$  (1 февраля 2019) - от продажи акции \\

То есть без учета инфляции  текущая стоимость этого денежного потока равна $0.428v + 0.449v^{2} + 7.5v^{2}$\\

Но нам надо учесть инфляцию, то есть сумма A в момент k эквивалентна сумме  
$A \cdot  \frac {S_0}{S_k}$  в момент  $t=0,$ где $S_k$ - это индекс инфляции в момент k.\\

То есть с  учетом инфляции  текущая стоимость этого денежного потока равна
 $0.428v \cdot \frac {211.0}{215.7} + 0.449v^{2} \cdot \frac {211.0}{221.2}  + 7.5v^{2}  \cdot \frac {211.0}{221.2}$\\
 
Приравниваем текущую стоимость к цене $ \pounds 7$, по которой акция была куплена, и получаем уравнение на v:\\

$7=0.428v \cdot \frac {211.0}{215.7} + 0.449v^{2} \cdot \frac {211.0}{221.2}  + 7.5v^{2}  \cdot \frac {211.0}{221.2}$\\

$ \Rightarrow 7=0,418674084v + 7,582454792v^2$\\

Это парабола, она имеет два корня:$ v= -0.988829$ и $ v= 0.933613$\\

Отрицательный нам не подходит по смыслу \\

$\Rightarrow v=0.933613 $\\
$ \Rightarrow  i  = \frac {1}{v} - 1 = 0,071107622 \approx  \bf 7,11\%$\\

Отметим, что с помощью функции  "ВСД({-7;0,418674084;7,582454792})"   из Microsoft Excel можно было сразу получить $i=0.071107555   \approx  \bf 7,11\%$\\



{\bf \large 6.} 1 января 2016 года был выдан заём на сумму \pounds 80,000. Он должен быть погашен за 10 лет постоянными ежемесячными платежами 1 числа каждого последующего месяца (вплоть до 1 января 2026 года включительно).\\
(i) Вычислите размер этого постоянного ежемесячного платежа используя годовую эффективную процентную ставку 8\%. [2 балла]\\
(ii) Вычислите размер непогашенной задолженности 1 ноября 2018 года (немедленно после того, как произведён очередной платёж в соответствии с установленным расписанием). [3 балла]\\
1 ноября 2018, немедленно после платежа очередной суммы в счёт погашения долга, заёмщик попросил уменьшить размер ежемесячной выплаты до £900 и продлить промежуток времени, оставшийся до погашения долга (чтобы непогашенную задолженность можно было полностью оплатить уменьшенными ежемесячными платежами). Последний платёж должен быть равен размеру оставшейся задолженности, если она меньше, чем \pounds 900.\\
Кредитор согласился с этими изменениями при следующих дополнительных условиях:\\
$\bullet$ в будущем будет применяться годовая процентная ставка 9\%, начисляемая ежемесячно;\\
$\bullet $ к непогашенному долгу по состоянию на 1 ноября 2018 года добавляется сбор за оформление документов в размере \pounds 250.\\
(iii) (a) Определите новую дату погашения долга. [2 балла]\\
(b) Вычислите размер последней выплаты по долгу. [4 балла]\\
$\left[\right. $Всего 11 баллов $\left.\right]$\\

{ \bf \large Решение:}
{\bf \large  (i)}  Примем момент 1 января 2016 в качестве начального и будем измерять время месяцами. \\
Тогда месячная эффективная процентная ставка есть $i_{*}^{(12)} = (1+i)^{\frac {1}{12}} - 1 =  (1+0.08)^{\frac {1}{12}} - 1 = 0,00643403$\\
Обозначим за P ежемесячный платеж заемщика.\\
Тогда обязательства заемщика - это стандартная запаздывающая  рента длины 120 с ежемесячным платежом величины P\\
Ее текущая стоимость равна $P  \cdot a_{\overset{-}120 | @  i_{*}^{(12)}}  = P \cdot \frac { 1 - (1+i_{*}^{(12)})^{-120} }{i_{*}^{(12)}}  = 83,43239039P $\\

А обязательства заимодавца- выплатить сейчас \pounds 80000\\
Поэтому из принципа эквивалентности обязательств $ 80000= 83,43239039P$
$\Rightarrow P = \frac {80000}{83,43239039} = 958,8602175 = \bf \pounds 958,86$\\




{\bf \large  (ii)}   Найдем размер непогашенной задолженности на момент 1 ноября  2018 сразу после  очередной выплаты ретроспективным способом;\\
до этого  момента  в прошлом   было произведено 34 выплаты,  каждая по  $\pounds 958,86,$ в моменты $t=1,2, \dots, 34$\\

Стоимость этих выплат в момент $t=34$ равна\\
 $PV_1= 958.86 ( (1+i_{*}^{(12)} )^{33} + \cdots + 1) = \frac {1 - (1+i_{*}^{(12)} )^{34} }{ 1 - ((1+i_{*}^{(12)} ))} = -958.86 \frac{1-1,00643403^{34}}{0,00643403} = 36312,079$\\

А стоимость долга к моменту $t=34$ возросла до \\
$PV_2= 80000 \cdot (1+i_{*}^{(12)} )^{34} = 80000 \cdot 1,00643403^{34} = 99492,564$\\

Поэтому стоимость оставшегося долга в момент $t=34$ равна\\

$PV=PV_2-PV_1 = 99492,564-36312,079= \bf \pounds63180,485$\\

{\bf Замечание}

Отметим, что проспективным способом получим
 $PV=958.86 a_{  \overset{-}86 | @  i_{*}^{(12)}  } = 958.86 \frac { 1 - (1+i_{*}^{(12)})^{-86} }{i_{*}^{(12)}} = 
   958.86 \frac { 1 - (1.00643403)^{-86} }{ 1.00643403} = \pounds 63180.46$\\
 
 Если же у $ i_{*}^{(12)}$ оставить только 4 знака после запятой:  0,006434, то:\\
 
 $PV=958.86 a_{\overset{-}86 | @  i_{*}^{(12)}} = 958.86 \frac { 1 - (1+i_{*}^{(12)})^{-86} }{i_{*}^{(12)}} = 
   958.86 \frac { 1 - (1.006434)^{-86} }{ 1.006434} = \pounds 63180.54$\\
 
 
 А измеряя время годами (и используя формулу  (2.28) со стр. 80 : \\
 $  \boxed {a^{(p)}_{\overset{-}n | } = \frac {i}{i^{(p)} }   a_{\overset{-}n | }}$,
 получим ответ 63180.76:\\
 
 $PV= 12\cdot 958.86 a^{(12)}_{\overset {-} {\frac {86 }{12} } | @  0.08}  =
 11506.32 \frac{ 1- (1+0.08)^{-\frac{86}{12} } }{12(1+0.08)^{- \frac {86}{12}} }=  
   11506.32 \frac{  0.42394614715974477 }{0.077208} = \bf 63180.76$
 
 
 

{\bf \large  (iii) (a)}  После изменений условий кредита используется номинальная ставка $i^{(12)} = 9\% = 0.09$\\

Следовательно месячная эффективная ставка равна $ i_{*}^{(12)} = \frac{ i^{(12)} }{12} = 0.0075$\\

Пусть после реструктуризации до полной выплаты долга нужно сделать n выплат; тогда стоимость  в момент $t=34$  этих выплат равна 
$ 900a_{\overset{-}n |@0.0075 } ,$

в то время как стоимость оставшегося долга в момент $t=34$  (после добавления \pounds 250 за оформление документов)  равна   $63180,485+250=63430,485$\\

Ищем n из условия:\\

 $ 900a_{\overset{-}n |@0.0075 } \geq  63430,485$\\
 
$900 \frac{  1 - (1+i_{*}^{(12)} )^{-n}  }{ i_{*}^{(12)} } \geq  63430,485$\\

$  1 - (1+i_{*}^{(12)} )^{-n}   \geq   i_{*}^{(12)}  \frac {63430,485}{900} =   0.0075 \frac {63430,485}{900}  = 0.5286$\\


$   (1+i_{*}^{(12)} )^{-n}   \leq  1- 0.5286 =  0.4714$\\

$  -n\cdot ln (1+i_{*}^{(12)} )   \leq  ln(  0.4714)$\\

$ n \geq \frac{ ln(  0.4714)}{  ln (1+i_{*}^{(12)} ) } = \frac{- ln(  0.4714)}{  ln (1.0075) } = \frac{ 0.752}{0.007472} = 100.64$\\

То есть нам потребуется 100 выплат по \pounds900 и 101-я выплата размером $X\leq900.$\\

То есть после реструктуризации пройдет 101 выплата, значит, с момента $t=0$ пройдет $ 34+101 = 135$
 месяцев = 11 лет и 3 месяца с момента 1 января 2016,  то есть долг будет погашен {\bf 1 апреля 2027.}\\
 
 {\bf (b)} Найдем величину последней выплаты X (произведенной в момент $t=34+101$ ), приравняв стоимость остатка долга и оставшихся выплат в момент $t=34+100:$\\
 
 $63430,485(1+i_{*}^{(12)})^{100} = 900(  (1+i_{*}^{(12)})^{99} + \dots + 1) + Xv $\\
 
 $ 63430,485(1+i_{*}^{(12)})^{100} = 900\frac{ 1 -  (1+i_{*}^{(12)})^{100}}{  1 - (1+i_{*}^{(12)} ) }  + Xv$\\
 
 $X= \frac{ 63430,485(1+i_{*}^{(12)})^{100}  -  900\frac{ 1 -  (1+i_{*}^{(12)})^{100}}{  - i_{*}^{(12)}  }  }{1/v}$\\
 
  $X= 1.0075(63430,485(1.0075)^{100}  -  900\frac{ 1 -  (1.0075)^{100}}{  - 0.0075  } ) $\\
  
  $X= 1.0075(63430,485(1.0075)^{100}  +  120000 ( 1 -  (1.0075)^{100} ) )$\\

$X= 1.0075(63430,485 \cdot 2.1110838400381224 +  120000 ( 1 -  2.1110838400381224 ) )$\\

$X= 1.0075(63430,485 \cdot 2.1110838400381224 -  120000  \cdot   1.1110838400381224)$\\

$X= 1.0075(63430,485(2.1110838400381224 -  120000  \cdot   1.1110838400381224)$\\

$X= 1.0075( 133907.07184928053 -133330.0608045747)$\\

$ \bf X = 581.338$\\

{\bf Замечание}
Если принимать за оставшийся долг  63180.76, то получим\\
 $X= \frac{ 63430,485(1+i_{*}^{(12)})^{100}  -  900\frac{ 1 -  (1+i_{*}^{(12)})^{100}}{  - i_{*}^{(12)}  }  }{1/v} = 581.97$


{\bf \large 7.} Эффективная годовая форвардная ставка для промежутка времени $\left[ t; t+r\right]$, где   t и r   измеряются годами, обозначена  $f_{t,r}$
  (иначе говоря,  $f_{t,r}$  – это  r-летняя форвардная ставка через t лет). Известно, что  $f_{0,1}=4\%,   f_{1,1}=5\%, f_{2,1}=6\%, f_{3,1}=7\%.$\\
(i) Определите доходность к погашению в момент эмиссии для четырёхлетней облигации, которая  гасится по номиналу и платит купон по ставке $4\%$ в конце каждого года.  [7 баллов]\\
(ii) Объясните, почему эта доходность меньше, чем $f_{3,1}.$   [3 балла]\\
(iii) Как вы могли бы интерпретировать тот факт, что последовательность   $f_{0,1}, f_{1,1}, f_{2,1}, f_{3,1}$  возрастающая. [4 балла].\\
$\left[\right. $Всего 14 баллов $\left.\right]$\\

{ \bf \large Решение:}
{\bf \large  (i)} Обозначим $P_t$ - текущая стоимость единичной суммы, которая должна быть выплачена в момент t.\\
Тогда текущая стоимость  четырехлетней облигации, которая  гасится по номиналу F и платит купон по ставке $4\%$  (то есть 0.04F) в конце каждого года равна:\\

$PV= 0.04F \cdot P_1 + 0.04F \cdot P_2 + 0.04F \cdot P_3  + 1.04F \cdot P_4$\\

А величины $P_1,P_2,P_3,P_4$ мы найдем, зная $f_{0,1}=4\%,   f_{1,1}=5\%, f_{2,1}=6\%, f_{3,1}=7\%$ по формуле (6.17) со страницы 289:\\

$ \boxed{ P_u= (1+f_{u,t})^{t} P_{u+t} }$\\

У нас $ t=1$\\

$\Rightarrow P_{u+1} = \frac {P_u}{1+f_{u,1} }$

Имеем:\\

$P_{0} = 1$\\

$P_1 = \frac {P_0}{1+f_{0,1}} =   \frac { 1 }{1.04} = 0.9615384615384615$

$P_2 = \frac {P_1}{1+f_{1,1}} =   \frac { 0.9615384615384615 }{1.05} = 0.9157509157509156$

$P_3 = \frac {P_2}{1+f_{2,1}} =   \frac { 0.9157509157509156 }{1.06} = 0.8639159582555808$

$P_4 = \frac {P_3}{1+f_{2,1}} =   \frac { 0.8639159582555808 }{1.07} = 0.8073980918276455$

$\Rightarrow PV=0.04F(P_1+ P_2 +P_3) + 1.04F \cdot P_4 =
 0.04F(0.9615384615384615+0.9157509157509156+0.8639159582555808) +1.04F \cdot 0.8073980918276455=
 F(0.04\cdot 2.741205335544958 + 1.04 \cdot 0.8073980918276455) = 
 F(0.10964821342179831 + 0.8396940155007514) = 0.9493422289225497F$

Обозначим доходность к погашению за YTM\\

Тогда YTM ищется из уравнения:\\

$0.04F \cdot a_{\overset{-}4 |} + Fv^4 = PV=0.9493422289225497F$

$4a_{\overset{-}4 |} + 100v^4 = 94.93422289225497$\\

Функция "=ВСД({-94,934223;4;4;4;104})"  выдает нам ответ\\

 $YTM=0.054433505 \approx \bf 5.4433\%$

{\bf \large  (ii)} Почему YTM оказалось меньше, чем $f_{3,1}=7\%$?\\
Вспомним, что YTM является своеобразным средним значением спотовых ставок $y_1, y_2, y_3, y_4$ - см. формулу (6.7) на стр. 278:\\

$ \boxed {\sum\limits_{k=1}^{n} C_k(1+YTM)^{-t_k} + F(1+YTM)^{-t_n} =  \sum\limits_{k=1}^{n} C_k(1+y_{t_k})^{-t_k} + F(1+y_{t_n})^{-t_n} }$

А в нашем случае все эти эти спотовые ставки оказались меньше $f_{3,1}=7\%$ : \\

$ 1 + y_1=1.04 \Rightarrow  y_1=4\%$\\

$ (1+y_2)^2 = 1.04 \cdot 1.05  \Rightarrow  y_2=(1.092)^{1/2} - 1= 0.04498803820905062 = 4.4988\%$\\

$ (1+y_3)^3 = 1.04 \cdot 1.05 \cdot 1.06   \Rightarrow  y_3=(1.15752)^{1/3} - 1=  0.04996825300838603 = 4.4968\%$\\

$ (1+y_4)^4 = 1.04 \cdot 1.05 \cdot 1.06 \cdot 1.07   \Rightarrow  y_4=(1.2385464)^{1/4} - 1=  0.05494075450104474= 5.494\%$\\

А раз все эти спотовые ставки оказались меньше $f_{3,1}=7\%$, то и их среднее будет меньше $f_{3,1}=7\%$.\\

Вообще можно было и не считать спотовые ставки, а сказать, что YTM является своеобразным средним значением спотовых ставок, 
а спотовые ставки являются средним геометрическим всех предыдущих однолетних форвардных ставок (см. формулу (6.24) на стр. 290):

$ \boxed { 1+y_n = ((1+f_0 )(1+f_1) \dots (1+f_{n-1}))^{1/n} }$, где $f_u  \equiv f_{u,1}$\\

А в нашем случае по условию все эти однолетние форвардные ставки меньше $f_{3,1}=7\%$, 
поэтому и спотовые ставки меньше $f_{3,1}=7\%$, поэтому YTM меньше $f_{3,1}=7\%$.\\


{\bf \large  (iii)}  Почему  может возрастать последовательность однолетних форвардных ставок?\\
Мы уже выяснили, что спотовые ставки являются средним геометрическим всех предыдущих однолетних форвардных ставок (см. формулу (6.24) на стр. 290):

$ \boxed { 1+y_n = ((1+f_0 )(1+f_1) \dots (1+f_{n-1}))^{1/n} }$, где $f_u  \equiv f_{u,1}$\\

Поэтому раз возрастает последовательность однолетних форвардных ставок, то возрастает и последовательность спотовых ставок, а это можно объяснить тремя способами:\\

1) (чистая теория ожиданий) инвесторы ожидают роста спотовых краткосрочных процентных ставок\\
2)(теория предпочтения ликвидности) инвесторы учитывают риски владения долгосрочными облигациями и требуют за них увеличенную доходность\\
3)(теория сегментации рынка) краткосрочные и долгосрочные облигации интересуют разные группы людей, и в данный момент на рынке сложился повышенный спрос( или недостаточное предложение)  на краткосрочные бумаги и имеется избыточное предложение ( или недостаточный спрос)  долгосрочных.\\




{\bf \large 8.}  Стоимость активов инвестиционного фонда 1 января 2015 года была £100m, но через два года, 1 января 2017 года, он оценивался только в £64.\\ Немедленно после оценки фонда 1 января 2017 года в фонд поступила сумма £16m и к 1 июля 2018 года стоимость фонда выросла до £270m.\\
(i) Дайте определение средней по времени эффективной годовой ставки дохода, TWRR. В каких случаях её разумно использовать? [2 балла]\\
(ii) Вычислите TWRR за период с 1 января 2015 года до 1 июля 2018 года; ответ округлите до целого числа базисных пунктов. [4 балла]\\
(iii) Дайте определение эквивалентной по финансовому результату ставки дохода, MWRR. В каких случаях её разумно использовать? [2 балла]\\
(iv) Докажите, что для рассматриваемого примера\\
(a) MWRR за период с 1 января 2015 года до 1 июля 2018 года существует; [4 балла]\\
(b) верно неравенство 29\% < MWRR < 30\%. [2 балла]\\
(c) вычислите MWRR; ответ округлите до целого числа базисных пунктов. [4 балла]\\
$\left[\right. $Всего 18 баллов $\left.\right]$\\

{ \bf \large Решение:}
{\bf \large  (i)} Величина TWRR, применяемая для оценки эффективности работы например фонда,  находится из условия\\

$  \boxed { (1+TWRR)^{T} = (1+r_1^{eff}) \dots (1+r_n^{eff}) } $, \\

 где $r_n^{eff} = \frac{ F(t_n)-0) - F(t_{n-1} +0) }{ F(t_{n-1} +0)}$, \\

где $F(t_n+0)$- стоимость фонда в начале промежутка $\left[ t_n+0; t_{n+1}-0\right]$, \\

а $F(t_{n+1}-0)$- стоимость фонда в конце промежутка $\left[ t_n+0; t_{n+1}-0\right]$\\

TWRR разумно использовать для оценки эффективности деятельности например управляющего фонда, 
потому что он никогда не знает, когда деньги поступят в фонд  или уйдут из него, а только может выбирать виды активов, в которые нужно инвестировать. 
И 1+ TWRR как раз  является взвешенным средним геометрическим эквивалентных коэффициентов годового роста для промежутков, где средства фонда росли/ убывали только за счет деятельности управляющих фонда.\\

{\bf \large  (ii)}  Примем момент 1 января 2015 за $t=0$  и будем измерять время годами, а деньги миллионами фунтов. \\
У нас T=3.5 \\
$F(0+0) = 100, F(2-0) = 64, F(2+0) = 80, F(2.5)=270$\\

$\Rightarrow (1+TWRR)^{3.5} = \frac {64}{100} \frac{270}{80} = 0.64 \cdot 3.375 = 2.16$

$\Rightarrow TWRR=  (2.16)^{1/2.5} - 1= 0.24611526 = \bf 24.6115\%$

{\bf \large  (iii)} Величина MWRR  -  это внутренняя ставка доходности IRR для фондов, 
то есть корень уравнения доходности, причем только в том случае, если это уравнение имеет единственный корень на промежутке  $ -1 < i < +\infty$\\

Уравнение доходности:\\
$F(0) + \sum\limits_{n=1}^{N} M(t_n) (1+i)^{-t_n} = F(T)(1+i)^{-T}$, где F(t) -стоимость фонда в момент t, а M(t) - сумма новых денег.\\

MWRR разумно использовать для оценки эффективности  инвестиционной деятельности индивидуальных инвесторов, поскольку они сами принимают решение о том, когда и сколько денег вкладывать.\\

{\bf \large  (iv) (a) } Запишем уравнение доходности для фонда (в момент $t=2$):\\

$100z^2 + 16 = 270z^{-1.5}$\\

$f(z)= 100z^{3.5} + 16z^{1.5} - 270 = 0$\\

Поскольку $100>0$ и $270>0$, то функция f(z) на промежутке $z>0$ монотонно возрастает от $-\infty$ до $+\infty$.\\

Поэтому уравнение $f(z)=0$ имеет единственный положительный корень, который и будет $k_{TW}$\\

{\bf (b)} Почему 29\% < MWRR < 30\%?\\

$f(1.29) = -2.740464789783573 < 0$\\
$f(1.30) = 4.21218973634273650 > 0$\\

То есть f имеет разные знаки на концах отрезка $\left[ 1.29, 1.3\right]$. \\
Значит,  f  имеет корень внутри этого отрезка.

{\bf (c)} Ищем корень уравнения  $f(z)= 100z^{3.5} + 16z^{1.5} - 270 = 0$ с помощью функции "=ЧИСТВНДОХ({100;16;-270};{42005;42736;43282};0,293965)",

где мы предварительно перевели даты (01.01.2015, 01.01.2017, 01.07.2018)  в их серийные номера (42005 ,42736 ,43282);\\

Получаем ответ  $MWRR= 0.294125372 =  \bf 29.4125\%$\\

Отметим, что этот корень не является корнем уравнения, тк при подстановке в уравнение не дает ноль.\\

 Корнем является  $MWRR= 0.2935 =  \bf 29.39\%$\



{\bf \large 9.} Инвестор предполагает вложить сумму P = £6 000 в некоторый фонд на n =10 лет. 
Он моделирует неопределённость в изменении стоимости активов фонда предположением, 
что годовая доходность от вложения средств в фонд является случайной величиной. \\
Пусть $i_k$ – эта доходность за k-й год.\\
Инвестор предполагает, что случайные величины $i_1, i_2 \dots i_n $ независимы в совокупности 
и одинаково распределены со средним 8\% и стандартным отклонением 7\%, 
причём годовые коэффициенты роста  $1 +  i_k$ имеют логнормальное распределение.\\
(i) Вычислите ожидаемый размер суммы S10, которую получит в результате инвестор. [2 балла]\\
(ii) По какому закону распределена сумма S10 ? [2 балла]\\
(iii) На какую максимальную сумму может рассчитывать инвестор через 10 лет практически гарантированно при доверительной вероятности 97.5\%? [12 баллов]\\
$\left[\right. $Всего 16 баллов $\left.\right]$\\


{ \bf \large Решение:}
{\bf \large  (i)}  
Обозначим за s10 накопление от \pounds 1, инвестированного на 10 лет.\\

$s1= (1+i_1) \dots (1+i_10)$\\

 Поскольку  случайные величины $i_1, i_2 \dots i_n $ независимы в совокупности  и одинаково распределены со средним 8\% , то \\

$Es10 = E(1+i_1)\dots (1+i_{10}) = (E(1+i_1))^{10} = (1+Ei_1)^{10} = 1.08^{10} = 2.158925$\\

$\Rightarrow ES10 = E6000s10 = 6000Es10 = 6000\cdot  2.158925 = \bf \pounds 12953.55$\\


{\bf \large  (ii)}  Величина $\eta $ имеет логнормальное распределение с параметрами a и $\sigma^2$, 
если $\eta  = e^ {a+\sigma \xi }$  , где  $\xi  \sim  N(0,1)$\\

Известно, что для логнормальной величины $\eta$ с параметрами a и $\sigma^2$ среднее значение $E\eta$ и дисперсия $Var\eta$ даются формулами \\

$ \boxed {E\eta = e^{a+ \frac {\sigma^2} {2}}}, $\\
$ \boxed {Var\eta = e^{2a + \sigma^2}(e^{\sigma^2 }- 1)}$\\

Из этих равенств выражаем параметры a и $\sigma^2$ логнормальной величины  $\eta=(1+i_k)$  через ее матожидание и дисперсию:\\

$ \boxed {\sigma^2 = ln(1+\frac{Var\eta}{ (E\eta)^2})}$\\

$\boxed {a = ln(E\eta) - 0.5\sigma^2}$\\

В нашем случае $E\eta=E(1+i_k)  = 1.08$\\

$Var\eta= Var(1+i_k) = Var(i_k) = 0.0049$\\

$\Rightarrow  \sigma^2 = ln(1+\frac{Var\eta}{ (E\eta)^2}) = ln(1 + \frac{0.0049}{ 1.08^2} = ln(1.004200960219479) = \underline {0.004192160821435326}$\\

$ a = ln(E\eta) - 0.5\sigma^2= ln(1.08) - 0.002096080410717663= 0.0769610411361284- 0.002096080410717663=  \underline {0.07486496072541073}$\\

А раз $ln(1+i_k) \sim  N(a,\sigma^2)$, 
а при сложении нормальных величин получается снова нормальная величина с параметрами, равными сумме параметров слагаемых, то\\

$lns10 = \sum\limits_{k=1}^{10} ln(1+i_k) \sim N(10a,10\sigma^2)$,\\

то есть $s10 \sim  e^{10a + \sigma\sqrt {10} \xi } $\\

$\Rightarrow S10= 6000s10 \sim 6000e^{10a + \sigma\sqrt {10} \xi } = e^{ln(6000)+10a + \sigma\sqrt {10} \xi}$\\


$\Rightarrow S10 \sim LogN ( 10a +ln(6000); 10\sigma^2)$\\

$10a + ln(6000) = 0.7486496072541073 + 8.699514748210191 = 9.448164355464298$\\

$10\sigma^2 = 0.04192160821435326$\\

$\Rightarrow  \bf {S10 \sim  LogN(9.448164; 0.0419216)} $\\


{\bf \large  (iii)}  Мы ещем S такое, что $P(S10>S) = 0.975$\\

$\Rightarrow P(ln(S10) > ln(S))=0.975$\\

$ P(\frac {ln(S10) - 9.448164}{0.0419216} > \frac {ln(S) - 9.448164}{0.0419216}) = 0.975$\\

Поскольку $\frac {ln(S10) - 9.448164}{0.0419216}$ имеет стандартное нормальное распределение, то 

$ \frac {ln(S) - 9.448164}{\sqrt {0.0419216}} = z_{0.025} = -1.96$\\

$\Rightarrow S = e^{-1.96 \cdot \sqrt {0.0419216} + 9.448164} =    e^{-1.96 \cdot  0.20474766961885857+ 9.448164} =  \\
e^{-0.4013054324529628+ 9.448164}  = e^{9.046865941628532} =  \pounds 8491.885140274939= \bf \pounds 8491.89$





\end{document}



