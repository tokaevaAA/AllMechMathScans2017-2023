
\documentclass{article}

\usepackage[T2A]{fontenc}  
\usepackage[utf8]{inputenc}
\usepackage[english,russian]{babel}
\usepackage{indentfirst}
\usepackage{misccorr}
\usepackage{graphicx}
\usepackage{amsmath}
\usepackage{amssymb}

\begin{document}


{\bf Развернутые ответы на вопросы 1(i, ii), 2(i) , 7, 9, 10(i ,ii, iii) }\\

{\bf \large1.}\\
{\bf \large  (i)}  Символ  $_{4|5}q_{[60]+1} $ обозначает вероятность того, что человек, которому сейчас 61 год и который прошел отбор 1 год назад, проживет еще 4 года, но умрет на протяжении последующих 5 лет, то есть его остаточное время жизни удовлетворяет условию $4 < T_{[60]+1} < 9$. \\
То есть $_{4|5}q_{[60]+1} = P(4 < T_{[60]+1} < 9) $.\\
{\bf \large  (ii)} Таблица АМ92 - это таблица, используемая для расчета смертности мужчин, застрахованных по договору пожизненного или смешанного страхования. Она является таблицей отбора риска с периодом отбора , действующим 2 года. Таблица подготовлена комитетом ИФА по непрерывному исследованию смертности ( Continuous Mortality Investigation - CMI) на основе британской страховой статистики за 1991-1994 гг.\\

{\bf \large2.}\\
{\bf \large  (i)} Таблица ELT15 (Females)-это английская таблица продолжительности жизни  для женщин (English Life Tables – ELT),
основанная на популяционной статистике смертности в Англии и Уэльсе за 1990, 1991, 1992 гг. (для численности населения используются не данные переписи, а оценки для середины года).  Таблица публикуется Управлением национальной статистики (The Office for National Statistics) Великобритании каждые 10 лет после очередной переписи населения. Техническую работу по составлению таблиц проводит Служба Правительственного Актуария (Government Actuary’s Department).\\
Таблица  ELT15 (Females) отличается от AM92 прежде всего тем, что АМ92 является таблицей с отбором, а ELT15(Females) является таблицей без отбора; 
во-вторых, ELT15 (Females) - относится к женщинам, а AM92 - к мужчинам; наконец, таблицы публикуются разными организациями.


{\bf \large7.}\\
{\bf \large Ответ:}\\ б) перспективный метод\\
Это написано в пункте 1.3 этого закона:\\
 В соответствии с настоящим Порядком страховые резервы по страхованию жизни формируются при условии, если методы расчета страховых тарифов основаны на уравнении эквивалентности (равенстве актуарной стоимости страховых выплат по ПРЕДСТОЯЩИМ  страховым случаям и актуарной стоимости страховых премий (поступлений) на начало срока действия договора страхования), с применением таблиц смертности, заболеваемости, инвалидности, начислении в период действия договора страхования нормы (ставки) доходности. Актуарная стоимость страховых выплат (страховых поступлений) вычисляется путем дисконтирования ПРЕДПОЛАГАЕМЫХ денежных выплат (поступлений) с учетом вероятностей, с которыми такие денежные выплаты (поступления) БУДУТ произведены.\\

{\bf \large9.}\\
{\bf \large Решение:}\\

$ \overline a_{x: \overline{n}| } = \int\limits_{0}^{n} v^{t} P(T_x > t) dt = \int\limits_{0}^{n} v^{t} \frac{s(x+t)}{s(x)}dt =   \int\limits_{0}^{n} e^{-\delta t} \frac{s(x+t)}{s(x)}dt $\\

Найдем $s(x)$, зная вид интенсивности смертности: \\
$s(x) = e^{-\int\limits_{0}^{x} \mu_t dt}$\\

$ \int\limits_{0}^{x} \mu_t dt = \int\limits_{0}^{x} \frac{t+b}{a} dt = \frac {(t+b)^2}{2a} |_{0}^{x} = \frac{1}{2a} ((b+x)^2 - b^2) = \frac{x^2 + 2bx}{2a}$\\

$\Rightarrow  s(x) = e^{-\int\limits_{0}^{x} \mu_t dt} = e^{- \frac{x^2 + 2bx}{2a}} $\\

$s(x+t) = e^{- \frac{(x+t)^2 + 2b(x+t)}{2a}}  = e^{- \frac{x^2 +2xt + t^2 +2bx + 2bt }{2a}} $\\

$\Rightarrow  \frac{s(x+t)}{s(x)} = \frac{ e^{- \frac{x^2 +2xt + t^2 +2bx + 2bt }{2a}} }{ e^{- \frac{x^2 + 2bx}{2a}}  } =  e^{- \frac{ t^2 +2t(x+b)  }{2a}}$\\


$\Rightarrow  {\bf  \overline a_{x: \overline{n}| }  } = \int\limits_{0}^{n} e^{-\delta t} \frac{s(x+t)}{s(x)}dt = 
\int\limits_{0}^{n} e^{-\delta t}  e^{- \frac{ t^2 +2t(x+b)  }{2a}} dt=
 \int\limits_{0}^{n}   e^{- \frac{ t^2 +2t(x+b+a\delta)  }{2a}} dt=
 \int\limits_{0}^{n}   e^{- \frac{ (t + x+b+a\delta)^2 }{2a}}  e^{ \frac{ (x+b+a\delta)^2  }{2a}}  dt =
 e^{\frac{A^2}{2} }  \int\limits_{0}^{n}   e^{- \frac{ (t + x+b+a\delta)^2 }{2a}}    dt=\\
 \{  x=\frac{ t + x+b+a\delta }{\sqrt{a}} ; dx = \frac{dt}{\sqrt{a}} ; dt = \sqrt{a}dx\}=
 \sqrt{a} e^{\frac{A^2}{2} }  \int\limits_{A}^{A + \frac{n}{ \sqrt{a} }}   e^{- \frac{ x^2}{2}}    dx=
  \sqrt{2 \pi a} e^{\frac{A^2}{2} }   \frac{1}{ \sqrt{2\pi} }\int\limits_{A}^{A + \frac{n}{ \sqrt{a} }}   e^{- \frac{ x^2}{2}}    dx =
 { \bf \sqrt{2 \pi a} e^{\frac{A^2}{2} }  (\Phi (A + \frac{n}{\sqrt a }) - \Phi(A)) }$\\



{\bf \large10.}\\
{\bf \large  (i)} Проспективный резерв –это   разность между  современной стоимостью предстоящих расходов (включая выплаты по договору страхования жизни) 
  (т.е. обязательств страховщика)  и   современной стоимостью предстоящих страховых премий по договору страхования жизни (т.е. обязательств страхователя).\\
  
  {\bf \large  (ii)} Условия, которые достаточны для того, чтобы проспективный резерв был равен ретроспективному резерву:\\
  1) Премии определяются исходя из принципа эквивалентности обязательств\\
  2) При расчете проспективного резерва, ретроспективного  резерва и премии использовался один и тот же базис.\\
 Действительно, если  премия по некоторому виду страхования или пенсионной схеме определена из принципа эквивалентности, то в среднем компания не должна 
 привлекать собственные средства для выполнения финансовых   обязательств перед клиентами. Это означает, что резерв в момент t, необходимый для выполнения 
 будущих финансовых обязательств по каждому еще действующему договору, должен быть равен сумме, накопленной к моменту t на каждый действующий договор.\\

 {\bf \large  (iii)} Допустим, что в конце $k$-го года (т.е. в момент $t = k$) договор все еще сохраняет силу (так что застрахованный еще жив и его возраст равен $x+k$).\\
 
При подсчете перспективным методом:\\
$_{k}V = _{k}a_{B} - _{k}a_{C}$\\

$ \bullet _{k}a_{B} = \sum\limits_{j=k+1}^{n}   Sv^{j-k}P(j-k-1 < T_{x+k} < j-k)  = \\
\sum\limits_{j=k+1}^{n}   Sv^{j-k}P(T_{x+k} > j-k-1) q_{x+j-1} =  \sum\limits_{j=k+1}^{n}   Sv^{j-k}  \frac {P(T_{x} >  j-1)}  { P(T_{x} > k)} q_{x+j-1}$\\

$ \bullet _{k}a_{c} = \sum\limits_{j=k+1}^{n}   P_{j} v^{j-k-1}P( T_{x+k} >  j-k-1)  =  \sum\limits_{j=k+1}^{n}   P_{j} v^{j-k-1}  \frac {P(T_{x} >  j-1)}  { P(T_{x} > k)} $\\

$\Rightarrow _{k}V = _{k}a_{B} - _{k}a_{C} = \frac{1}{v^{k}P(T_{x} >k) } \sum\limits_{j=k+1}^{n} v^{j}P(T_{x} > j-1) (Sq_{x+j-1} - P_{j}(1+i))$\\

Преобразуем последнюю сумму с учетом того, что \\
$ \sum\limits_{j=1}^{n} v^{j}P(T_{x} > j-1) (Sq_{x+j-1} - P_{j}(1+i)) = 0$ 
в силу принципа эквивалентности  обязательств, согласно которому и вычислялись премии.\\

$ _{k}V = \frac{1}{v^{k}P(T_{x} >k) } \sum\limits_{j=k+1}^{n} v^{j}P(T_{x} > j-1) (Sq_{x+j-1} - P_{j}(1+i))=\\
\frac{1}{v^{k}P(T_{x} >k) }  (\sum\limits_{j=1}^{n} v^{j}P(T_{x} > j-1) (Sq_{x+j-1} - P_{j}(1+i)) - \sum\limits_{j=1}^{k} v^{j}P(T_{x} > j-1) (Sq_{x+j-1} - P_{j}(1+i)))=
- \frac{1}{v^{k}P(T_{x} >k) }  \sum\limits_{j=1}^{k} v^{j}P(T_{x} > j-1) (Sq_{x+j-1} - P_{j}(1+i)) =  \\
\frac{1}{v^{k}P(T_{x} >k) }  \sum\limits_{j=1}^{k} v^{j}P(T_{x} > j-1) ( P_{j}(1+i)  - Sq_{x+j-1}  ) $\\

Теперь заметим, что последняя формула в точности равна ретроспективному резерву, потому что там написана разность между  текущей стоимостью 
накопления к моменту $k$ за счет премий и  текущей стоимостью накопления   к моменту $k$ от всех выплат на промежутке $(0,k)$.\\
Поэтому мы доказали, что при выполнении условий, упомянутых в пункте (ii), в любой момент времени t (число t – натуральное) проспективный резерв равен ретроспективному резерву.\\




{\large \bf Решения:}\\



{\bf \large1.}  (i) Разъясните смысл обозначения $_{4|5}q_{[60]+1} $.\\
 (ii) Что такое таблица AM92 ?\\
(iii) Вычислите значение величины  $_{4|5}q_{[60]+1} $ , используя следующий фрагмент таблицы AM92.\\

{\bf \large Решение:}\\

{\bf \large  (i)}  Символ  $_{4|5}q_{[60]+1} $ обозначает вероятность того, что человек, которому сейчас 61 год и который прошел отбор 1 год назад, проживет еще 4 года, но умрет на протяжении последующих 5 лет, то есть его остаточное время жизни удовлетворяет условию $4 < T_{[60]+1} < 9$. \\
То есть $_{4|5}q_{[60]+1} = P(4 < T_{[60]+1} < 9) $.\\

{\bf \large  (ii)} Таблица АМ92 - это таблица, используемая для расчета смертности мужчин, застрахованных по договору пожизненного или смешанного страхования. Она является таблицей отбора риска с периодом отбора , действующим 2 года. Таблица подготовлена комитетом ИФА по непрерывному исследованию смертности ( Continuous Mortality Investigation - CMI) на основе британской страховой статистики за 1991-1994 гг.\\

{\bf \large  (iii)} Воспользуемся формулой (5.3.9) на стр. 101 пособия Г.И.Фалин. Математические основы теории страхования жизни и
пенсионных схем. 3-е издание: АНКИЛ, Москва, 2007. 304 c. ISBN 978-5-86476-235-6.\\

$_{n|m}q_{[x] + t} = \frac{     l_{[x]+t+n} -  l_{[x]+t+n+m}   }{     l_{[x]+t}}$\\

$\Rightarrow  _{4|5}q_{[60] + 1} = \frac{     l_{[60]+1+4} -  l_{[60]+1+4+5}   }{     l_{[60]+1} } = \frac {  l_{65} - l_{70}   }{    l_{[60]+1} }   $

В последнем равенстве мы воспользовались тем, что период отбора равен 2 года, поэтому в числителе квадратные скобки можно убрать.\\

Из таблицы АМ92: $   l_{65}=8821.2612,  l_{70}=8054.0544,  l_{[60]+1} = 9209.6568$\\

Следовательно\\

$ _{4|5}q_{[60] + 1} = \frac{     l_{[60]+1+4} -  l_{[60]+1+4+5}   }{     l_{[60]+1} } = \frac {  l_{65} - l_{70}   }{    l_{[60]+1} }  
 =   \frac {  8821.2612 -  8054.0544  }{     9209.6568 }  =   \frac {  767.2068  }{     9209.6568 }  \approx \bf 0.0833 $\\ 
 {\bf \large  Ответ:} $\boxed { _{4|5}q_{[60] + 1} \approx \bf 0.0833} $\\



{\bf \large 2.}  (i) Что такое ELT15 (Females)? Чем она отличается от AM92 ?\\
(ii) Вычислите $_{2.75}q_{84.5 }$ используя метод равномерного распределения смертей.\\
База расчётов: ELT15 (Females), фрагмент которой приведён ниже.\\
{\bf \large Решение:}\\
{\bf \large  (i)} Таблица ELT15 (Females)-это английская таблица продолжительности жизни  для женщин (English Life Tables – ELT),
основанная на популяционной статистике смертности в Англии и Уэльсе за 1990, 1991, 1992 гг. (для численности населения используются не данные переписи, а оценки для середины года).  Таблица публикуется Управлением национальной статистики (The Office for National Statistics) Великобритании каждые 10 лет после очередной переписи населения. Техническую работу по составлению таблиц проводит Служба Правительственного Актуария (Government Actuary’s Department).\\

Таблица  ELT15 (Females) отличается от AM92 прежде всего тем, что АМ92 является таблицей с отбором, а ELT15(Females) является таблицей без отбора; 
во-вторых, ELT15 (Females) - относится к женщинам, а AM92 - к мужчинам; наконец, таблицы публикуются разными организациями.

{\bf \large  (ii)}  По общей формуле:\\
$ _{t}q_x = P(T_x \leq t) = P(T-x \leq t | T>x ) = \frac { P(x < T \leq x+t) } { P(T > x) }  = \frac { s(x) - s(x+t)} { s(x)} $

Поэтому $ _{2.75}q_{84.5 } =   \frac { s(84.5) - s(87.25)} { s(84.5)} $

Найдем $s(x)$ используя метод равномерного распределения смертей.\\
Согласно нему, предполагается, что  $s(x)$ между целочисленными узлами интерполируется линейными функциями:\\

$s(x) = a_n + b_nx$ для $x \in [n,n+1]$\\
Используя тот факт, что в концевых точках значения равны $s(n)$ и $ s(n+1)$ соответственно, находим, что \\

$s(x) = (n+1-x)s(n) + (x-n)s(n+1)$\\

Подставив $x=n+t; t \in [0,1],$ получим:\\

$ s(n+t) =  (1-t)s(n) + ts(n+1),   t \in [0,1] $ \\

Поэтому\\
 $s(84.5) = 0.5s(84) + 0.5s(85)$\\
  $s(87.25) = 0.75s(87) + 0.25s(88)$\\
 
 
$ \Rightarrow  _{2.75}q_{84.5 } =   \frac { s(84.5) - s(87.25)} { s(84.5)}  = \frac {   0.5s(84) + 0.5s(85) -  0.75s(87) - 0.25s(88)   } {    0.5s(84) + 0.5s(85)  }$  \\

$=  \frac {   0.5l_{84} + 0.5l_{85}-  0.75l_{87}- 0.25l_{88}  } {    0.5l_{84} + 0.5l_{85} }  =  
 \frac {   0.5 \dot 41736 + 0.5 \dot 38081-  0.75 \dot 30651- 0.25 \dot 27017  } {   0.5 \dot 41736 + 0.5 \dot 38081}  
  =   \frac {   20868 + 19040.5 -  22988.25 - 6754.25  } {  39908.5}  = \frac { 10166}{ 39908.5} \approx  \bf 0.2547327$\\
 {\bf \large  Ответ:} $\boxed { _{2.75}q_{84.5 } \approx \bf 0.2547327}$\\




{\bf \large 3.}(i) Известно, что $q_{67}= 17.824$ \textperthousand , а на промежутке [67;68] интенсивность смертности является постоянной величиной. Найдите её. \\
(ii) Подсчитайте значение $_{0.5}q_{67.25 }$ , используя предположение о постоянной интенсивности смертности и результат, полученный в пункте (i). 

{\bf \large Решение:}\\

{\bf \large  (i)}  Напомним, что интенсивность смертности $\mu_x = \frac {F'(x)} {1-F(x)} = -\frac {s'(x)} {s(x)} $\\

Поэтому $s(x) = e^{-\int\limits_{0}^{x} \mu_u du}$\\

Обозначим постоянную на  [67;68] интенсивность смертности за $\mu.$\\

Тогда для $x \in [67;68] $ : \\
 $s(x) =  e^{-\int\limits_{0}^{67} \mu_u du}  e^{-\int\limits_{67}^{x} \mu_u du} = s(67) e^{-\int\limits_{67}^{x} \mu_u du} = s(67)e^{-\mu(x-67)}$\\
 
 В частности,  $s(68) = s(67)e^{-\mu}$\\
 
 Но по общей формуле:\\
  $q_x = P(T_x < 1) = P(T-x < 1 | T>x) = \frac {P(x < T<x+1)}{ P(T>x)} = \frac{ s(x) - s(x+1)}{ s(x)}$\\
  
  Поэтому:
  
 $q_{67} = \frac{ s(67) - s(68)}{ s(67)} = \frac{ s(67) - s(68)}{ s(67)}  = \frac{ s(67) - s(67)e^{-\mu} }{ s(67)} = 1-e^{-\mu}$\\
 
 $\Rightarrow \mu = -ln(1-q_{67}) = -ln(1-0.17824) = -ln(0.982176) \approx \bf  0.01798476$\\
 
 
 {\bf \large  (ii)} По общей формуле:\\
$ _{t}q_x = P(T_x \leq t) = P(T-x \leq t | T>x ) = \frac { P(x < T \leq x+t) } { P(T > x) }  = \frac { s(x) - s(x+t)} { s(x)} $

Поэтому $ _{0.5}q_{67.25 } =   \frac { s(67.25) - s(67.75)} { s(67.25)} $\\

Найдем $ s(x)$  используя предположение о постоянной интенсивности смертности.\\
Согласно нему, предполагается, что  $s(x)$ на отрезке $  [n; n+1]$  показательной функцией $ a_n e^{b_nx}$.\\

Используя тот факт, что в концевых точках значения равны $s(n)$ и  $s(n+1) $ соответственно, находим, что  для  $x \in [67;68] $\\

$s(x) = s(n) {p_n}^{x-n}$, где $p_n = \frac {s(n+1)} {s(n)} $\\



$\Rightarrow s(n+t) = s(n) {p_n}^{t}$ для $t \in [0,1]$\\

Поэтому $s(67.25) = s(67) p_{67}^{0.25}$\\
$s(67.75) = s(67) p_{67}^{0.75}$\\

Следовательно:\\

$ _{0.5}q_{67.25 } =   \frac { s(67.25) - s(67.75)} { s(67.25)} = \frac{  s(67) p_{67}^{0.25} -  s(67) p_{67}^{0.75} } { s(67) p_{67}^{0.75}}  = 
1-  p_{67}^{0.5} = 1-\sqrt{ p_{67}} =  1-\sqrt {1- q_{67}} =  1- \sqrt{1-0.017824} = 1-\sqrt{0.982176} \approx \bf 0.008952$ \\
 {\bf \large  Ответ:} $\boxed {\mu \approx \bf  0.01798476;  _{0.5}q_{67.25 } \approx \bf 0.008952}$\\
					    
					    

{\bf \large 4.}Страховой агент получает вознаграждение, если по заключенным им договорам убыточность меньше чем $l=70\%$ . Известно, что:\\
1. убыточность рассчитывается как отношение всех выплаченных страховых возмещений к собранным премиям;\\
2. агент получает долю от собранной премии, равную $f=\frac{1}{3}$ разности между порогом $l=70\%$ и убыточностью;\\
3. вознаграждение не платится, если убыточность больше $70\%$;\\
4. агент заключил ряд договоров с общей премией P = 500 тыс. рублей;\\
5. суммарные выплаты L по договорам (в тыс. руб.) распределены по закону Парето со средним 600 и коэффициентом вариации $\sqrt3$ .\\
Подсчитайте ожидаемое вознаграждение R .\\

{\bf \large Решение:}\\

Возьмем 1000 рублей за единичную денежную сумму;\\

Напомним, что случайная величина $Y$ имеет распределение Парето с параметрами $\lambda > 0$ и $a>0$, если ее плотность дается формулой\\

$f(x) = \frac{a}{\lambda} (\frac{\lambda}{\lambda+x})^{a+1}; 0<x< +\infty$\\

Поэтому $F(x) = 1 - (\frac{\lambda}{\lambda+x})^{a+1}$\\

Тогда $EY= \int\limits_{0}^{\infty} (1-F(x))dx = \frac{\lambda}{a-1}$\\

$EY^2= \int\limits_{0}^{\infty} x(1-F(x))dx = \frac{2\lambda^2}{(a-1)(a-2)}$\\

$VarY = EY^2-(EY)^2 = \frac{2\lambda^2}{(a-1)(a-2)} -  \frac{\lambda^2}{(a-1)^2} = \frac{\lambda^2 a }{(a-1)^2(a-2)} $\\

$\sigma_Y = \sqrt{VarY}=  \frac{\lambda}{a-1} \sqrt{ \frac{a}{a-2}} $\\

$c_Y = \frac {\sigma_Y}{EY} = \sqrt{  \frac{a}{a-2}}$\\

По условию $EY= 600=  \frac{\lambda}{a-1}$\\
$ c_Y=\sqrt{3} = \sqrt{  \frac{a}{a-2}}$\\

Отсюда находим, что $a=3, \lambda=1200$\\

Но $R= \frac{1}{3} P *  (0.7 - \frac{L}{P}) * I\{ \frac{L}{P} <0.7\} =  \frac{1}{3}   (350 - L) *  I\{ L < 350 \}$\\

Поэтому $ER = \int\limits_{0}^{350}   \frac{1}{3}  (350 - x) f_L(x)dx = \\
 = \frac{1}{3}   \int\limits_{0}^{350}   (350 - x)    \frac{3}{1200} (\frac{1200}{1200+x})^{4}  dx=\\
=  \frac{1}{1200}   \int\limits_{0}^{350}   (350 - x)   (\frac{1200}{1200+x})^{4}  dx = \\
=  (1200)^3   \int\limits_{0}^{350}   (350 - x)   (1200+x)^{-4}  dx =\\
= \frac{ (1200)^3}{-3}   \int\limits_{0}^{350}   (350 - x)  d ((1200+x)^{-3})=\\
=  -\frac{ (1200)^3}{3} (350-x)   \frac{1}{(1200+x)^3}  |_{0}^{350}  +    \frac{ (1200)^3}{3}  \int\limits_{0}^{350}   (1200+x)^{-3} d(350-x)=\\
=    \frac{ (1200)^3}{3} 350   \frac{1}{(1200+x)^3}   -    \frac{ (1200)^3}{3}  \int\limits_{0}^{350}   (1200+x)^{-3} dx=\\
=     \frac{ 350}{3}  -    \frac{ (1200)^3}{3}   \frac{ 1}{-2}   \frac{1}{(1200+x)^2}   |_{0}^{350}  =\\
= \frac{ 350}{3}  +    \frac{ (1200)^3}{6}    \frac{1}{(1550)^2}    -    \frac{ (1200)^3}{6}    \frac{1}{(1200)^2}  =\\
 = \frac{ 350}{3}  +    \frac{ 10(120)^3}{6(155)^2}     -    \frac{ 1200}{6}    =\\
 = \frac{ -250}{3}  +    \frac{ 5(120)^3}{3(155)^2}   =  \frac{  - 250(155)^2 + 5(120)^3}{3(155)^2}  =  \\
 =\frac{  - 10(155)^2 + 24(120)^3}{3(31)^2}  = \frac{  - 240250 + 345600}{2883} =   \frac{ 105350}{2883}   = 36.54179673950746 $ тыс. руб\\
 
 $\Rightarrow ER \approx \bf  36541.80$ руб\\
  {\bf \large  Ответ:} \boxed {ER \approx \bf  36541.80 руб}\\
 
 
 
 
 
 {\bf \large 5.}  Страховщик только что заключил с человеком в возрасте $x= 80$ дискретный договор временного страхования жизни на срок $n =10$ лет со страховой суммой  $SA = \$100 000$ . По условиям договора страхователь вносит постоянную премию P в начале каждого года действия договора. Рассчитайте эту премию при следующих предположениях:\\
1. расходы и другие нагрузки не учитываются;\\
2. остаточное время жизни застрахованного описывается законом Мэкама с параметрами $A_x = 0.0001 , B_x = 0.1 , a_x = 0.075 $;\\
3. для дисконтирования используется кривая доходности $y_t$ , значения которой приведены в следующей таблице:

 \begin{tabular}{ccc}
$y_1, y_2, y_3, y_4, y_5, y_6, y_7, y_8, y_9, y_{10}$\\
$3.2\% , 3.5\%, 3.8\%, 4.1\% , 4.3\%, 4.5\%, 4.6\% , 4.7\%, 4.8\%, 4.8\%$ \\
\end{tabular}\\

{\bf \large Решение:}\\
Воспользуемся принципом эквивалентности обязательств:\\
Страховщик обязан выплатить сумму $SA$ в конце года смерти человека, если последний умрет за время действия договора, 
и не должен ничего выплачивать, если человек проживет эти 10 лет. \\
Такие обязательства равны $SA \cdot  A_{x: \overline{n}| }^{1}$\\
Страхователь обязан платить сумму $P$ в начале каждого года на протяжении 10 лет (при условии, что он еще жив, конечно).\\
Такие обязательства равны $P  \cdot  \ddot a_{x: \overline{n}| } $\\
Тогда по принципу эквивалентности обязательств:\\
$P= SA \frac{  A_{x: \overline{n}| }^{1} }{   \ddot  a_{x: \overline{n}| }  }$\\

Если бы у нас ставка дисконтирования не зависела от времени, то\\
$A_{x: \overline{n}| }^{1} = \sum\limits_{k=0}^{n-1} v^{k+1} P(K_x = k)$, \\
где $P(K_x=k) = P(k < T_x < k+1) = \frac {s(x+k)-s(x+k+1)}{s(x)}$\\
Но у нас ставка зависит от времени, то есть для дисконтирования используется кривая доходности, поэтому вместо $v^{k+1} $ надо писать $(1+y_{k+1})^{-k-1}$\\
Поэтому $ A_{x: \overline{n}| }^{1} = \sum\limits_{k=0}^{n-1}  (1+y_{k+1})^{-k-1}  \frac {s(x+k)-s(x+k+1)}{s(x)}$\\

Аналогично, если бы у нас ставка дисконтирования не зависела от времени, то\\
$\ddot a_{x: \overline{n}| } = \sum\limits_{k=0}^{n-1} v^{k} P(T_x > k) =  \sum\limits_{k=0}^{n-1} v^{k} \frac {s(x+k)}{s(x)}$\\
Но у нас ставка зависит от времени, то есть для дисконтирования используется кривая доходности, поэтому вместо $v^{k} $ надо писать $(1+y_{k})^{-k}$\\
Поэтому $\ddot a_{x: \overline{n}| } = \sum\limits_{k=0}^{n-1} (1+y_{k})^{-k}  \frac {s(x+k)}{s(x)}$\\
Тогда $P= SA \frac{  A_{x: \overline{n}| }^{1} }{   \ddot  a_{x: \overline{n}| }  } 
= SA \frac{ \sum\limits_{k=0}^{n-1}  (1+y_{k+1})^{-k-1}  \frac {s(x+k)-s(x+k+1)}{s(x)}  }{  \sum\limits_{k=0}^{n-1} (1+y_{k})^{-k}  \frac {s(x+k)}{s(x)} }=
SA \frac{ \sum\limits_{k=0}^{n-1}  (1+y_{k+1})^{-k-1}  (s(x+k)-s(x+k+1)) }{  \sum\limits_{k=0}^{n-1} (1+y_{k})^{-k}  s(x+k) }$\\

Теперь надо найти $s(t)$ пользуясь тем, что по условию $T_x $ распределено по закону Мэкама с параметрами $A_x = 0.0001 , B_x = 0.1 , a_x = 0.075 $\\
Вспомним, что если  $T_x $ распределено по закону Мэкама с параметрами $A_x = 0.0001 , B_x = 0.1 , a_x = 0.075 $, 
то $T$  распределено по закону Мэкама с параметрами $A=A_x  , B = B_x e^{-ax} , a=a_x  $\\
Это значит, что интенсивность смертности (соответствующая T) имеет вид\\
 $\mu_y = A+Be^{ay}$,\\
то есть $s(y) = e^{-Ay - \frac{B}{a} (e^{ay}-1)} =  e^{-A_xy - \frac{B_x  } {a_x}   e^{-ax} (e^{a_x y}-1)} $\\
то есть $s(x+k) =  e^{-A_x(x+k) - \frac{B_x  } {a_x}   e^{-ax} (e^{a_x (x+k)}-1)} $\\
то есть $s(x+k+1) =  e^{-A_x(x+k+1) - \frac{B_x  } {a_x}   e^{-ax} (e^{a_x (x+k+1)}-1)} $\\
Подставляя теперь эти выражения для $s(x+k)$ и $s(x+k+1)$ в нашу сумму, получим, что\\
$P= SA \frac{ \sum\limits_{k=0}^{n-1}  (1+y_{k+1})^{-k-1}  (s(x+k)-s(x+k+1)) }{  \sum\limits_{k=0}^{n-1} (1+y_{k})^{-k}  s(x+k) } = 
SA\frac{ 0.16572816696170486 }{ 1.391038680930967 } = 11913.986953316755 \approx \bf \$ 11913.99$\\
 {\bf \large  Ответ:} $\boxed {P \approx \bf \$ 11913.99}$\\



 
 
{\bf \large 6.} Договор смешанного страхования жизни на 10 лет гарантирует выплату страховой суммы $\pounds 100000$ в случае смерти застрахованного до истечения срока действия договора и выплату $\pounds50000$, если застрахованный проживёт эти 10 лет. Подсчитайте среднее значение современной стоимости обязательств страховщика по этому договору и стандартное отклонение от среднего. Техническая основа расчётов: постоянная интенсивность смертности $\mu=0.03$ на протяжении всего срока действия договора, годовая процентная ставка, используемая для дисконтирования, равна $5\%$.

{\bf \large Решение:}\\
 {\bf \large  (i)} Пусть $T_x$ - остаточное время жизни застрахованного, n=10 - срок действия договора, $S1=\pounds 100000$, $S2=\pounds50000$.\\
Тогда современная стоимость обязательств страховщика равна \\
$Z=Z_1 S_1+Z_2 S_2 = S_1  v^{T_x}  I\{ T_x < n\} +   S_2 v^{n}  I\{ T_x \geq n\} $\\

Чтобы посчитать $EZ$ и $VarZ$, нам нужна плотность остаточного времени жизни:\\
$f_x(t) = \frac{f(x+t)}{1-F(x)} = \frac {f(x+t)}{s(x)}$\\

А функции $f(x+t)$ и $s(x)$ мы найдем, зная постоянную интенсивность смертности $\mu=0.03$:\\

$\mu_x = \frac {f(x)}{1-F(x)}= - \frac{s'(x) } {s(x)}$\\

$\Rightarrow s(x) = e^{-\int\limits_{0}^{x} \mu_u du} = e^{-0.03x}$\\

$\Rightarrow f(x) = -s'(x) = 0.03e^{-0.03x}$\\

$\Rightarrow  f_x(t) = \frac{f(x+t)}{1-F(x)} = \frac {f(x+t)}{s(x)} =   \frac {   0.03e^{-0.03(x+t) } }{  e^{-0.03x} } =  0.03e^{-0.03t}$\\

Поэтому:\\

$EZ_1 =   E( v^{T_x}  I\{ T_x < n\} )= \int\limits_{0}^{n} v^t f_x(t) =  \int\limits_{0}^{n} e^{-\delta t } f_x(t) = \\
= \int\limits_{0}^{n} e^{-\delta t } \mu e^{-\mu t}dt = \mu  \int\limits_{0}^{n} e^{-t(\delta +\mu)}dt 
  = -\frac {\mu}{\mu + \delta} e^{-t(\delta +\mu)} |_{0}^{n}=   \frac {\mu}{\mu + \delta} (1- e^{-n(\delta +\mu)} ) $\\
  
 Аналогично:\\
  
$EZ_2 =    E( v^{n}  I\{ T_x \geq  n\})  =  v^{n}  P( T_x \geq n) = v^{n} \frac {s(x+n)}{s(x)} 
=  v^{n} \frac {  e^{-\mu(x+n)} }{ e^{-\mu x} }  = v^{n} e^{-\mu n }= e^{-\delta n }  e^{-\mu n } =  e^{-n( \mu + \delta) } $\\

$\Rightarrow EZ=S_1EZ_1 + S_2EZ_2 = S_1\frac {\mu}{\mu + \delta} (1- e^{-n( \mu + \delta) } )  + S_2 e^{-n(\mu + \delta) } =\\
S_1 \frac {0.03}{0.03+ ln(1.05) } (1- e^{-10(  0.03+ ln(1.05) )} ) + S_2 e^{-10(  0.03+ ln(1.05) ) } =\\
 S_1 \frac {0.03}{0.07879016416943205 } (1- e^{-0.7879016416943205} ) + S_2 e^{-0.7879016416943205 }=\\
  S_1* 0.38075818620567103 *  0.5452018758590105 + S_2 * 0.45479812414098947 = 
  S_1 *0.20759007736800625 + S_2 * 0.45479812414098947 =  20759.007736800625 + 22739.906207049473=  43498.9139438501 \approx   { \bf \pounds 43498.91 }$\\
  
  {\bf \large  (ii)}  Для квадрата обязательств страховщика мы имеем\\
  
 $Z^2=(S_1Z_1 + S_2Z_2)^2 = (S_1Z_1)^2 + (S_2Z_2)^2 + 2Z_1Z_2S_1S_2$, \\
  но последнее слагаемое равно нулю, поскольку $ ( I\{ T_x < n\} ) * (I\{ T_x \geq  n\}) = 0$\\
  
  Поэтому $EZ^2 = (S_1)^2E(Z_1)^2  + (S_2)^2E(Z_2)^2 $\\
  
  $E(Z_1)^2=   E( v^{2T_x}  I\{ T_x < n\} )= \int\limits_{0}^{n} v^{2t} f_x(t) =  \int\limits_{0}^{n} e^{-2\delta t } f_x(t) = \\
= \int\limits_{0}^{n} e^{-2\delta t )} \mu e^{-\mu t}dt = \mu  \int\limits_{0}^{n} e^{-t(2\delta +\mu)}dt 
  = -\frac {\mu}{\mu + 2\delta} e^{-t(2\delta +\mu)} |_{0}^{n}=   \frac {\mu}{\mu + 2\delta} (1- e^{-n(2\delta +\mu)} ) $\\
  
  Аналогично:\\
  
$E(Z_2)^2 =    E( v^{2n}  I\{ T_x \geq  n\})  =  v^{2n}  P( T_x \geq n) = v^{2n} \frac {s(x+n)}{s(x)} 
=  v^{2n} \frac {  e^{-\mu(x+n)} }{ e^{-\mu x} }  = v^{2n} e^{-\mu n }= e^{-2\delta n }  e^{-\mu n } =  e^{-n( \mu + 2\delta) } $\\

$\Rightarrow EZ^2=(S_1)^2E(Z_1)^2 + (S_2)^2E(Z_2)^2 = (S_1)^2\frac {\mu}{\mu + 2\delta} (1- e^{-n( \mu + 2\delta) } )  + (S_2)^2 e^{-n(\mu + 2\delta) } =\\
(S_1)^2 \frac {0.03}{0.03+ 2ln(1.05) } (1- e^{-10(  0.03+ 2ln(1.05) )} ) + (S_2)^2 e^{-10(  0.03+ 2ln(1.05) ) } =\\
 (S_1)^2 \frac {0.03}{0.1275803283388641 } (1- e^{-1.275803283388641} ) + (S_2)^2 e^{-1.275803283388641 }=\\
  (S_1)^2* 0.23514596952844855 *  0.7207934039043711 + (S_2)^2 * 0.27920659609562887 = 
  (S_1)^2 *0.16949166379080394+ S_2 * 0.27920659609562887 =  1694916637.9080393 + 698016490.2390722=  2392933128.1471114$\\
  
  $\Rightarrow VarZ = EZ^2 - (EZ)^2 = 2392933128.1471114- (43498.91)^2 =  2392933128.1471114 - 1892155171.1881003= 500777956.9590111$\\
  
  $\Rightarrow  \sqrt{VarZ} = 22378.06866016393 \approx \bf \pounds 22378.07$\\
  {\bf \large  Ответ:} $\boxed { EZ \approx\bf \pounds 43498.91;  \sqrt{VarZ} \approx\bf \pounds  22378.07}$\\
  
  
  
 {\bf \large 7.} Порядок формирования страховых резервов по страхованию жизни, утвержденный приказом Министерства финансов Российской Федерации от 09.04.2009 No 32н «Об утверждении Порядка формирования страховых резервов по страхованию жизни», предусматривает, что при расчете математического резерва допускается применение (при любых обстоятельствах):\\
а) ретроспективного метода;\\
б) перспективного метода;\\
в) как перспективного, так и ретроспективного методов.\\
г) метода, используемого для расчета выкупных сумм, выплачиваемых страхователю при расторжении договора страхования по виду страхования.\\

{\bf \large Ответ:}\\ б) перспективный метод\\
Это написано в пункте 1.3 этого закона:\\
 В соответствии с настоящим Порядком страховые резервы по страхованию жизни формируются при условии, если методы расчета страховых тарифов основаны на уравнении эквивалентности (равенстве актуарной стоимости страховых выплат по ПРЕДСТОЯЩИМ  страховым случаям и актуарной стоимости страховых премий (поступлений) на начало срока действия договора страхования), с применением таблиц смертности, заболеваемости, инвалидности, начислении в период действия договора страхования нормы (ставки) доходности. Актуарная стоимость страховых выплат (страховых поступлений) вычисляется путем дисконтирования ПРЕДПОЛАГАЕМЫХ денежных выплат (поступлений) с учетом вероятностей, с которыми такие денежные выплаты (поступления) БУДУТ произведены.\\


  
  
 {\bf \large 8.} Мужчина в возрасте $x = 55$ лет заключил 3-х летний договор страхования жизни. Если застрахованный умирает на протяжении действия договора, то страховая сумма $SA = \pounds 150,000$ выплачивается в очередную годовщину заключения договора; если же застрахованный доживает до окончания договора, то страховщик не платит ничего. Премия в размере $P = \pounds 900$ платится в начале каждого года действия договора. Заключение и поддержание договора требуют следующих расходов: начальные расходы $\pounds260$ в момент заключения договора, периодические расходы в размере $\pounds 70$ в начале второго и третьего года (если договор всё ещё действует).\\
Предполагая, что смертность описывается таблицей AM92 (её фрагмент приведён ниже), а для дисконтирования используется техническая процентная ставка
$i = 3\%$ , вычислите ожидаемый доход страховщика при заключении договора.\\
{\bf \large Решение:}\\
Напомним, что перед заключением договора человек проходит медицинское осведетельствование (отбор),
 поэтому его вероятность умереть в первый год обозначается $q_{[x]}$, а не просто $q_x$.\\
 Как обычно, $v=\frac{1}{1+i} = \frac{1}{1.03} = 0.970873786407767$\\
 
Расходы страховщика в размере \pounds260 в момент $t=0$, \pounds70 в момент $t=1$ и \pounds70 в момент $t=2$ можно интерпретировать как то, 
что премии от страхователя будут равны не \pounds900, а \pounds(900-70)= 830, и кроме того, 
в момент $t=0$ страховщик имеет дополнительные расходы в размере \pounds 190;\\

Поэтому доходы  $S_1 = 830\ddot a_{x: \overline{n}| }= 830\sum\limits_{k=0}^{n-1} v^{k} P(T_x > k) = 
830(1+v p_{[55]} + v^2 p_{[55]} p_{[55]+1} ) = 830(1+v (1-q_{[55]}) + v^2 (1-q_{[55]})(1- q_{[55]+1}) )=
830(1+v(1-0.003358) + v^2(1-0.003358)(1-0.004903) ) = 830(1+0.996642v + 0.996642\cdot 0.995097v^2) = 
830(1+0.996642v + 0.9917554642740001v^2) = 830(1 + 0.996642 \cdot 0.970873786407767 + 0.9917554642740001 \cdot 0.970873786407767 \cdot 0.970873786407767)=
830(1+0.9676135922330098 + 0.9348246434857197) = 830 \cdot 2.9024382357187295= 2409.0237356465454$\\

Расходы $S_2= 190+SA \cdot A_{x: \overline{n}| }^{1} = 190+SA \cdot \sum\limits_{k=0}^{n-1} v^{k+1} P(K_x = k)=
190+SA \cdot \sum\limits_{k=0}^{n-1} v^{k+1} P(k \leq  T_x < k+1)  = 190+SA(v P(T_x <1) + v^2 P(1 \leq T_x < 2) +  v^3 P(2 \leq T_x < 3)=
190+SA(v q_{[x]}  + v^2 p_{[x]}q_{[x]+1}  +  v^3 p_{[x]}p_{[x]+1} q_{x+2}  =  190+SA(v \cdot 0.003358  + v^2  (1-0.003358) 0.004903  +  v^3 (1-0.003358)(1-0.004903)0.005650)=
190+ SA(0.003358v + 0.004886535726000001v^2 +0.0056034183731481v^3)= 
 190+ SA(0.003358 \cdot  0.970873786407767 + 0.004886535726000001 \cdot  0.970873786407767 \cdot  0.970873786407767  +
 0.0056034183731481\cdot  0.970873786407767\cdot  0.970873786407767\cdot  0.970873786407767)
 = 190 + SA(0.0032601941747572817 + 0.004606028585163541 + 0.005127921588052734) =
 190 + SA \cdot 0.012994144347973557= 190+ 1949.1216521960337= 2139.1216521960337$\\
 
 Поэтому прибыль страховщика = доходы-расходы = $S_1 - S_2 = 2409.0237356465454- 2139.1216521960337 = 269.9020834505118 \approx\bf 269.90$\\
 {\bf \large  Ответ:} $\boxed { profit \approx\bf \pounds 269.90}$\\
 
 
{\bf \large 9.} Пусть при $ t \in [x,x + n]$  интенсивность смертности $\mu_t$ можно представить в виде
$\mu_t = \frac{t+b}{a}$, где $a>0$ и $b$ - некоторые константы (т.е. на этом промежутке функция $\mu_t$ линейна и возрастает).\\
Докажите, что $$ \overline a_{x: \overline{n}| } = \sqrt{2\pi a} \cdot  exp{   \{ \frac  {A^2}{2}  \} } (\Phi (A + \frac{n}{\sqrt a }) - \Phi(A)),$$
где $A=\frac{x+b+a\delta}{\sqrt{a}}$, а $\Phi$ - стандартная гауссовская функция распределнения. \\

{\bf \large Решение:}\\

$ \overline a_{x: \overline{n}| } = \int\limits_{0}^{n} v^{t} P(T_x > t) dt = \int\limits_{0}^{n} v^{t} \frac{s(x+t)}{s(x)}dt =   \int\limits_{0}^{n} e^{-\delta t} \frac{s(x+t)}{s(x)}dt $\\

Найдем $s(x)$, зная вид интенсивности смертности: \\
$s(x) = e^{-\int\limits_{0}^{x} \mu_t dt}$\\

$ \int\limits_{0}^{x} \mu_t dt = \int\limits_{0}^{x} \frac{t+b}{a} dt = \frac {(t+b)^2}{2a} |_{0}^{x} = \frac{1}{2a} ((b+x)^2 - b^2) = \frac{x^2 + 2bx}{2a}$\\

$\Rightarrow  s(x) = e^{-\int\limits_{0}^{x} \mu_t dt} = e^{- \frac{x^2 + 2bx}{2a}} $\\

$s(x+t) = e^{- \frac{(x+t)^2 + 2b(x+t)}{2a}}  = e^{- \frac{x^2 +2xt + t^2 +2bx + 2bt }{2a}} $\\

$\Rightarrow  \frac{s(x+t)}{s(x)} = \frac{ e^{- \frac{x^2 +2xt + t^2 +2bx + 2bt }{2a}} }{ e^{- \frac{x^2 + 2bx}{2a}}  } =  e^{- \frac{ t^2 +2t(x+b)  }{2a}}$\\


$\Rightarrow  {\bf  \overline a_{x: \overline{n}| }  } = \int\limits_{0}^{n} e^{-\delta t} \frac{s(x+t)}{s(x)}dt = 
\int\limits_{0}^{n} e^{-\delta t}  e^{- \frac{ t^2 +2t(x+b)  }{2a}} dt=
 \int\limits_{0}^{n}   e^{- \frac{ t^2 +2t(x+b+a\delta)  }{2a}} dt=
 \int\limits_{0}^{n}   e^{- \frac{ (t + x+b+a\delta)^2 }{2a}}  e^{ \frac{ (x+b+a\delta)^2  }{2a}}  dt =
 e^{\frac{A^2}{2} }  \int\limits_{0}^{n}   e^{- \frac{ (t + x+b+a\delta)^2 }{2a}}    dt=\\
 \{  x=\frac{ t + x+b+a\delta }{\sqrt{a}} ; dx = \frac{dt}{\sqrt{a}} ; dt = \sqrt{a}dx\}=
 \sqrt{a} e^{\frac{A^2}{2} }  \int\limits_{A}^{A + \frac{n}{ \sqrt{a} }}   e^{- \frac{ x^2}{2}}    dx=
  \sqrt{2 \pi a} e^{\frac{A^2}{2} }   \frac{1}{ \sqrt{2\pi} }\int\limits_{A}^{A + \frac{n}{ \sqrt{a} }}   e^{- \frac{ x^2}{2}}    dx =
 { \bf \sqrt{2 \pi a} e^{\frac{A^2}{2} }  (\Phi (A + \frac{n}{\sqrt a }) - \Phi(A)) }$\\
 
 
{\bf \large 10.} (i) Определите термин «проспективный резерв» (prospective reserve) применительно к договору страхования жизни.\\
(ii) Сформулируйте условия, которые достаточны для того, чтобы проспективный резерв был равен ретроспективному резерву.\\
Страховая компания заключает договор пожизненного страхования с человеком, возраст которого ровно x лет (число x – натуральное). Премии платятся в начале каждого года на протяжении всего срока действия договора, а страховая сумма S выплачивается немедленно после смерти застрахованного. Никаких расходов в связи с договором нет.\\
(iii) Покажите, что при выполнении условий, упомянутых в пункте (ii), в любой момент времени t (число t – натуральное) проспективный резерв равен ретроспективному резерву.\\

{\bf \large Решение:}\\

 {\bf \large  (i)} Проспективный резерв –это   разность между  современной стоимостью предстоящих расходов (включая выплаты по договору страхования жизни) 
  (т.е. обязательств страховщика)  и   современной стоимостью предстоящих страховых премий по договору страхования жизни (т.е. обязательств страхователя).\\
  
  {\bf \large  (ii)} Условия, которые достаточны для того, чтобы проспективный резерв был равен ретроспективному резерву:\\
  1) Премии определяются исходя из принципа эквивалентности обязательств\\
  2) При расчете проспективного резерва, ретроспективного  резерва и премии использовался один и тот же базис.\\
 Действительно, если  премия по некоторому виду страхования или пенсионной схеме определена из принципа эквивалентности, то в среднем компания не должна 
 привлекать собственные средства для выполнения финансовых   обязательств перед клиентами. Это означает, что резерв в момент t, необходимый для выполнения 
 будущих финансовых обязательств по каждому еще действующему договору, должен быть равен сумме, накопленной к моменту t на каждый действующий договор.\\
 
 {\bf \large  (iii)} Допустим, что в конце $k$-го года (т.е. в момент $t = k$) договор все еще сохраняет силу (так что застрахованный еще жив и его возраст равен $x+k$).\\
 
При подсчете перспективным методом:\\
$_{k}V = _{k}a_{B} - _{k}a_{C}$\\

$ \bullet _{k}a_{B} = \sum\limits_{j=k+1}^{n}   Sv^{j-k}P(j-k-1 < T_{x+k} < j-k)  = \\
\sum\limits_{j=k+1}^{n}   Sv^{j-k}P(T_{x+k} > j-k-1) q_{x+j-1} =  \sum\limits_{j=k+1}^{n}   Sv^{j-k}  \frac {P(T_{x} >  j-1)}  { P(T_{x} > k)} q_{x+j-1}$\\

$ \bullet _{k}a_{c} = \sum\limits_{j=k+1}^{n}   P_{j} v^{j-k-1}P( T_{x+k} >  j-k-1)  =  \sum\limits_{j=k+1}^{n}   P_{j} v^{j-k-1}  \frac {P(T_{x} >  j-1)}  { P(T_{x} > k)} $\\

$\Rightarrow _{k}V = _{k}a_{B} - _{k}a_{C} = \frac{1}{v^{k}P(T_{x} >k) } \sum\limits_{j=k+1}^{n} v^{j}P(T_{x} > j-1) (Sq_{x+j-1} - P_{j}(1+i))$\\

Преобразуем последнюю сумму с учетом того, что \\
$ \sum\limits_{j=1}^{n} v^{j}P(T_{x} > j-1) (Sq_{x+j-1} - P_{j}(1+i)) = 0$ 
в силу принципа эквивалентности  обязательств, согласно которому и вычислялись премии.\\

$ _{k}V = \frac{1}{v^{k}P(T_{x} >k) } \sum\limits_{j=k+1}^{n} v^{j}P(T_{x} > j-1) (Sq_{x+j-1} - P_{j}(1+i))=\\
\frac{1}{v^{k}P(T_{x} >k) }  (\sum\limits_{j=1}^{n} v^{j}P(T_{x} > j-1) (Sq_{x+j-1} - P_{j}(1+i)) - \sum\limits_{j=1}^{k} v^{j}P(T_{x} > j-1) (Sq_{x+j-1} - P_{j}(1+i)))=
- \frac{1}{v^{k}P(T_{x} >k) }  \sum\limits_{j=1}^{k} v^{j}P(T_{x} > j-1) (Sq_{x+j-1} - P_{j}(1+i)) =  \\
\frac{1}{v^{k}P(T_{x} >k) }  \sum\limits_{j=1}^{k} v^{j}P(T_{x} > j-1) ( P_{j}(1+i)  - Sq_{x+j-1}  ) $\\

Теперь заметим, что последняя формула в точности равна ретроспективному резерву, потому что там написана разность между  текущей стоимостью 
накопления к моменту $k$ за счет премий и  текущей стоимостью накопления   к моменту $k$ от всех выплат на промежутке $(0,k)$.\\
Поэтому мы доказали, что при выполнении условий, упомянутых в пункте (ii), в любой момент времени t (число t – натуральное) проспективный резерв равен ретроспективному резерву.\\





 





 
 
\end{document}


