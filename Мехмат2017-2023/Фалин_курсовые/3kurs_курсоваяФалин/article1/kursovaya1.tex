
\documentclass[12pt,a4paper]{article}
\usepackage[utf8]{inputenc}
\usepackage[english,russian]{babel}
\usepackage{indentfirst}
\usepackage{misccorr}
\usepackage{graphicx}
\usepackage{amsmath}
\begin{document}




\begin{titlepage}

 \begin{center}
 Московский Государственный Университет имени М.В.Ломоносова \\
Механико-математический факультет\\
Кафедра теории вероятностей
  \end{center}

 \vspace{3cm}
 
 \begin{center}
   
  {  Курсовая работа за 3 курс:\\
  Принципы назначения страховых премий.}
   
    \vspace{5cm}
\end{center}     
   
     
   \hspace{170pt}  {Выполнила: Александра Токаева,  309 \\}
       
 \vspace{0.1cm}
  \hspace{170pt} 	  Научный руководитель:  Г.И.Фалин\\

\vspace{4cm}

  \begin{center}
  {Москва\\
  2020}
  \end{center}  
 
 
\end{titlepage}

{\bf \largeВведение}\\
Нестрого говоря, {\bfпринцип назначения страховой премии} - это правило, назначающее премию конкретному страховому риску. В данной статье мы сконцентрируемся на премиях, отвечающих денежным выплатам страховщика (только)  в связи со страховыми убытками, а также надбавкой за риск, который берет на себя страховщик (эта надбавка  отражает  тот факт, что реальные убытки почти никогда не равны предполагаемым). Другими словами, мы не берем в расчет надбавки, отвечающие  за расходы и получение прибыли страховщиком.\\
В данной статье мы описываем три метода, которые используются актуариями для расчета премий. Различие между этими методами весьма условное, и конкретный принцип назначения премий может возникать из более, чем одного метода.\\
Мы называем первый метод   {\bf \underline {Специальным методом}} , потому что согласно ему актуарий назначает (потенциально) разумный принцип назначения премий, а потом определяет каким (если вообще хоть каким-то) из желаемых свойств этот принцип удовлетворяет. Например, {\bf the Expected Value Premium Principle} (согласно которому премия равняется ожидаемому значению, умноженному на какое-то число, большее или равное единице) - это очень специальный метод, но он обладает некоторыми хорошими свойствами. В самом деле, он аддитивен и сохраняет аффинные преобразования случайных величин. Однако он не обладает  всеми  желаемыми свойствами, как мы увидим в разделе " Каталог принципов назначения премий: специальный метод".\\
Более строгим методом является так называемый {\bf \underline {Характеристический метод}}, потому что согласно ему актуарий составляет список свойств, которые он хочет от принципа , а потом находит принцип (или семейство принципов), определяемый этими свойствами. Иногда актуарий может не  характеризовать все множество принципов, удовлетворяющее этому списку свойств,  а только найти один такой принцип. Нахождение только одного такого принципа является более слабым методом, чем полный характеристический метод, но на практике, его часто достаточно. Примером более слабого метода поиска принципа назначения премий является метод поиска принципа, который инвариантен относительно масштабирования, и мы увидим в разделе  " Каталог принципов назначения премий: специальный метод" , что {\bf принцип стандартных отклонений} является инвариантным относительно масштабирования, но не является единственным принципом, удовлетворяющим этому свойству. \\
Возможно, наиболее строгим принципом назначения премий является  {\bf \underline {Экономический метод}.} Согласно этому методу, актуарий принимает некоторую экономическую теорию и затем определяет вытекающий  из нее принцип. В разделе "Экономический метод"  мы увидим, что важный {\bf Принцип Эшера} относится к этому классу.\\
Отметим, что эти методы не являются взаимоисключающими. Например, {\bf принцип пропорционального деления риска}
сначала появился, когда Ванг искал принцип, удовлетворяющий условию аддитивности слоев; тем самым, Ванг использовал слабую форму характеристического метода. Потом Ванг, Янг и Панджер показали, что принцип пропорционального деления риска может быть получен посредством перечисления свойств и доказательства, что это единственный принцип с такими свойствами. Наконец, принцип пропорционального деления риска может быть обоснован и ходя из Дуальной теории Яри (см. Risk Utility Ranking).\\
Некоторые принципы появляются из более чем одного метода. Например, актуарий может хотеть от принципа конкретное свойство (или список свойств); он находит такой принцип и позднее узнает, что этот принцип может быть обоснован экономическим методом. В разделах "Характеристический метод" и "Экономический метод" мы увидим, что принцип Ванга следует одновременно их характеристического и экономического методов.\\
В разделе "Свойства принципов назначения страховых премий" мы приведем список желаемых свойств принципов для дальнейших ссылок. В разделе "Каталог принципов: специальный метод" мы определяем некоторые принципы и указываем некоторые их свойства. Таким образом, этот раздел написан в духе специального метода. В разделе "Характеристический метод" мы демонстрируем этот метод посредством составления списка свойств и дальнейшего указания принципов, удовлетворяющих  этим свойствам. Что касается экономического метода, мы описываем некоторые экономические теории, принятые актуариями,  и определяем вытекающие из них принципы. В разделе "Саммари" мы даем заключение  этой статье.


















\end{document}
