
\documentclass[12pt,a4paper]{article}
\usepackage[utf8]{inputenc}
\usepackage[english,russian]{babel}
\usepackage{indentfirst}
\usepackage{misccorr}
\usepackage{graphicx}
\usepackage{amsmath}
\begin{document}




\begin{titlepage}

 \begin{center}
 Московский Государственный Университет имени М.В.Ломоносова \\
Механико-математический факультет\\
Кафедра теории вероятностей
  \end{center}

 \vspace{3cm}
 
 \begin{center}
   
  {  Курсовая работа за 3 курс:\\
  Принципы назначения страховых премий и меры риска.}
   
    \vspace{5cm}
\end{center}     
   
     
   \hspace{170pt}  {Выполнила: Александра Токаева,  309 \\}
       
 \vspace{0.1cm}
  \hspace{170pt} 	  Научный руководитель:  проф. Г.И.Фалин\\

\vspace{4cm}

  \begin{center}
  {Москва\\
  2020}
  \end{center}  
  
\newpage
\tableofcontents
 
 
\end{titlepage}

{\center Мы применим простые геометрические принципы, чтобы показать, что хорошо известные подходы к назначению премий 
в страховом контракте минимизируют взвешенные квадраты разностей между индивидуальными премиями и индивидуальными выплатами, а также между суммарными премиями для классов однородных рисков и суммарными выплатами, поступающими из них. }

{   \section{ Введение}} 
Рассмотрим портфель из n неоднородных независимых страховых рисков. Пусть $X_i$ обозначает размер выплат по i-му риску за рассматриваемый период, S - суммарные потери, связанные с портфелем. При некоторых естественных предположениях (что портфель достаточно большой, не очень неоднородный и распределение размера выплат не очень ассиметричное) распределение случайной величины $\frac{S-ES}{\sqrt{VarS}}$ может быть приближено стандартным гауссовским распределением.\\
Предположим, что страховщик взимает премию $\pi_i$ по i-му риску и таким образом собирает суммарную премию
$\pi=\sum \limits_{i=1}^{n}\pi_i$. Из гаусовости  распределения  величины $\frac{S-ES}{\sqrt{VarS}}$ получаем, что для гарантии достаточно маленькой вероятности разорения $R=P(S>\pi)$ (например, R=5\%) страховщик должен собрать суммарную премию в размере 
$$ES + \sqrt{VarS}*z_{(1-R)} \eqno(1) $$, где $z_{\alpha} $
 - квантиль гаусовского распределения уровня альфа.

Ну действительно: $P(S>\pi)=R \Leftrightarrow P(\frac{S-ES}{\sqrt{VarS}} > \frac{\pi-ES}{\sqrt{VarS}})=R \Leftrightarrow \frac{S-ES}{\sqrt{VarS}}=z_{(1-R)} \Leftrightarrow \pi=ES + \sqrt{VarS}*z_{(1-R)}.$\\
Последнее равенство ничего не говорит о величине индивидуальных премий. Чтобы найти их, мы используем дополнительные принципы. \\
Вслед за Заксом, Фростигом и Левиксоном мы рассмотрим два подхода к задаче разбиения величины $\pi$ на n индивидуальных премий $\pi_1 \ldots \pi_n:$\\

1) Для заданной вероятности разорения $R=P(S>\pi)$ минимизировать взвешенного квадрата разности $\sum \limits_{i=1}^{n} \frac {1}{s_i} E(X_i-\pi_i)^2$ между индивидуальными рисками $X_i$ и индивидуальными премиями $\pi_i$ (где $s_i$ -это некоторые известные положительные числа, то есть веса)\\

2) Для заданной $D=\sum \limits_{i=1}^{n} \frac {1}{s_i} E(X_i-\pi_i)^2$  минимизировать вероятность разорения $P(S>\pi)$\\



Сейчас мы с помощью простых геометрических рассуждений покажем, что обе задачи минимизации имеют одно и то же решение. Кроме того, мы покажем, что оптимальные премиии $\pi_i$ минимизируют взвешенную сумму  квадратов разностей между индивидуальными премиями и индивидуальными выплатами, а также между суммарными премиями для классов однородных рисков и суммарными выплатами, поступающими из них. \\
Наше достижение опирается на недавнюю статью, написанную Заксом, Фростигом и Левиксоном, которые исследовали похожие задачи оптимальных цен на  неоднородный портфель( который может быть разделен на классы однородных рисков) с помощью алгебраических методов, основанных на теоремах о положительно определенных матрицах.\\


{\section { Общие результаты о случайных величинах}}
{\subsection{  Проблема оптимизации}}

Пусть $\xi_1, \ldots ,\xi_N$ - случайные величины с конечными матожиданиями $a_1 \cdots a_N$ и дисперсиями $Var\xi_1 \ldots Var\xi_N$. Мы предполагаем, что матожидание и дисперсии известны.\\
Нам бы хотелось заменить случайные величины $\xi_1, \ldots , \xi_N$ на неслучайные числа $A_1, \ldots, A_n$ таким образом, чтобы взвешенная сумма 
$$D \equiv \sum \limits_{i=1}^{n} \omega_i E(\xi_i-A_i)^2 \eqno(2)$$
 была бы минимальна.
Здесь $\omega_i, \ldots, \omega_N$ - это известные числа(веса). \\
Используя элементарные свойства случайных величин, мы можем переписать D следующим образом:
$$D= \sum \limits_{i=1}^{n} \omega_i E(\xi_i-A_i)^2 = \sum \limits_{i=1}^{n} \omega_i( Var(\xi_i-A_i) + (E(\xi_i-A_i))^2)$$ = $$ \sum \limits_{i=1}^{n} \omega_i( Var(\xi_i-A_i) + (a_i-A_i)^2)= \sum \limits_{i=1}^{n} \omega_i Var\xi_i + \sum \limits_{i=1}^{n} \omega_i( a_i-A_i)^2 \eqno(3)$$\\

Поскольку $\omega_i$ и $Var\xi_i$ фиксированы, то изначальная задача минимизации превращается в задачу нахождения минимального значения функции
$$f(A_1, \ldots, A_N) = \sum \limits_{i=1}^{n} \omega_i( a_i-A_i)^2 \eqno(4)$$
Очевидно, оптимальным значением являются $$A_1^*=a_1, \ldots, A_N^*=a_N$$
и минимальное значение этой функции равно нулю. Соответственно, минимальное значение величины D равно $\sum \limits_{i=1}^{n} \omega_i Var\xi_i $\\
Более интересной задача становится, если мы  накладываем  дополнительные ограничения на переменные $A_1, \ldots, A_N$. Принимая во внимания последующее приложение этой задачи к страхованию, мы рассматриваем следующую задачу:\\

{\bf Задача 1 }\\ Найти минимальное значение D при условии, что $$A_1 + \ldots+  A_N=C \eqno(5)$$, где C-известная константа.

Благодаря (3), достаточно найти минимальное значение функции(4) на множестве(5).\\

Для решения этой задачи введем новые переменные $x_i=\sqrt \omega_i (A_i-a_i)$, то есть 
$A_i=a_i + \frac{1} {\sqrt {\omega_i}} x_i$. Тогда задача 1 превращается в:\\

{\bf Задача 2 }\\Найти минимальное значение функции $$g(x_1, \ldots, x_N)= \sum \limits_{i=1}^{n} x_i^2 \eqno(6)$$ при условии, что 
$$\sum \limits_{i=1}^{n} \frac{1}{\sqrt \omega_i} x_i = C - \sum \limits_{i=1}^{n} a_i \eqno(7) $$.\\

Последовательности $X=(x_1, \ldots, x_N)$ и $Y=(\frac{1}{\sqrt {\omega_1} }, \ldots, \frac{1}{\sqrt {\omega_N}}$ можно понимать как N-мерные евклидов векторы в пространстве $R^N$. Соответственно, левая часть равенства(7) есть скалярное произведение X и Y, а функция $g(x_1, \ldots, x_N)$ есть $||X||^2$, где $$||X||= \sqrt{x_1^2+
\ldots+x_N^2}$$ - это длина вектора X.\\

Последующие рассуждения основаны на неравенстве Коши-Буняковского-Шварца, согласно которому для любых двух векторов $X,Y \in R^N$ верно 
$$|X \cdot Y| \leq ||X|| \cdot ||Y||$$ , причем равенство достигается тогда и только тогда, когда X и Y линейно зависимы(в частности, если вектор Y ненулевой, линейная зависимость означает, что X пропорционален Y: $ X=t \cdot Y$ для некоторого $t \in R$)\\

Применяя это неравенство, получаем:
$$g(x_1, \ldots, x_N)=||X||^2 \geq \frac{|X \cdot Y|^2}{||Y||^2} = \frac{(C-\sum\limits_{i=1}^{n} a_i)^2}
{\sum\limits_{i=1}^{n} \frac{1}{\omega_i} }  \eqno(7.5)$$

Поэтому для $(x_1, \ldots, x_N)$, удовлетворяющих (7), имеем:
$$ min g(x_1, \ldots, x_N) \geq \frac{(C-\sum\limits_{i=1}^{n} a_i)^2}{\sum\limits_{i=1}^{n} \frac{1}{\omega_i}} 
\eqno(8) $$

Поскольку вектор $Y=(\frac{1}{\sqrt{\omega_1}}, \ldots, \frac{1}{\sqrt{\omega_N}})$ ненулевой, то равенство в (8) достигается тогда и только тогда когда существует такое t что 
$$x_i = \frac{1}{\sqrt{\omega_i}} \cdot t , i=1, \ldots, N  \eqno(9)$$

Подставляя, что $X=t \cdot Y$ в (7.5), получаем, что 
$$ t^* = \frac{ C-\sum\limits_{i=1}^{n} a_i} {\sum\limits_{i=1}^{N} \frac{1}{\omega_i}}$$, 

$$ x_i^*=\frac{1}{\sqrt{\omega_i}} \cdot t^* $$

$$A_i^*=a_i + \frac{1}{\sqrt \omega_i x_i^*} = a_i + \frac{1}{\omega_i}t^*
 = a_i + \frac{1}{\omega_i} \frac{C-\sum\limits_{j=1}^{N} a_j}{\sum\limits_{j=1}^{N} \frac{1}{\omega_j} } \eqno(10)$$

$$D_{min}= \sum \limits_{i=1}^{N} \omega_i Var\xi_i  +  \frac{(C-\sum\limits_{j=1}^{N} a_j)^2}
{\sum\limits_{j=1}^{N} \frac{1}{\omega_j}} \eqno(11)$$



{\subsection { Двойственная проблема оптимизации}}
Такой же подход может быть применен для изучения двойственной задачи оптимизации:\\
{\bf Задача 3} Найти максимум суммы $A_1 + \ldots + A_N$ если  задано 
$$D= \sum \limits_{i=1}^{n} \omega_i E( \xi_i-A_i)^2 \eqno(12)$$


Заметим, что из (3) следует, что константа D должна быть больше или равна чем $\sum \limits_{i=1}^{n} \omega_i Var\xi_i$\\

Как и раньше, перепишем $$ \sum \limits_{i=1}^{n} \omega_i E( \xi_i-A_i)^2  = \sum \limits_{i=1}^{n} \omega_i Var\xi_i + \sum \limits_{i=1}^{n} \omega_i( a_i-A_i)^2 $$

Ограничение (12) превращается в равенство $$ D^{'} = \sum \limits_{i=1}^{n} \omega_i( a_i-A_i)^2 $$, где  
$$D^{'} = D - \sum \limits_{i=1}^{n} \omega_i Var\xi_i \geq 0 $$

Вводя $x_i=\sqrt \omega_i (A_i-a_i)$, мы сводим задачу 3 к следующему виду:\\


{\bf Задача 3} Найти максимум суммы $\sum \limits_{i=1}^{n} \frac{1}{\sqrt \omega_i} x_i$ если  задана
сумма $$D^{'} = \sum \limits_{i=1}^{N} x_i^2 \eqno(13)$$

Аналогично задаче 2, применяем неравенство Коши-Буняковского-Шварца:\\
$$\sum \limits_{i=1}^{N} \frac{1}{\sqrt \omega_i} x_i = X \cdot Y \leq ||X|| \cdot ||Y|| = \sqrt{D^{'}} \sqrt{\sum\limits_{i=1}^{N} \frac{1}{\omega_i^2}} \eqno(14)$$

Равенство в (14) достигается тогда и только тогда, когда существует t такое что 
$$x_i = \frac{1}{\sqrt{\omega_i}} \cdot t , i=1, \ldots, N  \eqno(15)$$

Подставляя выражение $X=t \cdot Y$ в (14), получаем единственное решение 
$$ t^* = \sqrt{ \frac{D^{'}}{\sum\limits_{i=1}^{N} \frac{1}{\omega_i^2}}}$$
$$x_i^* = \frac{1}{\sqrt{\omega_i}} \cdot t^* = \frac{1}{\sqrt{\omega_i}} \sqrt{ \frac{D^{'}}{\sum\limits_{i=1}^{N} \frac{1}{\omega_i^2}}}$$


Соответственно, такие $A_i$ задают решение оптимизационной задачи 3:
$$A_i^* = a_i + \frac{1}{\sqrt{\omega_i}} \cdot t^* = \frac{1}{\sqrt{\omega_i}} \sqrt{ \frac{D^{'}}{\sum\limits_{i=1}^{N} \frac{1}{\omega_i^2}}} \eqno(16)$$



{\section{ Приложение к модели индивидуального риска}}

В этом разделе мы применим полученные выше результаты к задаче оптимального назначения премий для неоднородного портфеля, рассмотренной Заксом, Фростигом и Левиксоном.\\

Рассмотрим модель индивидуального риска:\\
$$S= X_1 + \ldots + X_n,$$ где n - это общее число рисков в портфеле, случайная величина $X_i$ обозначает потери по i-my риску за рассматриваемый период, а S - это общие потери по портфелю.\\

Мы предполагаем, что случайные величины $X_1, \ldots, X_n$ независимы и имеют конечные матожидание $\mu_1, \ldots, \mu_n$ и дисперсии $\sigma_1^2, \ldots, \sigma_n^2$ соответственно. Тогда случайная величина S имеет конечные матожидание $\mu = \mu_1+ \ldots + \mu_n$ и дисперсию $\sigma^2 = \sigma_1^2+ \ldots + \sigma_n^2.$  Мы также предполагаем, что для достаточно больших n функция распределения центрированной и нормированной величины полных потерь $\frac {S-\mu}{\sigma}$ может быть приближена функцией распределения стандартной гауссовской величины 
$\Phi(x) =  \int \limits_{\infty}^{\infty}exp^{-\frac{t^2}{2}} dt$, то есть:

$$P(\frac{S-\mu}{\sigma} < x) \approx \Phi(x)$$


Предположим, что страховщик взимает премию $\pi_i$ по i-му риску, то есть всего собирает сумму $\pi=\sum \limits_{i=1}^{n}\pi_i.$ Тоогда вероятность разорения ( это вероятность того, что S будет больше собранной суммы $\pi$ дается формулой $ R=P(S>\pi).$\\
Используя гаусовость  $\frac{S-\mu}{\sigma}$, получаем, что:
$$R=P(\frac{S-\mu}{\sigma} > \frac{\pi-\mu}{\sigma} )\approx 1 - \Phi(\frac{\pi-\mu}{\sigma}). \eqno(17)$$

Предположим, что страховщик готов принять достаточно маленький риск разорения R (например, $R=1\%$). Тогда равенство (17) дает следующую(приближенную) формулу для суммарной премии:

$$\pi = \mu  + \sigma \cdot z_{(1-R)} \eqno(1) \eqno(18) $$, где $z_{\alpha} $
 - квантиль гаусовского распределения уровня альфа, то есть $\Phi(z_{\alpha}) = \alpha$.\\

Равенство (18) ничего не говорит про величины индивидуальных премий $\pi_i.$ Чтобы найти их, нам придется применить дополнительные принципы.

{\subsection { Минимизация разности между индивидуальными рисками и индивидуальными  премиями при заданной вероятности разорения }}

Рассмотрим взвешенную сумму $$ D= \sum \limits_{i=1}^{N} \frac{1}{s_i} E( X_i-\pi_i)^2$$
между индивидуальными рисками $X_1, \ldots, X_n$ и индивидуальными премиями $\pi_1, \ldots, \pi_n$ (где $s_1,   
\ldots, s_n$ - это некие известные положительные числа(веса)) и найдем минимум D:

$$ D \equiv D(\pi_1, \ldots, \pi_n) \rightarrow min. \eqno(19)$$

Применяя формулу(10) для $N=n, \xi_i=X_i, a_i = \mu_i, A_i=\pi_i, \omega_i=\frac{1}{s_i}, C= \mu + \sigma \cdot z_{(1-R)}$ мы можем утверждать, что минимизационная задача (19) с ограничением (18) имеет единственное решение 

$$ \pi_i^* = \mu_i +  \frac{s_i}{\sum\limits_{j=1}^{n} s_j } \cdot \sigma \cdot z_{(1-R)}. \eqno(20)$$


Теперь рассмотрим постановку задачи, рассмотренной Заксом, Фростигом и Левиксоном. Пусть портфель можно разбить на к классов однородных рисков с одинаковыми статистическими характеристиками потерь (обычно риски из одного класса принадлежат одному и тому же сектору бизнеса). Пусть i-й класс состоит из $n_i$ рисков с одинаковым средним $\mu_i$
и одинаковыми дисперсиями $\sigma_i^2.$ Тогда общее количество потерь $S_i$  в i-м классе  имеет среднее значение $ES_i= n_i \mu_i$ и дисперсию $ Var S_i = n_i \sigma_i^2.$ Общее число потерь от всего портфеля есть 
$S= S_1 + \ldots + S_k$, причем $\mu \equiv ES= \sum\limits_{i=1}^{k} n_i \mu_i $ , 
$\sigma^2 \equiv VarS= \sum\limits_{i=1}^{k} n_i \sigma_i^2 $


Благодаря однородности рисков внутри отдельного класса i, страховщик должен взимать со всех рисков в этом классе одну и ту же премию $\pi_i.$ Тогда общая премия за все риски в портфеле равна $\pi = \sum\limits_{i=1}^{k} n_i \pi_i.$\\

Рассмотрим $$  D= \sum \limits_{i=1}^{k} \frac{1}{r_i} E( S_i-n_i \pi_i)^2$$ между суммарными потерями по разным секторам бизнеса $S_1, \ldots, S_k$ и суммарными премиями $n_1 \pi_1, \ldots, n_k \pi_k$ от этих классов (где $r_1, \ldots, r_k$ - это некоторые известные положительные числа) и минимизируем D:

$$D = D \equiv D(\pi_1, \ldots, \pi_k) \rightarrow min. \eqno(21)$$


Для того, чтобы получить предписанную вероятность разорения, нужно, чтобы (18)  выполнялось.\\

Применяя формулу(10) для $N=k, \xi_i=S_i, a_i = n_i \mu_i, A_i=n_i \pi_i, \omega_i=\frac{1}{r_i}, C= \mu + \sigma \cdot z_{(1-R)}$ мы можем утверждать, что минимизационная задача (19) с ограничением (18) имеет единственное решение 

$$ n_i \pi_i^* = n_i \mu_i +  \frac{r_i}{\sum\limits_{j=1}^{k} r_j } \cdot \sigma \cdot z_{(1-R)}
\Leftrightarrow  \pi_i^* = \mu_i +  \frac{r_i}{n_i \sum\limits_{j=1}^{k} r_j } \cdot \sigma \cdot z_{(1-R)}. \eqno(22)$$

Теперь вернемся к минимизационной задаче (19) c ограничением (18) и положим для всех рисков из i-го класса одинаковое значение параметра s равным $\frac {r_i}{n_i}.$ Тогда  из (20) видно, что оптимальное решение для минимизационной задачи (19) совпадает с оптимальным решением минимизационной задачи (21). Таким образом, одни и те же значения премий минимизируют взвешенную сумму квадратов разностей как между индивидуальными премиями и индивидуальными потерями, так и между суммарными премиями для классов однородных рисков и суммарными потерями для этих блоков.


{\subsection {  Минимизация вероятности разорения при заданной разности между индивидуальными рисками и индивидуальными  премиями }}


{\bf Задача 5} Для модели индивидуального риска $$S= X_1 + \ldots + X_n$$ минимизировать вероятность разорения $R=P(S > \pi)$ при заданной величине $$D = \sum\limits_{i=1}^{n} \frac{1}{s_i} E(X_i - \pi_i)^2$$

Поскольку $P(S > \pi)$ уменьшается при увеличивающемся $\pi$, то задача состоит в нахождении максимального значения суммарной премии $\pi= \pi_1+ \ldots + \pi_n.$\\

Применяя формулу(16) для $N=n,  \xi_i = X_i, a_i= \mu_i, A_i= \pi_i , \omega_i= \frac{1}{s_i},$  мы можем утверждать, что минимизационная задача 5  имеет единственное решение 

$$\pi_i^* = \mu_i + s_i  \sqrt{ \frac{D - \sum\limits_{i=1}^{n} \frac{1}{s_i} \sigma_i^2 }{\sum\limits_{i=1}^{n} s_i}} \eqno(23)$$


Теперь опять предположим, что портфель может быть разделен на k классов однородных рисков с одинаковыми статистическими свойствами потерь. Пусть i-й класс состоит из $n_i$ рисков с одинаковым средним $\mu_i$
и одинаковыми дисперсиями $\sigma_i^2.$ Тогда общее количество потерь $S_i$  в i-м классе  имеет среднее значение $ES_i= n_i \mu_i$ и дисперсию $ Var S_i = n_i \sigma_i^2.$ Общее число потерь от всего портфеля есть 
$S= S_1 + \ldots + S_k$, причем $\mu \equiv ES= \sum\limits_{i=1}^{k} n_i \mu_i $ , 
$\sigma^2 \equiv VarS= \sum\limits_{i=1}^{k} n_i \sigma_i^2 $\\
 
Благодаря однородности рисков внутри отдельного класса i, страховщик должен взимать со всех рисков в этом классе одну и ту же премию $\pi_i.$ Тогда общая премия за все риски в портфеле равна $\pi = \sum\limits_{i=1}^{k} n_i \pi_i.$\\

Рассмотрим оптимизационную задачу:

{\bf Задача 6} Минимизировать вероятность разорения $R=P(S>\pi)$ при заданной величине 
$$D = \sum\limits_{i=1}^{k} \frac{1}{r_i} E(S_i - n_i \pi_i)^2$$

Поскольку $P(S > \pi)$ уменьшается при увеличивающемся $\pi$, то задача состоит в нахождении максимального значения суммарной премии $\pi= n_1 \pi_1+ \ldots + n_k \pi_k.$\\

Применяя формулу(16) для $N=k, \xi_i=S_i,  a_i= n_i \mu_i, A_i= n_i \pi_i , \omega_i= \frac{1}{r_i},$  мы можем утверждать, что минимизационная задача 6  имеет единственное решение 

$$\pi_i^* = \mu_i + \frac{r_i}{n_i}  \sqrt{ \frac{D - \sum\limits_{i=1}^{k} \frac{1}{r_i} n_i \sigma_i^2 }{\sum\limits_{i=1}^{k} r_i}} \eqno(24)$$




\end{document}
