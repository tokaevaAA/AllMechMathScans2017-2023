
\documentclass[12pt,a4paper]{article}
\usepackage[utf8]{inputenc}
\usepackage[english,russian]{babel}
\usepackage{indentfirst}
\usepackage{misccorr}
\usepackage{graphicx}
\usepackage{amsmath}
\begin{document}




\begin{titlepage}

 \begin{center}
 Московский Государственный Университет имени М.В.Ломоносова \\
Механико-математический факультет\\
Кафедра теории вероятностей
  \end{center}

 \vspace{3cm}
 
 \begin{center}
   
  {  Курсовая работа за 3 курс:\\
  Принципы назначения страховых премий.}
   
    \vspace{5cm}
\end{center}     
   
     
   \hspace{170pt}  {Выполнила: Александра Токаева,  309 \\}
       
 \vspace{0.1cm}
  \hspace{170pt} 	  Научный руководитель:  Г.И.Фалин\\

\vspace{4cm}

  \begin{center}
  {Москва\\
  2020}
  \end{center}  
 
 
\end{titlepage}




Таким образом, мы поняли, как назначать страховые премии исходя из минимизации риска разорения, а теперь разработам теорию меры риска.\\

{\bf \large Часть2:  Меры риска}\\


В актуарной науке недавно разработали меры риска, основанные на искаженных вероятностях, которые используются 
для формирования страховых тарифов. Примером является  пропорциональное преобразование факторов риска.
Мера риска должна удовлетворять свойствам  "неприятия риска"  и  "разнообразия" как в страховании, так и  при принятии решений о размещении активов. Мера риска, основанная на искаженных вероятностях, не согласуется с  изменением меры, которое используется в финансовой экономике для ценообразования. Также эта мера непоследовательна в своем обращении со страхованием и инвестиционными рисками.  Мы предлагаем меру риска, которая обладает свойствами неприятия риска и разнообразия, является аддитивной и  последовательной в своем отношении к страхованию и инвестиционным рискам. \\

{\bf \large 2.1 Введение}\\
В актуарной науке модели используются как для количественной оценки рисков, так и для ценообразования. 
Количественная оценка рисков требует меру риска для преобразования случайной будущей прибыли или убытка  в 
эквивалент уверенности, который может потом  использоваться для упорядочивания рисков и для целей принятия решений. Для количественной оценки рисков требуется указать распределение вероятности задействованных рисков и 
применить функцию предпочтения к этим распределениям вероятности.  Таким образом, этот процесс использует как статистические предположения, так и экономические.  Полученная мера риска должна удовлетворять желаемым свойствам. Эти свойства включают в себя неприятие риска (что является основополагающим для страхования) и разнообразие (что  является основополагающим для теории портфелей и выбора инвестиций). \\

Для того, чтобы назначить цену или премию за риск, нужно перевести случайную будущую прибыль или убыток  в финансовые термины. Кроме распределения вероятности этой прибыли или убытка, требуется также принцип назначения премии или цены. При оценке риска этот принцип назначения премий используется, чтобы преобразовать случайную прибыль или убыток в премию или цену. Таким образом, компонентами модели для оценки рисков являются статистическая модель для рисков, экономическая модель для предпочтения рисков и принцип назначения премий для преобразования меры риска в денежный эквивалент.\\
Цены или премии  также должны удовлетворять некоторым базовым свойствам. Используемая модель должна предоставлять последовательные(непротиворечивые) и разумные результаты. Это включает в себя соответствие с наблюдаемым поведением финансового или страхового рынка и последовательное обращение с разными рискам. Также важно иметь последовательное(логичное) обращение с доходами и убытками по активам и пассивам(обязательствам).\\
В актуарной науке было предложено множество принципов назначения премий.  Гувертц рассматривает много принципов и присущие им свойства. В финансовой экономике важным требованием к непротиворечивой финансовой модели является принцип отсутствия арбитража. Панждер рассматривает как равновесные оценки рисков, так и оценки, построенные на принципе отсутствия арбитража. Ванг предлагает принцип, основанный на пропорциональном преобразовании функции риска. Принцип назначения премий согласуется с эквивалентом уверенности/определенности в дуальной теории предполагаемой полезности, разработанной Яри. Эти подходы к оценке страховых контрактов обращаются со  страховыми потерями как с положительными случайными величинами и назначают премии выше, чем предполагаемое значение страховых потерь.

{\bf \large 2.2  Допущения и обозначения}\\
Случайные прибыль и потери, возникающие из страховых потерь и инвестиционных решений, обозначаются случайными величинами $X_i.$ Для каждой случайной прибыли или потери обозначим функцию распределения как $F_i(x)=P(X_i \leq x) $ и дополнительную функцию распределения как ${\overline F_i}(x)=P(X_i > x).$ Обобщенная обратная к дополнительной функции распределения определяется как 
$${\overline F^{-1}}(q) = inf\{x: {\overline F}(x) \leq q\}, 0 \leq q < 1, {\overline F^{-1}}(1)=0$$

Мы смотрим на все с точки зрения человека-страхователя, поэтому убыток $X_i$ является неположительным.  Выплата, совершаемая страховщиком, когда происходит страховой случай, обозначается $Y_i$ и является неотрицательной. 
У инвестиции из активов $X_i$ точно будет ниже, чем у ценной бумаги с ограниченной ответственностью. Обычно страховая компания будет покрывать (только)  убытки стоимости активов, происходящие из заранее оговоренных событий. В этом случае человек-страхователь тоже будет держателем активов и будет подвержен инвестиционным рискам, проистекающим из колебаний, вызванных экономическими и другими факторами. Полный риск по активу включает как экономические колебания в цене, так и потери по застрахованным событиям.\\
Мы предполагаем, что любая модели оценки рисков и назначения цен и премий должна удовлетворять свойству, что функция предпочтения рисков должна удовлетворять "неприятию риска" и "разнообразию портфеля".  Оба эти свойства считаются обязательными для финансовой модели, используемой для оценки рисков и распределения активов. Многие реальные финансовые решения, например, покупка страхования и инвестиционные решения, соответствуют  свойствам  "неприятия риска" и "разнообразия портфеля". Разнообразие портфеля подразумевает неприятие риска для предполагаемых моделей применения. Однако свойства неприятия риска недостаточно для наличия свойства разнообразия портфеля в случае общих функций предпочтения рисков. Мы определяем неприятие риска и разнообразие портфеля следующим образом:\\
{\bf Определение 1} { \itshape Неприятие риска }выражается в том, что неслучайная сумма (например, матожидание потери по риску) предпочтительнее, чем риск (то есть случайное событие), то есть честные с точки зрения актуарной науки азартные игры не предполагаются.\\
{\bf Определение 2} { \itshape Предпочтение разнообразия портфеля } выражается в том, что выпуклая комбинация (то есть линейная комбинация точек, где все коэффициенты неотрицательны, и их сумма равна 1) предпочтительнее, чем один риск, в предположении, что риски эквивалентны.\\

Предполагается, что на функциях распределения существует отношение предпочтительности $ \succeq$,  где символ $\succ$ обозначает строгую предпочтительность, а символ ~ обозначает безразличие/эквивалентность. Например, выражение $F_1 \succ F_2$ обозначает, что случайные прибыль или потери  $X_1$ с функцией распределения $F_1$  строго предпочтительнее, чем случайные прибыль или потери  $X_2$ с функцией распределения $F_2$.\\


{\bf 2.3 Ожидаемая полезность}\\
Аксиоматический подход к выводу функции предпочтительности, предназначенной для упорядочивания рисков, использующий предполагаемое применение, был дан фон Нейманом и Моргенштейном. Аксиомы также присутствуют у Ванга и Янга. Ключевая аксиомой является так называемая аксиома независимости. Эта аксиома утверждает, что если $X \succ Y$ и Z - произвольный риск, то
$$\{(\alpha,X),(1-\alpha,Z)\} \succ \{(\alpha,Y),(1-\alpha,Z)\} $$
для любых $0 \leq \alpha \leq 1 $, где $ \{ (\alpha,X),(1-\alpha,Z) \}$ - это вероятностная смесь с такой функцией распределения:
$$F_{  \{(\alpha,X),(1-\alpha,Z)\} } = \alpha F_X(x) + (1-\alpha)F_Z(x),$$
что эквивалентно:
$$\overline F_{  \{(\alpha,X),(1-\alpha,Z)\} } = \alpha {\overline  F}_X(x) + (1-\alpha ){ \overline F} _Z(x)$$.

Мы это можем переписать как 
$$\alpha {\overline  F}_X(x) + (1-\alpha ){ \overline F} _Z(x) \succ \alpha {\overline  F}_Y(x) + (1-\alpha ){ \overline F} _Z(x).$$

Предположим, что у человека есть изначальный капитал W. Из этой аксиомы можно показать, что существует функция полезности u такая что 
$F_1 \succ F_2 \Leftrightarrow  E \left[ u(W + X_1) \right]  >  E \left[u(W + X_2) \right] .$  
Свойства и приложения функций применения рассмотрены в статье Гербера и Пафуми.\\
Поскольку мы предполагаем неприятие риска, $u(x)$ - это возрастающая выпуклая вверх функция от X,  которая дифференцируема по крайней мере 2 раза, причем $u^{'} > 0$ и $u^{''} < 0.$ Функции полезности единственны с точностью до аффинных преобразований, то есть $u^{*}(X) = au(X) + b $ дает то же самое упорядочивание рисков, что и $u(X).$ Функции полезности можно стандартизовать, положив $u(k)=0$ и $u^{'}(k) = 1$ для некоторой точки k.\\
Для страхования убыток $X_i$ будет отрицательным. Если мы запишем потери как положительную случайную величину $Y=-X,$ то $ E \left[ u(X) \right]  = E  \left[ u(-Y) \right] . $\\

Из-за неприятия риска $$ E \left[  u(X) \right]   \leq  u(E  \left[  X \right]  ),$$
поскольку функция u выпукла вверх (типа модуль интеграла не превосходит интеграла от модуля, только наоборот, поскольку модуль выпуклый вниз, а функция u выпукла вверх), и это верно для всех рисков, где X отрицательно для страховых потерь.\\
Аналогично (просто взяв монотонно возрастающую функцию u от обеих частей предыдущего неравенства) мы имеем $$u^{-1}  \left[ E \left[ u(X) \right] \right]   \leq  E  \left]  X \right]  $$

Для страховых рисков, обозначив $Y=-X,$ где Y - это неотрицательная случайная величина, получим 
$$ - u^{-1}  \left[ E \left[ u(X) \right] \right]   \geq  E  \left]  X \right]  $$

Безрисковый эквивалент  определяется так:\\

{\bf Определение 3}  Если человеку безразлично обладание риском и получение неслучайной суммы, то эта неслучайная сумма называется { \itshape безрисковым эквивалентом } этого риска.\\

Обозначим безрисковый эквивалент риска X как $C_X.$ Для человека с текущим капиталом W мы имеем:
$$u(W + C_X) = E \left[   u(W + X)  \right]  $$
$$\Leftrightarrow C_X = u^{-1}  \left[  E \left[   u(W + X)  \right]  \right]  - W$$

Для страховых рисков, обозначив величину убытка $Y=-X$ и безрисковый эквивалент как $\pi_Y,$ то

$$ u(W  -  \pi_Y ) = E \left[   u(W - Y)  \right] $$
$$\Leftrightarrow  \pi_Y =  W- u^{-1}  \left[  E \left[   u(W - Y)  \right]  \right]  $$

Подчеркнем, что $  \pi_Y = - C_{-Y}$

По неравенству Йенсена для любой случайной величины X и выпуклой вверх функции u:
$$ u(W + E \left[ X \right] )  \geq  E \left[   u(W + X)  \right]  $$

Поэтому из системы 
$$\left[ \begin{array}{crl}
u(W + C_X) = E \left[   u(W + X)  \right] \\
u(W + E \left[ X \right] )  \geq  E \left[   u(W + X)  \right]\\
X \succeq Y \Leftrightarrow u(X) \geq  u(Y)
\end{array}\right. $$

получаем:

$$\Rightarrow  E \left[ X \right]   \geq  C_X.$$

Аналогично получаем  $$  \pi_Y \geq  E \left[ Y \right] $$ для неотрицательной случайной величины потерь Y.\\

Рассмотрим портфель рисков $ \{ X_1, \dots, X_n \} ,$ заданный как $\sum\limits_{i=1}^{n} \alpha_i X_i.$ Из-за свойства предпочтения разнообразия портфеля мы считаем, что человек предпочитает владение таким портфелем рисков владению любым из рисков портфеля по отдельность, в предположении идентичности всех этих рисков $X_i.$ Поэтому из-за предпочтения разнообразия портфеля мы имеем:
$$C_{portfolio} \succ C_{X_i}$$
где $C_{X_1}= C_{X_2} = \dots = C_{X_n}$\\

Для ожидаемой полезности свойство разнообразия тоже выполнено.\\

Для ценообразования, рассмотрим такой выбор портфеля рисков $\{ X_i: i=1, \dots, n \}$ и страховых политик с положительными выплатами убытков $\{ Y_i: i=n+1, \dots, m \}, $ чтобы максимизировать ожидаемую полезность богатства. Предположим, что риск (или политика) номер i имеет цену $P_i.$ Для инвестиционных рисков $X_i$: $P_i$ является величиной начальных вложений, а для инвестиционных политик с выплатами $Y_i$: $P_i$ является премией.\\

Наша задача максимизировать 
$$E \left[ u \left(  \sum\limits_{i=1}^{n} \alpha_i X_i +  \sum\limits_{i=n+1}^{m} \alpha_i Y_i \right)  \right] $$
с учетом того, что $$W=  \sum\limits_{i=1}^{m} \alpha_i P_i$$

То есть у нас изначально было W денег, мы на часть денег купили рискованные финансовые инструменты, а на оставшуюся часть денег приобрели договор страхования.

Применим метод множителей Лагранжа для нахождения условного экстремума нашей функции (условие - наше единственное ограничение  $W=  \sum\limits_{i=1}^{m} \alpha_i P_i$)

Функция Лагранжа  равна
$$  L(\alpha_1, \dots, \alpha_m, \lambda)= E \left[ u \left(  \sum\limits_{i=1}^{n} \alpha_i X_i +  \sum\limits_{i=n+1}^{m} \alpha_i Y_i \right)  \right]  - \lambda  \left[ W - \sum\limits_{i=1}^{m} \alpha_i P_i   \right] $$



Дифференцируем $L(\alpha_1, \dots, \alpha_m, \lambda)$ по $\alpha_i$ и приравниваем производные к нулю:

$$ L^{'}_{\alpha_i}= E \left[  u^{'}(W^{*}) X_i \right]  + \lambda P_i = 0  ,   i = 1, \dots, n$$

$$ L^{'}_{\alpha_i}= E \left[  u^{'}(W^{*}) Y_i \right]  + \lambda P_i= 0  ,  i = n+1, \dots, m$$

где $ W^{*} = \sum\limits_{i=1}^{n} \alpha^{*}_i X_i +  \sum\limits_{i=n+1}^{m} \alpha^{*}_i Y_i $\\

Следовательно, 
$$  \lambda P_i = -  E \left[  u^{'}(W^{*}) X_i \right] ,   i = 1, \dots, n$$
$$  \lambda P_i = -  E \left[  u^{'}(W^{*}) Y_i \right] ,   i = n+1, \dots, m$$


Предположим, что $X_i$ не содержит риска обладания инвестиционным или страховым риском. То есть это безрисковое владение текущим капиталом без подверженности экономическим рискам или страховым событиям. Тогда $$ P_i= X_i= k,$$  где k - константа, тогда оптимальное значение лагранжева множителя $\lambda$ есть 

$$ \hat  {\lambda } = -  E \left[  u^{'}(W^{*}) \right] $$

Тогда мы имеем 

$$  P_i = \frac { E \left[  u^{'}(W^{*}) X_i \right] } { E \left[  u^{'}(W^{*}) \right] },   i = 1, \dots, n$$
$$  P_i = \frac { E \left[  u^{'}(W^{*}) Y_i \right] } { E \left[  u^{'}(W^{*}) \right] },   i = n+1, \dots, m$$


Следовательно, мы можем выразить цену/премию за риск $X_i$ как 
$$P_i= E \left[ \Psi X_i  \right]$$

а премию за страховой контракт как 

$$P_i= E \left[ \Psi Y_i  \right]$$

где $$ \Psi =  \frac 
{  u^{'}(\sum\limits_{i=1}^{n} \alpha^{*}_i X_i +  \sum\limits_{i=n+1}^{m} \alpha^{*}_i Y_i )} 
{ E \left[  u^{'}(\sum\limits_{i=1}^{n} \alpha^{*}_i X_i +  \sum\limits_{i=n+1}^{m} \alpha^{*}_i Y_i ) \right] }$$


Теперь используем случайную величину $\Psi$ как производную Родона-Никодима для того, чтобы изменить вероятностную меру P на Q так, чтобы для новой меря мы имели $P_i= E_{Q} \left[ X_i  \right].$
Тогда для новой вероятностной меры ценой каждого риска является его матожидание, поэтому мы можем называть меру Q  "нейтральной мерой риска". \\
Премии за страховые контракты, использующие такое преобразование меры, являются аддитивными для всех рисков. Поэтому они соответствуют нашему требованию непротиворечивого обращения с рисками (ну было бы логично, что премии  можно складывать, и так и получилось).\\


{\bf  2.4 Двойственная теория полезности и функция искажения }

Двойственная теория Яри разработала альтернативу ожидаемой полезности, используя аксиоматический подход, в котором аксиома независимости заменена на двойственную аксиому независимости.  Двойственная аксиома независимости говорит, что если $ X \succ Y$ и Z- произвольный риск, то

$$\alpha {\overline  F}^{-1}_X(x) + (1-\alpha ){ \overline F}^{-1} _Z(x) \succ \alpha {\overline  F}^{-1}_Y(x) + (1-\alpha ){ \overline F}^{-1} _Z(x).$$
для любых $0 \leq \alpha \leq 1.$

Эти аксиомы подразумевают, что существует такая непрерывная неубывающая функция g, что 

$$F_1 \succ F_2 \Leftrightarrow 
 - \int^{1}_{0} g(q)d{\overline  F}^{-1}_1(q)   >   - \int^{1}_{0} g(q)d{\overline  F}^{-1}_2(q)  $$ 


Подставив $q={\overline  F}_i(x),$ получим,  что:

$$  - \int^{1}_{0} g(q)d{\overline  F}^{-1}_1(q)  = \int^{\infty}_{0} g({\overline  F}_i(x))dx$$

Яри отмечает, что $U_{X_i} =   \int^{\infty}_{0} g({\overline  F}_i(x))dx$ является полезностью, которая приписывает случайной величине безрисковый эквивалент. Таким образом, человек будет безразличен по отношению к получении неслучайной суммы $U_{X_i}$ или риска $X_i.$\\

Ванг предлагает назначать цену страховым рискам, используя функцию искажения, основанную на пропорциональном преобразовании риска. Для страхового риска $Y$ (неотрицательной случайной величины) с дополнительной функцией распределения $ {\overline  F}_Y(x) = P(Y > x),$ Ванг предлагает принцип назначения премий $H_{r}(X) = \int^{\infty}_{0} ({\overline  F}_i(x))^{r}dx, $ где $0 \leq r \leq 1.$

 $H_{r}(X)$ используется для расчета премий с поправкой на риск. Отметим, что функция $g(x)= x^{r}, 0 \leq r \leq 1$ выпукла вверх.\\
 
 Ванг и Янг определяют функцию искажения как неубывающую функцию, удовлетворяющую условиям  $g(0)=0$ и $g(1)=1,$  такую что для неотрицательной случайной величины Y безрисковые эквиваленты  
 $ H_{g} \left[ Y \right] $ и $ H_{g} \left[ -Y \right] $   задаются как 
 
 $ H_{g}(Y) = \int^{\infty}_{0} g({\overline  F}_Y(x))dx 
 =  - \int_{0}^{1} g(q)d{\overline  F}^{-1}_Y(q) \\
 = - (g(q){\overline  F}^{-1}_Y(q) \|_{0}^{1}) + \int_{0}^{1} {\overline  F}^{-1}_Y(q))dg(q)
 =  \int_{0}^{1} {\overline  F}^{-1}_Y(q))dg(q)$\\
 
 
 и $$ H_{g} (-Y) = - H_{\tilde g} (Y),$$  
 
 где $\tilde g $ - это функция искажения, определяемая как $ \tilde g = 1 - g(1-q), 0 \leq r \leq 1.$\\
 
 Отметим, что  $\tilde g $ выпукла вверх, если g выпукла вверх.\\
 
 Безрисковый эквивалент  $ H_{g}(Y) $ удовлетворяет условиям:\\
 
$\bullet$ Если g выпукла вверх, то  $ H_{g}(Y)  \geq E(Y).$\\
$\bullet$  $ H_{g}(aY+b) = a H_{g}(Y) + b, a,b \geq 0.$\\
$\bullet$ Если g выпукла вверх, то $ H_{g}(Y_1 + Y_2 )  \leq  H_{g}(Y_1)  + H_{g}(Y_1).$\\
 

Третье свойство называется суб-аддитивность. $H_{g}(Y)$ будет аддитивна в специальном случае 
ко-монотонных рисков. Риски $X_1$ и  $X_2$ являются ко-монотонными, если существует риск Z и неубывающие действительнозначные функции f и h, такие что $X_1= f(Z)$ и  $X_2= h(Z).$  Понятие ко-монотонных рисков является расширением полной корреляции. \\

Если бы нам нужно было применить этот безрисковый эквивалент к инвестиционным рискам, то из-за неприятия риска, мы бы требовали, чтобы $H_{g} \left[ X \right] \leq E \left[ X \right] , $ что означало бы, что g выпукла вверх. Для страхового риска $Y=-X$ (неотрицательной случайной величины), согласно Вангу и Янгу мы имеем:

$$H_{g} \left[ Y \right]  =  H_{g} \left[ - X \right]  = - H_{\tilde g}(X) \eqno (1)$$

В этом случае $\tilde g$ выпукла вверх из-за неприятия риска, поэтому g тоже выпукла вверх. Поэтому дальше мы получаем, что 
$$H_{\tilde g}(X) \leq E \left[ X \right]$$

$$\Rightarrow - H_{\tilde g}(X) \geq  - E \left[ X \right] =  E \left[ Y \right]$$

$$\Rightarrow H_{ Y}(X) \geq   E \left[ Y \right]$$

Отметим, что для выпуклой вверх функции g:

$$H_g \left[ \sum\limits_{i=1}^{n} \alpha_i X_i \geq  H_{g} \left[ Y \right]  \right],$$

если $ H_g \left[ X_i] \right] = H_g \left[ X_1] \right] , i= 1 \dots, n.$ Таким образом, свойство разнообразия выполнено для инвестиционных рисков.\\

Если мы рассматриваем страховые убытки как неотрицательные случайные величины $Y_i$, то необходимо использовать выпуклую вверх функцию g, соответствующую выпуклой вверх функции ${\tilde g},$ используемой для инвестиционного риска $X_i.$ Однако свойство разнообразия не выполняется для выпуклой функции g. В результате,  рисковая мера искажений  не подходит для управления активами и пассивами. Пассивы (или убытки по рискам) и риски  активов обрабатываются не  соответствующим образом, поскольку для рисков  активов  выполнено свойство разнообразия, но оно не выполнено для рисков пассивов/обязательств. \\

 Мы уже отмечали, что безрисковый эквивалент  $ H_{g}(Y) $ аддитивен только для ко-монотонных рисков. Но мы требуем, чтобы принцип назначения премий был аддитивным, поскольку цены финансовых активов аддитивны.  В действительности означает, что страховые премии назначаются на рынке  на основе полной информации, без учетов стоимости транзакций и других дефектов, а также с учетом принципа отсутствия арбитража на страховом рынке. Это соответствует понятию равновесия на страховом ранке и ценообразованию на финансовые активы при идеальных рыночных условиях. В этой статье мы не будем разбирать задачи ценообразования при ослаблении этих условий.\\
 
 Для меры риска   $ H_{g} \left[ X \right]  $ при условии выпуклости вверх функции g мы имеем свойства неприятия риска и разнообразия портфеля. Но мы не хотим использовать этот принцип, поскольку мы хотим, чтобы премии были аддитивными. Это свойство аддитивности выполняется только для ко-монотонных рисков, а мы хотим, чтобы оно выполнялось для всех рисков. \\
 
Заинтересованные подходом функцией искажения  в построении  меры риска $ H_{g} \left[ X \right]  ,$ мы предлагаем метод назначения премий, кторый будет обладать свойством аддитивности премий. Кроме того, на основе этой меры ценообразования мы определяем безрисковый эквивалент, используя выпуклые вверх функции, похожие на функции полезности. Этот безрисковый эквивалент обладает желаемыми свойствами неприятия риска, разнообразия портфеля и еще имеет более логичное упорядочивание рисков, похожее на упорядочивание функцией полезности.


{\bf  2.5 Изменение меры для премий с использованием функций искажения }

Подход  Ванга к страховым премиям  с использованием функций искажения  использует распределение потерь как положительные случайные величины. Получающаяся премия эквивалента безрисковому эквиваленту в дуальной теории ожидаемой полезности.  Если бы нам нужно было применить этот безрисковый эквивалент к инвестиционным рискам, то мы бы использовали выпуклую вверх функцию $\tilde g,$  эквивалентную выпуклой вверх функции g, которая используется для страхового ценообразования.\\

Страховая премия $\geq$ матожидания риска, это соответствует свойству неприятия риска. Однако, если наш подход  функций искажения  для страховых рисков применить для их ценообразования, то получающиеся страховые премии оказываются неаддитивными, кроме случая ко-монотонных рисков.\\

Мы требуем такой принцип назначения премий для портфеля рисков, что он удовлетворяет свойству неприятия риска, и соответствующая мера риска не противоречит использованию выпуклой вверх функции полезности, которая применяется к страховым рискам.\\

Для выплат по портфелю $Y_\alpha = \sum\limits_{i=1}^{n} \alpha_i Y_i $ страховых контрактов, где выплачиваемые суммы являются положительными случайными величинами $\{ Y_1, \dots, Y_n \}$ с распределениями $(F_1, \dots, F_n),$ мы назначаем премию 
$$\pi_{r}(Y_\alpha) = E_{P_{n, r}}Y_{\alpha},$$ 
где $P_{n,r} $ есть вероятностная мера, соответствующая распределению
$$ F_{n,r} = \prod_{i=1}^{n} F_{i}^{r}, 0 < r \leq 1.$$

Очевидно, что такой принцип назначения премий будет аддитивным по портфелю рисков, причем
$$\pi_{r}(Y_{\alpha}) \geq EY_{\alpha}.$$

Чтобы непротиворечиво обращаться с инвестиционными и страховыми рисками, мы определяем меру риска портфеля как 
$$ U_{r}(Y_{\alpha} = u^{-1}(E_{P_{n, r}}u(Y_{\alpha})),$$
где u выпукла вверх.\\

{\bf  Теорема 1} Пусть u - выпуклая вверх, возрастающая и дважды дифференцируемая нелинейная функция, $u^{'} > 0,$  и пусть матожидания $E_{P_{n,r}}Y_{j}$ - непрерывные слева функции от r в точке $r=1.$\\
Тогда существует такое $0 < r^{*} < 1,$ что для всех $r^{*} \leq r < 1$ и любого $\alpha = (\alpha_1, \dots, \alpha_n),$  причем $\sum\limits_{i=1}^{n} \alpha_i = 1,$ будет верно:

$$U_{r}(Y_{\alpha} ) \leq EY_{\alpha},$$

то есть мера риска $U_{r}( \dot ) $ обладает свойством неприятия риска.\\

{\bf  Доказательство} Из выпуклости вверх  функции u следует, что $u(x) \leq u(0) + u^{'}(0)x.$ Без потери общности, gредположим, что $u(0) = 0.$ Тогда, из условий теоремы, функция 

$$ \Pi(\alpha_1, \dots, \alpha_n, r) = u^{-1} (  E_{P_{n,r}} u (\sum\limits_{i=1}^{n} \alpha_i Y_i))$$

является непрерывной слева функцией от r  в точке $r=1$ для любого $\alpha = (\alpha_1, \dots, \alpha_n).$ \\

Это значит, что $$\lim \limits_{r  \to 1-0}  \Pi(\alpha_1, \dots, \alpha_n, r) = \Pi(\alpha_1, \dots, \alpha_n, 1)  = u^{-1}(Eu  (\sum\limits_{i=1}^{n} \alpha_i Y_i)),$$

и эта сходимость равномерная по $ \alpha \in H = \{  \sum\limits_{i=1}^{n} \alpha_i = 1\},$ потому что гиперплосткость H компактна. \\

Тогда для $\Delta > 0$ существует $ 0 < r^{*} < 1$ такое что для всех $r^{*} \leq r < 1$ 

$$ \left| \Pi(\alpha_1, \dots, \alpha_n, r)  - u^{-1}(Eu  (\sum\limits_{i=1}^{n} \alpha_i Y_i)) \right| < \frac {\Delta}{2} \eqno (2)$$

С другой стороны, для выпуклой вверх нелинейной функции u, функция :

$$ \Delta(\alpha) = \pi_{r_{\left| r=1 \right.}} (Y_{\alpha} - \Pi(\alpha_1, \dots, \alpha_n, 1) $$ 

непрерывна по $\alpha$ для всех $\alpha \in H$\\

А поскольку $\alpha \in H,$ а H - компактное множество, то

$$ \inf \limits_{\alpha \in H} \Delta(\alpha) = \Delta(\alpha^{*}) > 0, \alpha^{*} \in H.$$

Тогда 

$$ EY_{\alpha} - U_{r}(Y_{\alpha})  =   \pi_{r_{\left| r=1 \right.}} (Y_{\alpha} - \Phi(\alpha_1, \dots, \alpha_n, \alpha )$$
$$ \geq \Delta(\alpha) - \left| \Phi(\alpha_1, \dots, \alpha_n, 1) - \Phi(\alpha_1, \dots, \alpha_n, \alpha ) \right|$$
$$ \geq \Delta(\alpha^{*}) - \Delta(\alpha^{*})/2 > 0, r^{*} < r < 1$$


Предложенная мера риска также будет обладать свойством разнообразия для страховых потерь, когда эти потри рассматриваются как отрицательные (неположительные) случайные величины.

{\bf  2.6 Непротиворечивость меры риска для инвестирования и меры риска для потерь в отношениях предпочтения} \\

Предположим, что мы непротиворечиво определяем меру риска $R_{m_{I}}(X) $ для инвестиций
и меру риска $R_{m_{L}}(X) $  для неотрицательных  рисков. В предыдущих разделах мы рассматривали некоторые специальные определения  $R_{m_{I}}(X) $ и $R_{m_{L}}(X) :$\\

1) Ожидаемая полезность: для выпуклой вверх функции u

$$ R_{m_{I}}(X)= u^{-1}  \left[  E \left[   u(W + X)  \right]  \right]  - W$$
$$ R_{m_{L}}(Y)= W- u^{-1}  \left[  E \left[   u(W - Y)  \right]  \right]  - W \eqno (3)$$

2) Функции искажения : для выпуклой вверх функции u

$$ R_{m_{I}}(X)= H_{\tilde g}(X)$$
$$ R_{m_{L}}(Y)= H_{g}(Y)$$

где $\tilde g (q) = 1 - g(1-q)$ - выпуклая вниз функция.

3) Изменение меры: для $r^{*} \leq r < 1$

$$  R_{m_{I}}(X) = U_r(X) = u^{-1}(E_{P_{n,r}}u(X))$$
$$  R_{m_{L}}(Y) =E_{P_{n,r}}(Y) \eqno (4)$$

Меры  $R_{m_{I}}(X) $ и $R_{m_{L}}(X) $ будут непротиворечивыми относительно отношения предпочтения, если они порождают одинаковое отношение предпочтения, то есть для двух неотрицательных рисков из портфеля:

$$ X \prec _{ R_{m_{I}}} Y \eqno (5) $$
 и 
 
 $$ X \prec _{ R_{m_{L}}} Y \eqno (6) $$

должны выполняться одновременно. Отметим, что меры, заданная как ожидаемая полезность, очень близки к нашему требованию непротиворечивости относительно отношения предпочтения. На самом деле, если мы определяем $R_{m_{I}}(X) $ и $R_{m_{L}}(X) $ по формулам (3), то из (5) мы получаем, что

$$ W + X \prec_{u} W + Y \eqno(7)$$,

где отношение предпочтения $\succ_{u}$ задается $Eu(\dot)$ и эквивалентно 

$$ W - Y \succ_{u} W - X \eqno(8)$$

Видно, что (7) и (8) могут считаться эквивалентными для любой функции u, симметричной относительно W, то есть $u(W+X) = -u(W-X).$\\

Мы также можем рассматривать менее ограничительные формы непротиворечивости мер для инвестиций и для потерь. Например, мы можем потребовать эквивалентность (5) и  (6) для инвестиций и потерь, которые упорядочены  во втором стохастическом доминировании (или порядке остановки потерь).

{\bf  Определение 4 } Риск X меньше, чем риск Y в SSD, если для всех $x \geq 0$

$$ \int_{x}^{\infty} \overline F_{X}(u)du \leq  \int_{x}^{\infty} \overline F_{Y}(u)du \eqno(9)$$

где $ \overline F_{X}(x) = 1 - F_{X}(x)$ - дополнительная функция распределения X.

Определение 4 эквивалентно тому, что $Eu(X) \leq Eu(Y).$



{\bf  Определение 5 }  Две меры $R_{m_{I}}(X) $ и $R_{m_{L}}(X) $ дают непротиворечивое отношение предпочтения по отношению к SSD, если для $X \prec_{SSD} Y$ соотношения (5)  и (6) выполнены одновременно. 

Покажем сначала, что вообще говоря, меры, построенные по функциям искажения, не дают  непротиворечивое отношение предпочтения по отношению к SSD. 


{\bf  Пример 1 }  Рассмотрим $g(t) = 1 - (1-t)^m,$ $m>1,$ m- целое, тогда $\tilde g (t) = t^m.$
Видно, что $g(t)$ выпукла вверх, а $\tilde g $ выпукла вниз, причем 

$$ R_{m_{I}}(X) = H_{\tilde g}(X) = \int_{0}^{\infty}  (\overline F_{X}(x))^m dx = E(min(X_1, \dots, X_m)) \eqno(10)$$

$$ R_{m_{L}}(Y) = H_{ g}(Y) = \int_{0}^{\infty}(1-  (F_{Y}(x))^m) dx = E(max(X_1, \dots, X_m)) \eqno(11)$$

где $(X_1, \dots, X_m)$ - независимые одинаково распределенные величины с функцией распределения $F_X(x),$ а  $(Y_1, \dots, Y_m)$ - независимые одинаково распределенные величины с функцией распределения $F_Y(x).$\\

Предположим, что $$ \overline F_{X}(x) = I_{(-\infty, 0)} + (1-x)I_{\left[ 0,1 \right]}$$

$$ \overline F_{Y}(x) = I_{(-\infty, 1/4)} + (2-4x)I_{\left[ 1/4,1/2 \right]}$$

то есть $$X \sim R\left[ 0,1 \right]$$
$$Y \sim R\left[ 1/4,1/2 \right]$$


В этом случае, функция $$J(x) = \int_{x}^{1} (\overline F_{Y} (u) - \overline F_{X} (u))du$$

Мы видим, что она положительна на всем промежутке $\left[ 0,1 \right], $ а это значит, что $X \prec_{SSD} Y.$

С другой стороны, используя формулу матожидания min и max равномерных величин::

$$  E(min(X_1, \dots, X_m)) = a + \frac{1}{m+1}(b-a)$$

$$  E(max(X_1, \dots, X_m)) = b- \frac{1}{m+1}(b-a)$$

$$ \Rightarrow  R_{m_{I}}(X)  = E(min(X_1, \dots, X_m)) =  \frac{1}{m+1}  <  R_{m_{I}}(Y)  = E(min(Y_1, \dots, Y_m)) =  \frac{1}{4} + \frac{1}{4} \frac{1}{m+1}$$

$$ R_{m_{L}}(X)  = E(max(X_1, \dots, X_m)) = 1-  \frac{1}{m+1}  >  R_{m_{L}}(Y)  = E(max(Y_1, \dots, Y_m)) =  \frac{1}{2} - \frac{1}{4} \frac{1}{m+1}$$

Это показывает, что  $R_{m_{I}}(X) $ и $R_{m_{L}}(X) $ не являются непротиворечивыми по отношению к SSD для меры функций искажения.\\

Теперь рассмотрим меры риска $R_{m_{I}}(X) $ и $R_{m_{L}}(X), $  определенные соотношением (4), построенные заменой меры, предложенной в этой статье. Мы покажем, что они могут давать непротиворечивое отношение предпочтения относительно SSD после замены меры.\\


{\bf  Определение 6 }  $X \prec_{r-SSD} Y $ в r-SSD, если для всех $x \geq 0$

$$ \int_{x}^{\infty} \overline {F_{X}(u)^r }du \leq  \int_{x}^{\infty} \overline{ F_{Y}(u)^r} du \eqno(12)$$

Подставив $x=0$ и использовав (4), мы получим 

$$ R_{m_{L}}(X)  \leq R_{m_{L}}(Y) \eqno(13)$$

то есть  

$$ X \prec_{ R_{m_{L}}} Y  \eqno(14) $$

С другой стороны, из  (12) следует, что 

$$ E_{P_{n,r}}u(-X) \geq  E_{P_{n,r}}u(-Y)$$

$$\Rightarrow  -X \succ_{ R_{m_{I}}} -Y  \eqno(15) $$

Если u(x) нечетная, то из (15)  следует, что

$$X \prec_{ R_{m_{I}}} Y  \eqno(16) $$

и тогда $R_{m_{I}}(X) $ и $R_{m_{L}}(X) $  являются непротиворечивыми в предпочтении относительно r-SSD. Если же функция u не является нечетной, то мы имеем только (14) и (15).\\



{\bf  Замечание 1  }  Стохастическое доминирование второго порядка также определяется в экономической и финансовой литературе. \\
Риск Y предпочтительнее риска X, если  для всех $x \geq 0:$

$$ \int_{0}^{x} F_{Y}(u)du \leq  \int_{0}^{x} F_{x}(u)du \eqno(17)$$
 
 Если это выполнено, то (17) эквивалентно тому, что $Eu(X) \leq Eu(Y)$ для любой возрастающей выпуклой вверх функции u, то есть это свойство неприятия риска.\\
 
 
 {\bf  Замечание 2  }
 Отметим, что  $$  \int_{x}^{\infty} \overline F_{X}(u)du \leq  \int_{x}^{\infty} \overline F_{Y}(u)du \eqno (9)$$
  и 
  
  $$  \int_{0}^{x} F_{Y}(u)du \leq  \int_{0}^{x} F_{x}(u)du \eqno(17)  $$

 не эквивалентны.\\
 
 Например, в примере 1 
 
 $$F_X(x) = xI_{\left[ 0,1 \right]} + I_{(0,\infty)}$$
 
 $$F_Y(x) = (4x-1)I_{\left[ 1/4,1/2 \right]} + I_{(1/2,\infty)}$$
 
 Тогда функция  $$J(x) = \int_{0}^{x} (F_{X} (u) -  F_{Y} (u))du \eqno(18) $$
 не является знакопостоянной на отрезке $\left[ 0,1 \right].$\\
 
 Отметим, что меры, построенные по функциям искажения, не дают  непротиворечивое отношение предпочтения относительно SSD и в смысле нового определения SSD(см . (17))
 

{\bf  2.7 Выводы  }

Страховые премии могут получаться из мер риска,  для которых отношение предпочтения удовлетворяет свойствам неприятия риска и разнообразия портфеля. В предположениях отсутствия арбитража на рынке и идеальных рыночных условиях, равновесные премии должны быть аддитивными. Кроме того, мы должны применять меру риска к обеим сторонам бухгалтерской ведомости и при этом получать непротиворечивое упорядочивание рисков.\\

Мера риска, основанная на искаженных вероятностях, не удовлетворяет свойству аддитивности равновесных премий для страховых премий.  Премии, основанные на искаженных вероятностях, обычно только суб-аддитивны, а аддитивными являются только для ко-монотонных рисков. Таким образом, для портфеля из разных классов страховых рисков, премии для разных классов будут противоречить свойству равновесности.\\


Мы предложили принцип назначения страховых премий, используя замену меры, которая делает премии по портфелю аддитивными. Этот принцип назначения премий также обладает свойствами неприятия риска, то есть премии всегда превышают ожидаемое значение потерь.  Кроме того, мы предложили меру безрискового эквивалента для риска, основываясь на выпуклой вверх функции полезности и замене меры, которая упорядочивает инвестиционные риски непротиворечиво относительно страховых рисков.









\end{document}
