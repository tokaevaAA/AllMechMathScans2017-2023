\documentclass[a4paper,11pt,english]{article}
\usepackage[utf8]{inputenc}
\usepackage{babel,amsmath,amssymb,amsthm,enumitem}
\usepackage[sort]{natbib}
\usepackage[colorlinks,allcolors=blue]{hyperref}
\usepackage[margin=1.25in]{geometry}
\usepackage[small,bf]{titlesec}
\titlelabel{\thetitle.\hspace{0.5em}}
\PassOptionsToPackage{usenames,dvipsnames}{xcolor}
\usepackage{tikz}

\newtheorem{theorem}{Theorem}
\newtheorem{lemma}{Lemma}
\newtheorem{corollary}{Corollary}
\newtheorem{proposition}{Proposition}
\theoremstyle{definition}
\newtheorem{remark}{Remark}
\newtheorem{definition}{Definition}

\newcommand{\floor}[1]{\lfloor{#1}\rfloor}
\newcommand{\ceil}[1]{\lceil{#1}\rceil}
\DeclareMathOperator{\E}{E}
\DeclareMathOperator{\Var}{Var}
\DeclareMathOperator{\I}{I}
\DeclareMathOperator{\sgn}{sgn}
\DeclareMathOperator{\Law}{Law}
\DeclareMathOperator{\argmax}{arg\,max}
\renewcommand{\hat}{\widehat}
\renewcommand{\tilde}{\widetilde}
\renewcommand{\epsilon}{\varepsilon}
\renewcommand{\P}{\mathrm{P}}
\newcommand{\F}{\mathcal{F}}
\newcommand{\R}{\mathbb{R}}
\newcommand{\bt}{\tilde{b}}
\renewcommand{\b}{\bar b}
\newcommand{\M}{\mathcal{M}}
\let\Ss\S
\renewcommand{\S}{\mathcal{S}}
\newcommand{\cadlag}{c\`adl\`ag}

\newcommand{\red}[1]{{\color{red}#1}}

\title{Interpretation of the existence of a survival strategy in an evolutionary finance model with short-lived assets and affine payoffs}
\author{}
\date{}

\begin{document}
\maketitle

\section{Abstract}
The paper examines a game-theoretic discrete time model of a financial market 
in which asset prices are determined endogenously in terms of a short-run equilibrium. 
This model was proposed in the recent article "Survival strategies in an evolutionary finance 
model with endogenous asset payoffs". The main novelty of the model with affine payoffs 
is that we allow the payoffs of the assets to depend on the agents' strategies, 
which makes the task of a survival strategy search a non-trivial one. 
We elaborate on one of the key results obtained in the above-mentioned article. 
Our main goal is to interpret the existence of a survival strategy as 
the extension of the Tikhonov fixed point theorem.

\section{The model}

Let $(\Omega,\F,\P)$ be a probability space with a complete discrete-time
 filtration $\mathbb{F} = (\F_t)_{t\ge 0}$.
The market in the model consists of $N\ge 2$ agents and $K\ge 2$ short-lived assets. Agent
$n$, where $m=1,\ldots,N$, is  characterized by his wealth $W^n$ and his
strategy $\Lambda^n$. The wealth $W^n = (W_t^n)_{t=0}^\infty$ is an
$\mathbb{F}$-adapted random sequence, where $W_t^n=W_t^n(\omega)$ is the wealth
held by the agent at time $t$. The variables $W_t^n$  depend on the payoffs yielded by the assets and the strategies of the other agents through a certain equation which will be
stated later. The total market wealth at time $t$ will be denoted by $C_t =
W_t^1+\ldots+W_t^N$. 

A strategy of agent $n$ is a sequence $\Lambda^n =
(\Lambda_t^n)_{t=0}^\infty$ of vector-valued functions $\Lambda_t^n =
\Lambda_t^n(\omega,w)$ with values in the standard $K$-simplex $\Delta_K
=\{\bar \lambda \in \R^K_+ : \lambda^1+\ldots+\lambda^K = 1\}$ and measurable with
respect to $\F_t\otimes \mathcal{B}(\R_+)$. For $n=1,\ldots,K$, the $k$-th
coordinate $\Lambda_t^{n,k}$ specifies the proportion of wealth which agent
$n$ allocates for buying asset $k$ at time $t$. Short sales are not allowed.
A strategy may depend on a random state of the world $\omega$ and the total
market wealth $C_t$ (through the argument $c$ of $\Lambda_t^m$), so that if
the total market wealth at state $\omega$ is $C_t(\omega)$, then agent $m$
allocates her wealth in proportions given by the vector $\Lambda_t^m(\omega,
W_t(\omega))$.

It is possible to consider strategies of a more general form, for example
depending on full market history, but this will not increase the generality
of the main results. On the other hand, the dependence on the total market
wealth is necessary and cannot be removed from the model because the
survival strategy that we construct below needs it.

Each asset $n=1,\ldots,K$ at each moment of time $t\ge 1$ yields a random
payoff which depends on the total amount of wealth invested by the agents in
this asset at time $t-1$ according to the formula
\begin{equation}
A_t^k(\omega) = X_t^k(\omega) \sum_{n=1}^N
\lambda_{t-1}^{n,k}(\omega)W_{t-1}^n(\omega) + Y_t^k(\omega),\label{P}
\end{equation}
where $X^k = (X_t^k)_{t=1}^\infty$ and $Y^k = (Y_t^k)_{t=1}^\infty$ are some exogenously
given non-negative random sequence adapted to the filtration $\mathbb{F}$, and
\[
\lambda_{t-1}^{n,k}(\omega) = \Lambda_{t-1}^{n,k}(\omega,
W_{t-1}(\omega)),\qquad W_{t-1}(\omega) = \sum_{n=1}^N W_{t-1}^n(\omega).
\]
We will assume that for each $t\ge 0$ and
$k=1,\ldots,K$ it holds that
\begin{equation}
\P(Y_{t+1}^k >0 \mid \F_t) >0\ \text{a.s.}\label{Y-positive}
\end{equation}

The payoff of each asset is divided between the agents proportionally to the amount of
wealth they allocated for investing in this asset at time $t-1$. As a result, the wealth
sequence of agent $m$ satisfies the recursive relation

\begin{equation}
W_{t+1}^n = \sum_{k=1}^K \frac{\lambda_t^{n,k} W_t^n}{\sum_{n=1}^N
\lambda_t^{n,k} W_t^n} A_{t+1}^n = \sum_{k=1}^K\lambda_t^{n,k} W_t^n
\biggl(X_{t+1}^k + \frac{Y_{t+1}^k}{\sum_{n=1}^N \lambda_t^{n,k} W_t^n}
\biggr).\label{capital}
\end{equation}

It is clear that given an initial condition $W_0=(W_0^1,\ldots,W_0^N)$ and a
strategy profile $\Lambda=(\Lambda^1,\ldots,\Lambda^N)$, the sequence
$C_t=(W_t^1,\ldots,W_t^N)$ is well-defined by the above relation, provided
that for all $t \ge 0$ with probability 1 we have
\begin{equation}
\sum_{n=1}^N \lambda_t^{n,k} W_t^n \neq 0.\label{feasible-1}
\end{equation}

In what follows we will always assume that the objects defining the
market model (i.e.\ agents' strategies and asset payoffs) are such
that inequality \eqref{feasible-1} holds true. A  sufficient condition for its
validity consists in that for each $t\ge 1$ we have, with probability 1,
\begin{equation}
\label{feasible-2}
\sum_{k=1}^K (X_t^k + Y_t^k) >0,
\end{equation}
and there is at least one agent who uses a fully diversified strategy, i.e.\
for some $n=1,\ldots,N$ and all $t\ge 0$, $k=1,\ldots, K$ we have
\begin{equation}
\label{feasible-3}
\lambda_t^{n,k} > 0.
\end{equation}

It is not difficult to see that if inequalities
\eqref{feasible-2}--\eqref{feasible-3} are true, then agent $n$ has 
strictly positive wealth at all moments of time, so \eqref{feasible-1} also
holds. The survival strategy $\hat\Lambda$ which was constructed in the article 2023
satisfies assumption \eqref{feasible-2}.


\section{Results from article 2023: existence of a survival strategy}
\begin{definition}
A strategy $\hat \Lambda$ is called \emph{survival} if for any strategy profile
$\Lambda = (\Lambda^{1},\ldots,\Lambda^N)$ with $\Lambda^1 = \hat \Lambda$ and any
initial wealth vector $W_0=(W_0^1,\ldots,W_0^N)$ with $W_0^1 > 0$ it holds that
\[
\inf_{t\ge 0} W_t^1 > 0\ \text{a.s.}
\]
\end{definition}

Denote by $C_t$ the total market wealth, and $r_t^m$ the relative wealth of agent $n$:
\[
C_t = \sum_{n=1}^N W_t^n, \qquad r_t^n = \frac{W_t^n}{W_t}.
\]

\begin{definition}
A strategy $\hat \Lambda$ is called \emph{relative growth optimal} if for any strategy profile
$\Lambda = (\Lambda^{1},\ldots,\Lambda^N)$ with $\Lambda^1 = \hat \Lambda$ and any
initial wealth vector $W_0=(W_0^1,\ldots,W_0^N)$ with $W_0^1 > 0$ it holds that $\ln r_t^1$ is a submartingale.
\end{definition}

In the article 2023 the authors proved that a relative growth optimal strategy is survival and constructed a relative growth optimal strategy. Let us formulate their result in the form of the following lemma.


\begin{lemma}
\label{lemma1}
For each $t\ge 0$, consider the $\F_t\otimes \mathcal{B}(\R_+)\otimes
\mathcal{B}(\Delta_N)$-measurable function $L_t \colon \Omega
\times \R_+ \times \Delta_N \to \Delta_N$ defined by
\[
L_t^k(\omega,c,\lambda) = \E_t\biggl(\frac{c\lambda^k
X_{t+1}^n + Y_{t+1}^n}{\sum_{k=1}^K (c \lambda^k X_{t+1}^k + Y_{t+1}^k)}\biggr).
\]
Then there exists an $\F_t\otimes\mathcal{B}(\R_+)$-measurable function $\hat
\Lambda_t(\omega,c)$ with values in $\Delta_N$ such that
\begin{equation}
L(\omega,c,\hat \Lambda_t(\omega,c)) = \hat \Lambda_t(\omega,c)\
\text{for all $\omega,c$}.\label{lambda-hat}
\end{equation}
And this strategy $\hat \Lambda$ is relative growth optimal (and hence, survival).
\end{lemma}

\section{Main result}
We will show that in affine model the set of strategies $\bar\lambda$ form a *-weak (but not strong) compact set 
and on this set the above-mentioned function $L_t$ is strongly (but not *-weakly) continuous. 
So Tikhonov's fixed point theorem does not guarantee the existence of a fixed point. 
Nevertheless, in article 2023 it was proved that the fixed point exists.

Let's interpret strategy $\bar\lambda$ as random variables from $L^{\infty}$ 
(defined on $\Omega \otimes \R_+$) and having values in simplex $\Delta^k$.
Then by Banah-Alaogly's theorem we obtain that the set of such random variables in a compact in *-weak topology.
Indeed, this theorem states that a closed bounded set in a dual space to some normed vector 
space is a compact in *-weak topology. Let's check boundedness and closedness for our set of random variables.

1) It is clearly bounded, since $||\xi||_{\infty}=1$, because $|\xi_1| +\dots+ |\xi_k| =1$.

2) Now let's show that this set is closed. To do this, we will show that it contains all its limit points
 (limit is taken in *-weak sence).

 So we take $\bar\xi_t$ satisfying $\sum_{k=1}^K \xi_t^k =1$ and $\xi_t^k \geq 0$ for all $k=1,\dots,K$.
 And let's denote the limit random variable as $\bar\xi$: $\bar\xi_t \to^{*} \bar\xi$.

 2a) Let's show that $\sum_{k=1}^K \xi^k =1$.
 We have that $\xi_t \to^{*} \xi$ and $\bar 1 \cdot \bar \xi_t =1$.\\
 Let's consider a new random variable $\eta:=I\left(\bar 1\cdot \xi \geq 1\right) \cdot \bar 1$.\\
 Then, due to *-weak convergence definition, 
 we have $\E\left(\eta \cdot \xi_t \right) \to \E\left( \eta \cdot \xi\right)$.\\
 It means 
 $\P(\bar 1 \cdot \xi \geq 1 ) \to \E\left( I\left( \bar 1 \cdot \xi \geq 1 \right) \cdot \bar 1 \cdot \xi \right) 
 \geq \P\left( \bar 1 \cdot \xi \geq 1\right)$.\\

 So $\bar 1 \cdot \xi =1$ on set, where $\bar 1 \cdot \xi \geq 1$.\\
 Analogously considering $\eta:=I\left(\bar 1\cdot \xi \leq 1\right) \cdot \bar 1$, we obtain that 
 $\bar 1 \cdot \xi =1$ on set, where $\bar 1 \cdot \xi \leq 1$.\\
 So $\bar 1 \cdot \xi =1$ on the whole set.

 2b) Let's show that $\xi^k \geq 0$ for all $k=1,\dots,K$.\\
 Let's consider a new random variable 
 $\eta:=I\left(\bar e_k \cdot \xi \leq 0\right) \cdot \bar e_k$.\\

 Then, due to *-weak convergence definition, 
 we have $\E\left(\eta \cdot \xi_t \right) \to \E\left( \eta \cdot \xi\right)$.\\
 It means 
 $\E \left( I \left( \bar e_k \cdot \xi \leq 0 \right) \cdot e_k \cdot \xi_t \right) \to
 \E\left( I\left( \bar e_k \cdot \xi \leq 0 \right) \cdot \bar e_k \cdot \xi \right)$.\\

Lefthandside is $\geq 0$ while righthandside is $\leq 0$.
Hence, $I\left(\bar e_k \cdot \xi \geq 0\right) \equiv 1$.\\

So we have proved that our set is a compact in *-weak topology. 
But now we will show that $L_t$ is not necessarily continuous in *-weak toplogy even for two assets.

Recall that 
\[
L_t^k(\omega,c,\lambda) = \E_t\biggl(\frac{c\lambda^k
X_{t+1}^n + Y_{t+1}^n}{\sum_{k=1}^K (c \lambda^k X_{t+1}^k + Y_{t+1}^k)}\biggr).
\]

Let's put $X^1 \equiv 2, X^2 \equiv 1, Y^1\equiv 1, Y^2 \equiv 1, C\equiv 1$. 
This means that there are two assets, and total capital is always 1.
Then $L^1 = \frac{2\lambda + 1}{2\lambda + 1 +(1-\lambda) + 1} 
=\frac{2\lambda +1 }{\lambda +3 } = 2-\frac{5}{\lambda + 3}$.

Let's show that $f(\lambda) =frac{5}{\lambda + 3}$ is not *-weak continuous.
Take $\lambda_t \to^{*} \lambda$, where $\lambda_t =
\sum_{i=0}^{s^{t-1}-1} I\left( \frac{2i}{2^t},  \frac{2i+1}{2^t}\right)$.

And we want to prove that $\forall \xi \in L_1: \E f(\lambda_t)\xi \to^{*} \E f(\lambda)\xi$.

But we can see that even for $\xi \equiv 1$: $ \E f(\lambda_t) \not\to \E f(\lambda)$.

Indeed, $\E f(\lambda_t) = 0.5( \frac{5}{3} +\frac{5}{4}) = \frac{35}{24} \not\to \E f(\lambda)=\frac{5}{0.5+3}$.

So we have proven that $f(\lambda) =frac{5}{\lambda + 3}$ is not *-weak continuous and hence $L_t$ is not *-weak continuous.

And at the same time we have found a contradiction to 
the fact that our set of random variables is strongly (sequentially) compact in $L_{\infty}$, 
because we have found a sequence, that is *-weak convergent, but does not a strongly convergent subsequence.\\
Indeed, let's take sequence $\lambda_t$. By constraction, for all $t,s$ 
we have $|| \L^1_t - \L^1_s ||_{L_{\infty}} \equiv 1$. 
But space $L_{\infty}$ is complete, which means that if any sequence converges, it must be fundamental.
But our sequence is not fundamental.

So we have proven that our set of strategies $\bar\lambda$ form a *-weak (but not strong) compact set 
and on this set the above-mentioned function $L_t$ is strongly (but not *-weakly) continuous. 
Tikhonov's fixed point theorem says that if we have a strong compact and strong continuity, 
then there will be a fixed point. In our case we have strong continuity but weak compact - and nevertheless there is a fixed point.


\end{document}
